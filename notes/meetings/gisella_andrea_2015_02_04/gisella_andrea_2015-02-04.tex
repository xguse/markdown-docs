\documentclass[letterpaper]{scrartcl}
\usepackage{lmodern}
\usepackage{amssymb,amsmath}
\usepackage{ifxetex,ifluatex}
\usepackage{fixltx2e} % provides \textsubscript
\ifnum 0\ifxetex 1\fi\ifluatex 1\fi=0 % if pdftex
  \usepackage[T1]{fontenc}
  \usepackage[utf8]{inputenc}
\else % if luatex or xelatex
  \ifxetex
    \usepackage{mathspec}
    \usepackage{xltxtra,xunicode}
  \else
    \usepackage{fontspec}
  \fi
  \defaultfontfeatures{Mapping=tex-text,Scale=MatchLowercase}
  \newcommand{\euro}{€}
\fi
% use upquote if available, for straight quotes in verbatim environments
\IfFileExists{upquote.sty}{\usepackage{upquote}}{}
% use microtype if available
\IfFileExists{microtype.sty}{%
\usepackage{microtype}
\UseMicrotypeSet[protrusion]{basicmath} % disable protrusion for tt fonts
}{}
\usepackage[margin=1in]{geometry}
\usepackage{graphicx}
\makeatletter
\def\maxwidth{\ifdim\Gin@nat@width>\linewidth\linewidth\else\Gin@nat@width\fi}
\def\maxheight{\ifdim\Gin@nat@height>\textheight\textheight\else\Gin@nat@height\fi}
\makeatother
% Scale images if necessary, so that they will not overflow the page
% margins by default, and it is still possible to overwrite the defaults
% using explicit options in \includegraphics[width, height, ...]{}
\setkeys{Gin}{width=\maxwidth,height=\maxheight,keepaspectratio}
\ifxetex
  \usepackage[setpagesize=false, % page size defined by xetex
              unicode=false, % unicode breaks when used with xetex
              xetex]{hyperref}
\else
  \usepackage[unicode=true]{hyperref}
\fi
\hypersetup{breaklinks=true,
            bookmarks=true,
            pdfauthor={Gus, Gisella, Andrea},
            pdftitle={Status of dead positives recovery},
            colorlinks=true,
            citecolor=blue,
            urlcolor=blue,
            linkcolor=magenta,
            pdfborder={0 0 0}}
\urlstyle{same}  % don't use monospace font for urls
\setlength{\parindent}{0pt}
\setlength{\parskip}{6pt plus 2pt minus 1pt}
\setlength{\emergencystretch}{3em}  % prevent overfull lines
\setcounter{secnumdepth}{5}

\title{Status of dead positives recovery\\\vspace{0.5em}{\large Meeting notes}}
\author{Gus, Gisella, Andrea}
\date{2015-02-04 (Wednesday)}
\usepackage[T1]{fontenc}
\usepackage{lxfonts}

\begin{document}
\maketitle

{
\hypersetup{linkcolor=black}
\setcounter{tocdepth}{3}
\tableofcontents
}
\section{Overview of Discussed}\label{overview-of-discussed}

\begin{itemize}
\itemsep1pt\parskip0pt\parsep0pt
\item
  Andrea's write up of current results

  \begin{itemize}
  \itemsep1pt\parskip0pt\parsep0pt
  \item
    \emph{need to be collated into one doc}
  \end{itemize}
\item
  Status of LD analysis and how to choose cut-off

  \begin{itemize}
  \itemsep1pt\parskip0pt\parsep0pt
  \item
    results of discussion with Mark
  \item
    success in generating expected behavior of mean LD behavior vs
    distance
  \item
    Gus's proposed idea to identify ``outlier'' snp pairs (\emph{see
    section below})
  \end{itemize}
\item
  First figure should have a map of population locations as a panel
\item
  Structure of paper \emph{(\textbf{mostly} unchanged since last
  meeting)}
\item
  plans for future
\end{itemize}

\section{Current structure of paper}\label{current-structure-of-paper}

\emph{~\textbf{mostly} unchanged since last meeting}:

\begin{enumerate}
\def\labelenumi{\arabic{enumi}.}
\itemsep1pt\parskip0pt\parsep0pt
\item
  Development of the base SNP set
\item
  Linkage Analysis

  \begin{enumerate}
  \def\labelenumii{\alph{enumii}.}
  \itemsep1pt\parskip0pt\parsep0pt
  \item
    \textbf{(new)} linkage-based grouping of contigs by physical
    proximity
  \end{enumerate}
\item
  Functional Annotation of filtered SNPs
\item
  Discussion

  \begin{itemize}
  \itemsep1pt\parskip0pt\parsep0pt
  \item
    establish the ability to do this scale of work in \emph{G. f.
    fuscipes}
  \item
    limits of the dataset as it now stands
  \item
    Never-the-less, hypotheses can be formulated and here they
    are\ldots{}
  \item
    \textbf{(new)} provides more information pertaining to the physical
    proximity of the supercontigs
  \end{itemize}
\end{enumerate}

\section{Gus's proposal to identify LD ``outlier''
snp-pairs}\label{guss-proposal-to-identify-ld-outlier-snp-pairs}

\begin{enumerate}
\def\labelenumi{\arabic{enumi}.}
\itemsep1pt\parskip0pt\parsep0pt
\item
  for each group of SNPs \(x\) bp apart: collect \(r^2\) from
  \(\pm \sim5\) bp distance window around \(x\):

  \begin{enumerate}
  \def\labelenumii{\alph{enumii}.}
  \itemsep1pt\parskip0pt\parsep0pt
  \item
    across genome
  \item
    across scaffold
  \end{enumerate}
\item
  calculate modified z-score (based on \emph{median absolute deviation}
  rather than standard deviation: \textbf{MAD is more robust than SD for
  HTS-type data})
\item
  flag any SNP-pair with \(z \geq 3.5\)
\item
  possibly randomize data and calculate FDR to evaluate performance.

  \begin{enumerate}
  \def\labelenumii{\alph{enumii}.}
  \itemsep1pt\parskip0pt\parsep0pt
  \item
    perhaps vary the window-size from step 1 to use FDR to chose
    window-size that minimizes FDR.
  \end{enumerate}
\end{enumerate}

\section{Current/future plans}\label{currentfuture-plans}

\textbf{Gisella:}

\begin{itemize}
\itemsep1pt\parskip0pt\parsep0pt
\item
  \texttt{{[}X{]}} email Washington group about methods of Seq prep and
  analysis
\item
  \texttt{{[}X{]}} email Aksoy group about linkage status of \emph{G. m.
  morsitans}
\end{itemize}

\textbf{Gus and Andrea:}

\begin{itemize}
\itemsep1pt\parskip0pt\parsep0pt
\item
  \texttt{{[} {]}} add methods in ``final'' style to growing document
\item
  \texttt{{[} {]}} place document in shared location
\end{itemize}

\textbf{Gus:}

\begin{itemize}
\itemsep1pt\parskip0pt\parsep0pt
\item
  \texttt{{[} {]}} add LD methods and results to document
\item
  \texttt{{[} {]}} generate population-location map with zoom out to all
  Uganda
\item
  \texttt{{[}-waiting-{]}} upon decision of which near-by genes are
  ``interesting'' send summary of info known about them to Aksoy group
  for ideas.
\end{itemize}

\end{document}
