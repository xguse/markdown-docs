\documentclass[letterpaper]{scrartcl}
\usepackage{lmodern}
\usepackage{amssymb,amsmath}
\usepackage{ifxetex,ifluatex}
\usepackage{fixltx2e} % provides \textsubscript
\ifnum 0\ifxetex 1\fi\ifluatex 1\fi=0 % if pdftex
  \usepackage[T1]{fontenc}
  \usepackage[utf8]{inputenc}
\else % if luatex or xelatex
  \ifxetex
    \usepackage{mathspec}
    \usepackage{xltxtra,xunicode}
  \else
    \usepackage{fontspec}
  \fi
  \defaultfontfeatures{Mapping=tex-text,Scale=MatchLowercase}
  \newcommand{\euro}{€}
\fi
% use upquote if available, for straight quotes in verbatim environments
\IfFileExists{upquote.sty}{\usepackage{upquote}}{}
% use microtype if available
\IfFileExists{microtype.sty}{%
\usepackage{microtype}
\UseMicrotypeSet[protrusion]{basicmath} % disable protrusion for tt fonts
}{}
\usepackage[margin=1in]{geometry}
\usepackage{graphicx}
\makeatletter
\def\maxwidth{\ifdim\Gin@nat@width>\linewidth\linewidth\else\Gin@nat@width\fi}
\def\maxheight{\ifdim\Gin@nat@height>\textheight\textheight\else\Gin@nat@height\fi}
\makeatother
% Scale images if necessary, so that they will not overflow the page
% margins by default, and it is still possible to overwrite the defaults
% using explicit options in \includegraphics[width, height, ...]{}
\setkeys{Gin}{width=\maxwidth,height=\maxheight,keepaspectratio}
\ifxetex
  \usepackage[setpagesize=false, % page size defined by xetex
              unicode=false, % unicode breaks when used with xetex
              xetex]{hyperref}
\else
  \usepackage[unicode=true]{hyperref}
\fi
\hypersetup{breaklinks=true,
            bookmarks=true,
            pdfauthor={Gus, Gisella, Kirstin},
            pdftitle={Status of dead positives recovery},
            colorlinks=true,
            citecolor=blue,
            urlcolor=blue,
            linkcolor=magenta,
            pdfborder={0 0 0}}
\urlstyle{same}  % don't use monospace font for urls
\setlength{\parindent}{0pt}
\setlength{\parskip}{6pt plus 2pt minus 1pt}
\setlength{\emergencystretch}{3em}  % prevent overfull lines
\setcounter{secnumdepth}{5}

\title{Status of dead positives recovery\\\vspace{0.5em}{\large Meeting notes}}
\author{Gus, Gisella, Kirstin}
\date{2015-01-14 (Wednesday)}
\usepackage[T1]{fontenc}
\usepackage{lxfonts}

\begin{document}
\maketitle

{
\hypersetup{linkcolor=black}
\setcounter{tocdepth}{3}
\tableofcontents
}
\section{Discussed}\label{discussed}

\subsection{Changes to screen
procedures}\label{changes-to-screen-procedures}

\begin{itemize}
\itemsep1pt\parskip0pt\parsep0pt
\item
  pooling rxns abandoned for individual PCR
\item
  gels run as pools instead
\item
  this is probably best since even with individuals, the sensitivity is
  not great
\item
  added additional PCR rxn using product of first as template to
  increase sensitivity
\end{itemize}

\subsection{Results}\label{results}

\begin{itemize}
\itemsep1pt\parskip0pt\parsep0pt
\item
  2 plates tested (\textasciitilde{}200 individuals)
\item
  Plate 1:

  \begin{itemize}
  \itemsep1pt\parskip0pt\parsep0pt
  \item
    \textbf{Village:} AKA
  \item
    \textbf{Results:} 3 positives detected
  \end{itemize}
\item
  Plate 2:

  \begin{itemize}
  \itemsep1pt\parskip0pt\parsep0pt
  \item
    \textbf{Village:} Mix of the high infection villages
  \item
    \textbf{Results:} 1 positive detected
  \end{itemize}
\end{itemize}

\subsection{Robert's Work}\label{roberts-work}

\begin{itemize}
\itemsep1pt\parskip0pt\parsep0pt
\item
  will be focused on MicroSats
\item
  should only need legs as genetic material
\item
  will use the Zygem (spelling?) DNA extraction kit
\item
  some talk about stability of DNA from this kit but will use it anyway
  and test quality when Robert begins work
\end{itemize}

\subsection{Undergrad}\label{undergrad}

\begin{itemize}
\itemsep1pt\parskip0pt\parsep0pt
\item
  will work on extracting DNA for Robert's MicroSat work ahead of his
  arrival
\item
  Kirstin will train
\item
  Needs to be briefed on over-all project for a talk she must give
\item
  Gisella will have her apply for the small Alumni grant
\end{itemize}

\section{Current/future plans}\label{currentfuture-plans}

\subsection{Dead positives screen}\label{dead-positives-screen}

\begin{itemize}
\itemsep1pt\parskip0pt\parsep0pt
\item
  Plate 3:

  \begin{itemize}
  \itemsep1pt\parskip0pt\parsep0pt
  \item
    \textbf{Village:} ? not sure. Will check.
  \item
    \textbf{Status:} waiting for extraction?
  \end{itemize}
\item
  Plate 4:

  \begin{itemize}
  \itemsep1pt\parskip0pt\parsep0pt
  \item
    \textbf{Village:} Mostly ACA
  \item
    \textbf{Status:} needs \textasciitilde{}15 more dissections
  \end{itemize}
\item
  Pick out deads from the ``most infected'' areas of the \emph{Spring
  2014} data for next round of screens.
\end{itemize}

\subsection{Robert's stuff}\label{roberts-stuff}

\begin{itemize}
\itemsep1pt\parskip0pt\parsep0pt
\item
  Pick out 26 flies (13 M, 13 F) from each area we want to look at for
  Robert's work in March
\item
  make sure we have an updated map with all the villages from
  \emph{Spring and Summer 2014}
\item
  Meet with Gisella to pick out which locations Robert's data will come
  from while looking at the updated map.
\end{itemize}

\end{document}
