\documentclass[letterpaper]{scrartcl}
\usepackage{lmodern}
\usepackage{amssymb,amsmath}
\usepackage{ifxetex,ifluatex}
\usepackage{fixltx2e} % provides \textsubscript
\ifnum 0\ifxetex 1\fi\ifluatex 1\fi=0 % if pdftex
  \usepackage[T1]{fontenc}
  \usepackage[utf8]{inputenc}
\else % if luatex or xelatex
  \ifxetex
    \usepackage{mathspec}
    \usepackage{xltxtra,xunicode}
  \else
    \usepackage{fontspec}
  \fi
  \defaultfontfeatures{Mapping=tex-text,Scale=MatchLowercase}
  \newcommand{\euro}{€}
\fi
% use upquote if available, for straight quotes in verbatim environments
\IfFileExists{upquote.sty}{\usepackage{upquote}}{}
% use microtype if available
\IfFileExists{microtype.sty}{%
\usepackage{microtype}
\UseMicrotypeSet[protrusion]{basicmath} % disable protrusion for tt fonts
}{}
\usepackage[margin=1in]{geometry}
\usepackage{graphicx}
\makeatletter
\def\maxwidth{\ifdim\Gin@nat@width>\linewidth\linewidth\else\Gin@nat@width\fi}
\def\maxheight{\ifdim\Gin@nat@height>\textheight\textheight\else\Gin@nat@height\fi}
\makeatother
% Scale images if necessary, so that they will not overflow the page
% margins by default, and it is still possible to overwrite the defaults
% using explicit options in \includegraphics[width, height, ...]{}
\setkeys{Gin}{width=\maxwidth,height=\maxheight,keepaspectratio}
\ifxetex
  \usepackage[setpagesize=false, % page size defined by xetex
              unicode=false, % unicode breaks when used with xetex
              xetex]{hyperref}
\else
  \usepackage[unicode=true]{hyperref}
\fi
\hypersetup{breaklinks=true,
            bookmarks=true,
            pdfauthor={GusD, GisellaC, RobH},
            pdftitle={G. pallidipes sample database summary and extractions},
            colorlinks=true,
            citecolor=blue,
            urlcolor=blue,
            linkcolor=magenta,
            pdfborder={0 0 0}}
\urlstyle{same}  % don't use monospace font for urls
\setlength{\parindent}{0pt}
\setlength{\parskip}{6pt plus 2pt minus 1pt}
\setlength{\emergencystretch}{3em}  % prevent overfull lines
\setcounter{secnumdepth}{5}

\title{\emph{G. pallidipes} sample database summary and extractions}
\author{GusD, GisellaC, RobH}
\date{2015-02-12 (Thursday)}
\usepackage[T1]{fontenc}
\usepackage{lxfonts}

% blockquote


\begin{document}
\maketitle

{
\hypersetup{linkcolor=black}
\setcounter{tocdepth}{3}
\tableofcontents
}
\section{Overview of discussed}\label{overview-of-discussed}

\subsection{Message to Eliot}\label{message-to-eliot}

\begin{itemize}
\itemsep1pt\parskip0pt\parsep0pt
\item
  email asking about:

  \begin{itemize}
  \itemsep1pt\parskip0pt\parsep0pt
  \item
    GEO COORDINATES for ambiguous location names

    \begin{itemize}
    \itemsep1pt\parskip0pt\parsep0pt
    \item
      include pictures of samples/boxes
    \end{itemize}
  \item
    Samples/boxes listed on the PDF manifest but missing once shipment
    received
  \end{itemize}
\item
  Drafted by \textbf{{[}Rob{]}} and sent to Gisella
\item
  Passed on to destination by \textbf{{[}Gisella{]}}
\end{itemize}

\subsection{Sample catalog summaries}\label{sample-catalog-summaries}

\subsubsection{Next iteration}\label{next-iteration}

\begin{itemize}
\itemsep1pt\parskip0pt\parsep0pt
\item
  data types:

  \begin{itemize}
  \itemsep1pt\parskip0pt\parsep0pt
  \item
    location
  \item
    symbols when present (\emph{I assume you mean location symbol?})
  \item
    number of individuals
  \item
    date range
  \item
    is tissue?
  \item
    is extraction?
  \item
    analysis status
  \end{itemize}
\item
  will be done in \texttt{python} for increased flexibility by
  \textbf{{[}Gus{]}}
\end{itemize}

\subsubsection{Cold-room/freezer
samples}\label{cold-roomfreezer-samples}

\begin{itemize}
\itemsep1pt\parskip0pt\parsep0pt
\item
  \textbf{Important Note:} extracted samples from freezer are sourced
  from cold-room tissues
\item
  read Johnson Ouma paper for:

  \begin{itemize}
  \itemsep1pt\parskip0pt\parsep0pt
  \item
    protocol starting points
  \item
    MicroSats
  \item
    mDNA
  \end{itemize}
\item
  First Steps for \textbf{{[}Rob{]}}:

  \begin{itemize}
  \itemsep1pt\parskip0pt\parsep0pt
  \item
    shadow Kirsten to learn extractions
  \item
    take 5 samples with legs for extraction
  \item
    test extractions with PCR:

    \begin{itemize}
    \itemsep1pt\parskip0pt\parsep0pt
    \item
      ITS if available or can use \emph{G. f. fuscipes} primers

      \begin{itemize}
      \itemsep1pt\parskip0pt\parsep0pt
      \item
        find out if this will work
      \end{itemize}
    \item
      MicroSat primers from Ouma paper or otherwise \textbf{{[}being
      obtained by Kirstin{]}}
    \end{itemize}
  \end{itemize}
\end{itemize}

\subsubsection{Nonsensical/ambiguous location
names}\label{nonsensicalambiguous-location-names}

\begin{itemize}
\itemsep1pt\parskip0pt\parsep0pt
\item
  \textbf{{[}Rob{]}} will locate and bring to \textbf{{[}Gisella{]}} for
  help
\end{itemize}

\subsection{Questions for Gisella}\label{questions-for-gisella}

\subsubsection{Which extraction kit for
legs?}\label{which-extraction-kit-for-legs}

\begin{itemize}
\itemsep1pt\parskip0pt\parsep0pt
\item
  Same as for midguts
\item
  Kirstin is trouble shooting it currently for legs
\end{itemize}

\end{document}
