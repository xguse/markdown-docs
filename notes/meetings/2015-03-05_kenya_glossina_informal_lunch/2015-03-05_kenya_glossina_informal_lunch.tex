\documentclass[letterpaper]{scrartcl}
\usepackage{lmodern}
\usepackage{amssymb,amsmath}
\usepackage{ifxetex,ifluatex}
\usepackage{fixltx2e} % provides \textsubscript
\ifnum 0\ifxetex 1\fi\ifluatex 1\fi=0 % if pdftex
  \usepackage[T1]{fontenc}
  \usepackage[utf8]{inputenc}
\else % if luatex or xelatex
  \ifxetex
    \usepackage{mathspec}
    \usepackage{xltxtra,xunicode}
  \else
    \usepackage{fontspec}
  \fi
  \defaultfontfeatures{Mapping=tex-text,Scale=MatchLowercase}
  \newcommand{\euro}{€}
\fi
% use upquote if available, for straight quotes in verbatim environments
\IfFileExists{upquote.sty}{\usepackage{upquote}}{}
% use microtype if available
\IfFileExists{microtype.sty}{%
\usepackage{microtype}
\UseMicrotypeSet[protrusion]{basicmath} % disable protrusion for tt fonts
}{}
\usepackage[margin=1in]{geometry}
\usepackage{longtable,booktabs}
\usepackage{graphicx}
\makeatletter
\def\maxwidth{\ifdim\Gin@nat@width>\linewidth\linewidth\else\Gin@nat@width\fi}
\def\maxheight{\ifdim\Gin@nat@height>\textheight\textheight\else\Gin@nat@height\fi}
\makeatother
% Scale images if necessary, so that they will not overflow the page
% margins by default, and it is still possible to overwrite the defaults
% using explicit options in \includegraphics[width, height, ...]{}
\setkeys{Gin}{width=\maxwidth,height=\maxheight,keepaspectratio}
\ifxetex
  \usepackage[setpagesize=false, % page size defined by xetex
              unicode=false, % unicode breaks when used with xetex
              xetex]{hyperref}
\else
  \usepackage[unicode=true]{hyperref}
\fi
\hypersetup{breaklinks=true,
            bookmarks=true,
            pdfauthor={GusD,},
            pdftitle={Glossina informal lunch},
            colorlinks=true,
            citecolor=blue,
            urlcolor=blue,
            linkcolor=magenta,
            pdfborder={0 0 0}}
\urlstyle{same}  % don't use monospace font for urls
\setlength{\parindent}{0pt}
\setlength{\parskip}{6pt plus 2pt minus 1pt}
\setlength{\emergencystretch}{3em}  % prevent overfull lines
\setcounter{secnumdepth}{5}

\title{Glossina informal lunch}
\author{GusD,}
\date{2015-03-05 (Thursday)}
\usepackage[T1]{fontenc}
\usepackage{lxfonts}

% blockquote


\begin{document}
\maketitle

{
\hypersetup{linkcolor=black}
\setcounter{tocdepth}{3}
\tableofcontents
}
\section{Interest in Uganda}\label{interest-in-uganda}

\begin{itemize}
\itemsep1pt\parskip0pt\parsep0pt
\item
  aside from pure academic curiosity, there are real public health
  reasons to do work here
\item
  Convergence of the chronic and acute forms of Sleeping Sickness is
  imminent
\item
  Currently \emph{G. f. fuscipes} populations exist in semi-isolation
  due likely to water features
\item
  Yet cases of the two forms of sickness, while rare indicate that at
  least the parasites are moving closer to a situation where diagnosis
  (already difficult) will become even harder.
\item
  important because treatments are different for the two forms
\end{itemize}

\subsection{Some questions about the
Convergence}\label{some-questions-about-the-convergence}

\begin{itemize}
\itemsep1pt\parskip0pt\parsep0pt
\item
  what can we expect from \emph{G. f. fuscipes} populations when they
  encounter the \textbf{other} parasite-type?
\item
  can we identify the corridors of gene-flow between

  \begin{itemize}
  \itemsep1pt\parskip0pt\parsep0pt
  \item
    the sub populations
  \item
    the larger north/south divide?
  \end{itemize}
\item
  how can we use this information to inform vector control strategies?
\item
  how long has the north/south divide existed and what is/was the main
  cause
\end{itemize}

\subsection{Positives are in SHORT
supply}\label{positives-are-in-short-supply}

\begin{itemize}
\itemsep1pt\parskip0pt\parsep0pt
\item
  VERY SHORT
\end{itemize}

\section{Collection numbers}\label{collection-numbers}

\subsection{how many villages have we sampled in
2014?}\label{how-many-villages-have-we-sampled-in-2014}

\begin{itemize}
\itemsep1pt\parskip0pt\parsep0pt
\item
  47
\end{itemize}

\subsection{total files collected}\label{total-files-collected}

\begin{itemize}
\itemsep1pt\parskip0pt\parsep0pt
\item
  6867
\end{itemize}

\subsection{total flies dissected and assessed for infection of any
Tryps}\label{total-flies-dissected-and-assessed-for-infection-of-any-tryps}

\begin{itemize}
\itemsep1pt\parskip0pt\parsep0pt
\item
  3699
\end{itemize}

\subsection{positives}\label{positives}

\begin{itemize}
\itemsep1pt\parskip0pt\parsep0pt
\item
  \textbf{total:} 85
\item
  \textbf{positive rate (overall):} 2.3\%
\item
  \textbf{positive rates in best villages:}
\end{itemize}

\begin{longtable}[c]{@{}ll@{}}
\toprule
Village & positive (\%)\tabularnewline
\midrule
\endhead
OCU & 6.321839\tabularnewline
OD & 6.217617\tabularnewline
OCA & 3.832753\tabularnewline
ACA & 2.290076\tabularnewline
AMI & 2.222222\tabularnewline
DUK & 1.626016\tabularnewline
AKA & 1.326260\tabularnewline
GAN & 1.265823\tabularnewline
APU & 1.041667\tabularnewline
UWA & 0.632911\tabularnewline
\bottomrule
\end{longtable}

\section{Methods}\label{methods}

\begin{itemize}
\itemsep1pt\parskip0pt\parsep0pt
\item
  past has focused on MicroSat

  \begin{itemize}
  \itemsep1pt\parskip0pt\parsep0pt
  \item
    less DNA needed
  \end{itemize}
\item
  moving to ddRAD for more SNPs
\end{itemize}

\section{Collection protocol}\label{collection-protocol}

\begin{itemize}
\itemsep1pt\parskip0pt\parsep0pt
\item
  created and distributed
\item
  to help standardize problems with curration
\end{itemize}

\section{Database/sample tracking
efforts}\label{databasesample-tracking-efforts}

\begin{itemize}
\itemsep1pt\parskip0pt\parsep0pt
\item
  you know this
\end{itemize}

\end{document}
