\documentclass[letterpaper]{scrartcl}
\usepackage{lmodern}
\usepackage{amssymb,amsmath}
\usepackage{ifxetex,ifluatex}
\usepackage{fixltx2e} % provides \textsubscript
\ifnum 0\ifxetex 1\fi\ifluatex 1\fi=0 % if pdftex
  \usepackage[T1]{fontenc}
  \usepackage[utf8]{inputenc}
\else % if luatex or xelatex
  \ifxetex
    \usepackage{mathspec}
    \usepackage{xltxtra,xunicode}
  \else
    \usepackage{fontspec}
  \fi
  \defaultfontfeatures{Mapping=tex-text,Scale=MatchLowercase}
  \newcommand{\euro}{€}
\fi
% use upquote if available, for straight quotes in verbatim environments
\IfFileExists{upquote.sty}{\usepackage{upquote}}{}
% use microtype if available
\IfFileExists{microtype.sty}{%
\usepackage{microtype}
\UseMicrotypeSet[protrusion]{basicmath} % disable protrusion for tt fonts
}{}
\usepackage[margin=1in]{geometry}
\usepackage{graphicx}
\makeatletter
\def\maxwidth{\ifdim\Gin@nat@width>\linewidth\linewidth\else\Gin@nat@width\fi}
\def\maxheight{\ifdim\Gin@nat@height>\textheight\textheight\else\Gin@nat@height\fi}
\makeatother
% Scale images if necessary, so that they will not overflow the page
% margins by default, and it is still possible to overwrite the defaults
% using explicit options in \includegraphics[width, height, ...]{}
\setkeys{Gin}{width=\maxwidth,height=\maxheight,keepaspectratio}
\ifxetex
  \usepackage[setpagesize=false, % page size defined by xetex
              unicode=false, % unicode breaks when used with xetex
              xetex]{hyperref}
\else
  \usepackage[unicode=true]{hyperref}
\fi
\hypersetup{breaklinks=true,
            bookmarks=true,
            pdfauthor={Gus, Kirstin},
            pdftitle={Status of Positive Recovery from Dead Flies},
            colorlinks=true,
            citecolor=blue,
            urlcolor=blue,
            linkcolor=magenta,
            pdfborder={0 0 0}}
\urlstyle{same}  % don't use monospace font for urls
\setlength{\parindent}{0pt}
\setlength{\parskip}{6pt plus 2pt minus 1pt}
\setlength{\emergencystretch}{3em}  % prevent overfull lines
\setcounter{secnumdepth}{5}

\title{Status of Positive Recovery from Dead Flies}
\author{Gus, Kirstin}
\date{2015-01-08 (Thursday)}
\usepackage{bbding}
\usepackage[T1]{fontenc}
\usepackage{lxfonts}

\begin{document}
\maketitle

{
\hypersetup{linkcolor=black}
\setcounter{tocdepth}{3}
\tableofcontents
}
\section{Discussed}\label{discussed}

\subsection{Change in pooling}\label{change-in-pooling}

\begin{itemize}
\itemsep1pt\parskip0pt\parsep0pt
\item
  Pooling was abandoned for the reaction and moved to the gel/detection
  stage
\item
  PCR done on individuals
\item
  pooled PCR results run on gel for examination
\item
  This is probably good as Serap mentioned she did not like the pooled
  idea bc of low DNA abundance
\end{itemize}

\subsection{Results}\label{results}

\begin{itemize}
\itemsep1pt\parskip0pt\parsep0pt
\item
  2 plates tested (\textasciitilde{}200 individuals)
\item
  Plate 1:

  \begin{itemize}
  \itemsep1pt\parskip0pt\parsep0pt
  \item
    \textbf{Village:} AKA
  \item
    \textbf{Results:} no positives detected
  \end{itemize}
\item
  Plate 2:

  \begin{itemize}
  \itemsep1pt\parskip0pt\parsep0pt
  \item
    \textbf{Village:} Mix of the high infection villages
  \item
    \textbf{Results:} no positives detected
  \end{itemize}
\end{itemize}

\subsection{Observations}\label{observations}

\begin{itemize}
\itemsep1pt\parskip0pt\parsep0pt
\item
  Control rxns with \textbf{FLY} primers show expected results
\item
  In control rxn with 100 ng of DNA extracted from known positive flies,
  she \textbf{still} can barely see the band when using the
  \textbf{TRYP} primers
\item
  It is possible that some \textbf{false negatives} exist in these data
\item
  Further supports abandoning the pooled PCR rxns for individual rxns
\end{itemize}

\section{Current/future plans}\label{currentfuture-plans}

\begin{itemize}
\itemsep1pt\parskip0pt\parsep0pt
\item
  she plans to repeat at least a subset of the rxns using the first rxn
  as template
\item
  I feel that it would be interesting to see the results of this but
  fear that it may introduce \textbf{false positives} if not confirmed
  with more stringent conditions.
\item
  such confirmation however is cheap in both time and cost so I think
  its definitely worth doing.
\item
  more plates are on deck for screening as well
\end{itemize}

\subsection{Estimated time to
completion}\label{estimated-time-to-completion}

\begin{itemize}
\itemsep1pt\parskip0pt\parsep0pt
\item
  two weeks depending on facility load
\end{itemize}

\end{document}
