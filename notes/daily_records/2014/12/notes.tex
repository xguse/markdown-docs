\documentclass[letterpaper]{scrartcl}
\usepackage{concmath}
\usepackage{amssymb,amsmath}
\usepackage{ifxetex,ifluatex}
\usepackage{fixltx2e} % provides \textsubscript
\ifnum 0\ifxetex 1\fi\ifluatex 1\fi=0 % if pdftex
  \usepackage[T1]{fontenc}
  \usepackage[utf8]{inputenc}
\else % if luatex or xelatex
  \ifxetex
    \usepackage{mathspec}
    \usepackage{xltxtra,xunicode}
  \else
    \usepackage{fontspec}
  \fi
  \defaultfontfeatures{Mapping=tex-text,Scale=MatchLowercase}
  \newcommand{\euro}{€}
\fi
% use upquote if available, for straight quotes in verbatim environments
\IfFileExists{upquote.sty}{\usepackage{upquote}}{}
% use microtype if available
\IfFileExists{microtype.sty}{%
\usepackage{microtype}
\UseMicrotypeSet[protrusion]{basicmath} % disable protrusion for tt fonts
}{}
\usepackage{color}
\usepackage{fancyvrb}
\newcommand{\VerbBar}{|}
\newcommand{\VERB}{\Verb[commandchars=\\\{\}]}
\DefineVerbatimEnvironment{Highlighting}{Verbatim}{commandchars=\\\{\}}
% Add ',fontsize=\small' for more characters per line
\newenvironment{Shaded}{}{}
\newcommand{\KeywordTok}[1]{\textcolor[rgb]{0.00,0.44,0.13}{\textbf{{#1}}}}
\newcommand{\DataTypeTok}[1]{\textcolor[rgb]{0.56,0.13,0.00}{{#1}}}
\newcommand{\DecValTok}[1]{\textcolor[rgb]{0.25,0.63,0.44}{{#1}}}
\newcommand{\BaseNTok}[1]{\textcolor[rgb]{0.25,0.63,0.44}{{#1}}}
\newcommand{\FloatTok}[1]{\textcolor[rgb]{0.25,0.63,0.44}{{#1}}}
\newcommand{\CharTok}[1]{\textcolor[rgb]{0.25,0.44,0.63}{{#1}}}
\newcommand{\StringTok}[1]{\textcolor[rgb]{0.25,0.44,0.63}{{#1}}}
\newcommand{\CommentTok}[1]{\textcolor[rgb]{0.38,0.63,0.69}{\textit{{#1}}}}
\newcommand{\OtherTok}[1]{\textcolor[rgb]{0.00,0.44,0.13}{{#1}}}
\newcommand{\AlertTok}[1]{\textcolor[rgb]{1.00,0.00,0.00}{\textbf{{#1}}}}
\newcommand{\FunctionTok}[1]{\textcolor[rgb]{0.02,0.16,0.49}{{#1}}}
\newcommand{\RegionMarkerTok}[1]{{#1}}
\newcommand{\ErrorTok}[1]{\textcolor[rgb]{1.00,0.00,0.00}{\textbf{{#1}}}}
\newcommand{\NormalTok}[1]{{#1}}
\usepackage{graphicx}
\makeatletter
\def\maxwidth{\ifdim\Gin@nat@width>\linewidth\linewidth\else\Gin@nat@width\fi}
\def\maxheight{\ifdim\Gin@nat@height>\textheight\textheight\else\Gin@nat@height\fi}
\makeatother
% Scale images if necessary, so that they will not overflow the page
% margins by default, and it is still possible to overwrite the defaults
% using explicit options in \includegraphics[width, height, ...]{}
\setkeys{Gin}{width=\maxwidth,height=\maxheight,keepaspectratio}
\ifxetex
  \usepackage[setpagesize=false, % page size defined by xetex
              unicode=false, % unicode breaks when used with xetex
              xetex]{hyperref}
\else
  \usepackage[unicode=true]{hyperref}
\fi
\hypersetup{breaklinks=true,
            bookmarks=true,
            pdfauthor={},
            pdftitle={Daily Records},
            colorlinks=true,
            citecolor=blue,
            urlcolor=blue,
            linkcolor=magenta,
            pdfborder={0 0 0}}
\urlstyle{same}  % don't use monospace font for urls
\setlength{\parindent}{0pt}
\setlength{\parskip}{6pt plus 2pt minus 1pt}
\setlength{\emergencystretch}{3em}  % prevent overfull lines
\setcounter{secnumdepth}{5}

\title{Daily Records\\\vspace{0.5em}{\large Caccone PostDoc}}
\date{December, 2014}

\begin{document}
\maketitle

{
\hypersetup{linkcolor=black}
\setcounter{tocdepth}{3}
\tableofcontents
}
\section{2014-12-19}\label{section}

\subsection{Argot2 batch size
reduction:}\label{argot2-batch-size-reduction}

\begin{verbatim}
tags = [argot2, ddRAD58, python, spartan]
\end{verbatim}

\begin{itemize}
\itemsep1pt\parskip0pt\parsep0pt
\item
  dumping the whole proteome on the online server seems to have broken
  it.
\item
  I believe I will need to split the input up into around 3000 proteins
  per submission
\item
  this will need some python code (maybe added to \texttt{spartan}?)
\end{itemize}

\subsubsection{Problems:}\label{problems}

\begin{verbatim}
tags = [python, pip, virtualenvwrapper, py279, admin]
\end{verbatim}

\begin{itemize}
\itemsep1pt\parskip0pt\parsep0pt
\item
  cant run ipython notebook bc of py279 from py278 issues.
\item
  must reinstall most/all python dependencies.
\end{itemize}

\begin{center}\rule{0.5\linewidth}{\linethickness}\end{center}

\section{2014-12-21}\label{section-1}

\subsection{Argot2 batch size
reduction:}\label{argot2-batch-size-reduction-1}

\begin{itemize}
\itemsep1pt\parskip0pt\parsep0pt
\item
  see 2014-12-19
\item
  finished updating \texttt{py279} \texttt{python} requirements
\item
  \texttt{ipython} seems to work again
\end{itemize}

\subsubsection{Completed:}\label{completed}

\begin{itemize}
\itemsep1pt\parskip0pt\parsep0pt
\item
  writing filter functions for \texttt{blast} and \texttt{hmmer}
  outputs:

  \begin{itemize}
  \itemsep1pt\parskip0pt\parsep0pt
  \item
    \texttt{spartan/src/spartan/utils/blast/output.py}:

    \begin{itemize}
    \itemsep1pt\parskip0pt\parsep0pt
    \item
      \texttt{filter\_for\_argot(path, protein\_names)}
    \end{itemize}
  \item
    \texttt{spartan/src/spartan/utils/hmmer/output.py}:

    \begin{itemize}
    \itemsep1pt\parskip0pt\parsep0pt
    \item
      \texttt{filter\_for\_argot(path, protein\_names)}
    \end{itemize}
  \end{itemize}
\item
  created \texttt{sublime text 3} snippet to make new
  \texttt{ipython notebook} file with meta info and template.
\item
  wrote and tested split up logic:

  \begin{itemize}
  \itemsep1pt\parskip0pt\parsep0pt
  \item
    \texttt{spartan/src/spartan/utils/misc.py}:

    \begin{itemize}
    \itemsep1pt\parskip0pt\parsep0pt
    \item
      \texttt{split\_stream(stream, divisor)}
    \end{itemize}
  \end{itemize}
\end{itemize}

\subsubsection{Problems:}\label{problems-1}

\begin{itemize}
\itemsep1pt\parskip0pt\parsep0pt
\item
  coding the split up logic
\end{itemize}

\subsection{Working on:}\label{working-on}

\begin{itemize}
\itemsep1pt\parskip0pt\parsep0pt
\item
  split \texttt{blastp} and \texttt{hmmer} outputs by protein
  (\textasciitilde{}3000 per group)
\end{itemize}

\begin{center}\rule{0.5\linewidth}{\linethickness}\end{center}

\section{2014-12-22 (Monday)}\label{monday}

\subsection{Argot2 batch size
reduction:}\label{argot2-batch-size-reduction-2}

\begin{itemize}
\itemsep1pt\parskip0pt\parsep0pt
\item
  see 2014-12-21
\item
  finished splitting
\item
  files were zipped
\item
  file pairs were submitted to Argot2
\end{itemize}

\subsubsection{Files:}\label{files}

\begin{verbatim}
    Glossina-fuscipes-IAEA_PEPTIDES_GfusI1.0.hmmscan.zip
    Glossina-fuscipes-IAEA_PEPTIDES_GfusI1.1.hmmscan.zip
    Glossina-fuscipes-IAEA_PEPTIDES_GfusI1.2.hmmscan.zip
    Glossina-fuscipes-IAEA_PEPTIDES_GfusI1.3.hmmscan.zip
    Glossina-fuscipes-IAEA_PEPTIDES_GfusI1.4.hmmscan.zip
    Glossina-fuscipes-IAEA_PEPTIDES_GfusI1.5.hmmscan.zip
    Glossina-fuscipes-IAEA_PEPTIDES_GfusI1.6.hmmscan.zip
    Glossina-fuscipes-IAEA_PEPTIDES_GfusI1.7.hmmscan.zip
    Glossina-fuscipes-IAEA_PEPTIDES_GfusI1.8.hmmscan.zip
    Glossina-fuscipes-IAEA_PEPTIDES_GfusI1.9.hmmscan.zip

    Glossina-fuscipes-IAEA_PEPTIDES_GfusI1.union.0.blastp.zip
    Glossina-fuscipes-IAEA_PEPTIDES_GfusI1.union.1.blastp.zip
    Glossina-fuscipes-IAEA_PEPTIDES_GfusI1.union.2.blastp.zip
    Glossina-fuscipes-IAEA_PEPTIDES_GfusI1.union.3.blastp.zip
    Glossina-fuscipes-IAEA_PEPTIDES_GfusI1.union.4.blastp.zip
    Glossina-fuscipes-IAEA_PEPTIDES_GfusI1.union.5.blastp.zip
    Glossina-fuscipes-IAEA_PEPTIDES_GfusI1.union.6.blastp.zip
    Glossina-fuscipes-IAEA_PEPTIDES_GfusI1.union.7.blastp.zip
    Glossina-fuscipes-IAEA_PEPTIDES_GfusI1.union.8.blastp.zip
    Glossina-fuscipes-IAEA_PEPTIDES_GfusI1.union.9.blastp.zip
\end{verbatim}

\subsubsection{Next:}\label{next}

\begin{itemize}
\itemsep1pt\parskip0pt\parsep0pt
\item
  Collect the resulting analysis files into single repository.
\end{itemize}

\begin{center}\rule{0.5\linewidth}{\linethickness}\end{center}

\section{2014-12-23 (Tuesday)}\label{tuesday}

\subsection{Meeting with Dan}\label{meeting-with-dan}

\begin{itemize}
\itemsep1pt\parskip0pt\parsep0pt
\item
  \href{file:///home/gus/Dropbox/repos/git/markdown-docs/notes/meetings/dan-2014-12-23/dan-2014-12-23.md}{markdown-docs/notes/meetings/dan-2014-12-23/dan-2014-12-23.md}
\end{itemize}

\subsection{Argot2 batch size
reduction:}\label{argot2-batch-size-reduction-3}

\begin{itemize}
\itemsep1pt\parskip0pt\parsep0pt
\item
  see 2014-12-22
\item
  \href{http://localhost:8888/jupiter/notebooks/YALE/ddrad58/2014-12-23_create_argot2_functional_annotation_db_GfusI1.1_prerelease.ipynb}{2014-12-23\_create\_argot2\_functional\_annotation\_db\_GfusI1.1\_prerelease.ipynb}
\end{itemize}

\begin{center}\rule{0.5\linewidth}{\linethickness}\end{center}

\section{2014-12-24 (Wednesday)}\label{wednesday}

\subsection{Argot batch size
reduction}\label{argot-batch-size-reduction}

\begin{itemize}
\itemsep1pt\parskip0pt\parsep0pt
\item
  see 2014-12-23
\end{itemize}

\subsubsection{Storing in \texttt{Python}-friendly format
(\texttt{pandas.HDFStore})}\label{storing-in-python-friendly-format-pandas.hdfstore}

\begin{itemize}
\itemsep1pt\parskip0pt\parsep0pt
\item
  had to install \texttt{pytables}
\item
  test work well and data is stored
\item
  documentation:
  \href{http://localhost:8888/jupiter/notebooks/YALE/ddrad58/2014-12-24_store_argot2_functional_annotation_db_GfusI1.1_prerelease_as_HDF5.ipynb}{2014-12-24\_store\_argot2\_functional\_annotation\_db\_GfusI1.1\_prerelease\_as\_HDF5.ipynb}
\end{itemize}

\subsubsection{File location}\label{file-location}

\begin{verbatim}
louise/data/genomes/glossina_fuscipes/annotations/ \
functional/GfusI1.1_pre/argot2_out/argot_functional_annotations_ts150.h5
\end{verbatim}

\subsubsection{PROJECT COMPLETED}\label{project-completed}

``Argot batch size reduction'' project is now considered completed.

\subsection{Installing \texttt{PyTables}}\label{installing-pytables}

\textbf{First attempt failed due to cryptic or at \emph{least} slightly
misleading error about \texttt{numpy} and \texttt{numexpr}:}

\begin{Shaded}
\begin{Highlighting}[]
\NormalTok{$ }\KeywordTok{pip} \NormalTok{install git+https://github.com/PyTables/PyTables.git@v.3.1.1#egg=tables}
\end{Highlighting}
\end{Shaded}

\textbf{Tried installing \texttt{numexpr} directly with:}

\begin{Shaded}
\begin{Highlighting}[]
\NormalTok{$ }\KeywordTok{pip} \NormalTok{install numexpr}
\end{Highlighting}
\end{Shaded}

\textbf{Tried \texttt{PyTables} again:}

\begin{Shaded}
\begin{Highlighting}[]
\NormalTok{$ }\KeywordTok{pip} \NormalTok{install git+https://github.com/PyTables/PyTables.git@v.3.1.1#egg=tables}
\end{Highlighting}
\end{Shaded}

\subsection{T: Functional Annotations of genes near SNPs of
interest}\label{t-functional-annotations-of-genes-near-snps-of-interest}

\begin{itemize}
\itemsep1pt\parskip0pt\parsep0pt
\item
  Project start.
\end{itemize}

\begin{center}\rule{0.5\linewidth}{\linethickness}\end{center}

\section{2014-12-26 (Friday)}\label{friday}

\subsection{T: Functional Annotations of genes near SNPs of
interest}\label{t-functional-annotations-of-genes-near-snps-of-interest-1}

\begin{itemize}
\itemsep1pt\parskip0pt\parsep0pt
\item
  \texttt{{[}\_{]}} write code to create table of functional annotation
  info, given gene-names and annotation database.

  \begin{itemize}
  \itemsep1pt\parskip0pt\parsep0pt
  \item
    \texttt{{[}\_{]}} draft in ipython notebook
  \item
    \texttt{{[}\_{]}} copy to \texttt{spartan}
  \end{itemize}
\item
  \texttt{{[}\_{]}} write methods for the functional annotation paper
  section
\end{itemize}

\end{document}
