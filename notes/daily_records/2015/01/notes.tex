\documentclass[letterpaper]{scrartcl}
\usepackage{lmodern}
\usepackage{amssymb,amsmath}
\usepackage{ifxetex,ifluatex}
\usepackage{fixltx2e} % provides \textsubscript
\ifnum 0\ifxetex 1\fi\ifluatex 1\fi=0 % if pdftex
  \usepackage[T1]{fontenc}
  \usepackage[utf8]{inputenc}
\else % if luatex or xelatex
  \ifxetex
    \usepackage{mathspec}
    \usepackage{xltxtra,xunicode}
  \else
    \usepackage{fontspec}
  \fi
  \defaultfontfeatures{Mapping=tex-text,Scale=MatchLowercase}
  \newcommand{\euro}{€}
\fi
% use upquote if available, for straight quotes in verbatim environments
\IfFileExists{upquote.sty}{\usepackage{upquote}}{}
% use microtype if available
\IfFileExists{microtype.sty}{%
\usepackage{microtype}
\UseMicrotypeSet[protrusion]{basicmath} % disable protrusion for tt fonts
}{}
\usepackage{color}
\usepackage{fancyvrb}
\newcommand{\VerbBar}{|}
\newcommand{\VERB}{\Verb[commandchars=\\\{\}]}
\DefineVerbatimEnvironment{Highlighting}{Verbatim}{commandchars=\\\{\}}
% Add ',fontsize=\small' for more characters per line
\newenvironment{Shaded}{}{}
\newcommand{\KeywordTok}[1]{\textcolor[rgb]{0.00,0.44,0.13}{\textbf{{#1}}}}
\newcommand{\DataTypeTok}[1]{\textcolor[rgb]{0.56,0.13,0.00}{{#1}}}
\newcommand{\DecValTok}[1]{\textcolor[rgb]{0.25,0.63,0.44}{{#1}}}
\newcommand{\BaseNTok}[1]{\textcolor[rgb]{0.25,0.63,0.44}{{#1}}}
\newcommand{\FloatTok}[1]{\textcolor[rgb]{0.25,0.63,0.44}{{#1}}}
\newcommand{\CharTok}[1]{\textcolor[rgb]{0.25,0.44,0.63}{{#1}}}
\newcommand{\StringTok}[1]{\textcolor[rgb]{0.25,0.44,0.63}{{#1}}}
\newcommand{\CommentTok}[1]{\textcolor[rgb]{0.38,0.63,0.69}{\textit{{#1}}}}
\newcommand{\OtherTok}[1]{\textcolor[rgb]{0.00,0.44,0.13}{{#1}}}
\newcommand{\AlertTok}[1]{\textcolor[rgb]{1.00,0.00,0.00}{\textbf{{#1}}}}
\newcommand{\FunctionTok}[1]{\textcolor[rgb]{0.02,0.16,0.49}{{#1}}}
\newcommand{\RegionMarkerTok}[1]{{#1}}
\newcommand{\ErrorTok}[1]{\textcolor[rgb]{1.00,0.00,0.00}{\textbf{{#1}}}}
\newcommand{\NormalTok}[1]{{#1}}
\usepackage{longtable,booktabs}
\usepackage{graphicx}
\makeatletter
\def\maxwidth{\ifdim\Gin@nat@width>\linewidth\linewidth\else\Gin@nat@width\fi}
\def\maxheight{\ifdim\Gin@nat@height>\textheight\textheight\else\Gin@nat@height\fi}
\makeatother
% Scale images if necessary, so that they will not overflow the page
% margins by default, and it is still possible to overwrite the defaults
% using explicit options in \includegraphics[width, height, ...]{}
\setkeys{Gin}{width=\maxwidth,height=\maxheight,keepaspectratio}
\ifxetex
  \usepackage[setpagesize=false, % page size defined by xetex
              unicode=false, % unicode breaks when used with xetex
              xetex]{hyperref}
\else
  \usepackage[unicode=true]{hyperref}
\fi
\hypersetup{breaklinks=true,
            bookmarks=true,
            pdfauthor={Gus Dunn},
            pdftitle={Daily Records},
            colorlinks=true,
            citecolor=blue,
            urlcolor=blue,
            linkcolor=magenta,
            pdfborder={0 0 0}}
\urlstyle{same}  % don't use monospace font for urls
\setlength{\parindent}{0pt}
\setlength{\parskip}{6pt plus 2pt minus 1pt}
\setlength{\emergencystretch}{3em}  % prevent overfull lines
\setcounter{secnumdepth}{5}

\title{Daily Records\\\vspace{0.5em}{\large Caccone PostDoc}}
\author{Gus Dunn}
\date{January, 2015}
\usepackage{bbding}
\usepackage[T1]{fontenc}
\usepackage{lxfonts}

\begin{document}
\maketitle

{
\hypersetup{linkcolor=black}
\setcounter{tocdepth}{3}
\tableofcontents
}
\begin{center}\rule{0.5\linewidth}{\linethickness}\end{center}

\section{2015-01-02 (Friday)}\label{friday}

\subsection{Writing Methods}\label{writing-methods}

\begin{itemize}
\itemsep1pt\parskip0pt\parsep0pt
\item
  \texttt{{[}x{]}} Functional Annotations

  \begin{itemize}
  \itemsep1pt\parskip0pt\parsep0pt
  \item
    done-ish at 2015-01-02 08:58
  \end{itemize}
\item
  \texttt{{[} {]}} Linkage

  \begin{itemize}
  \itemsep1pt\parskip0pt\parsep0pt
  \item
    \textbf{STILL NEED TO DO THIS ANALYSIS\ldots{}}
  \end{itemize}
\end{itemize}

\subsection{Linkage Analysis}\label{linkage-analysis}

\begin{itemize}
\itemsep1pt\parskip0pt\parsep0pt
\item
  Still not understanding how you can calculate LD without phased data
  but it \textbf{seems} like many programs claim to\ldots{}
\end{itemize}

\subsubsection{PLINK}\label{plink}

\begin{itemize}
\itemsep1pt\parskip0pt\parsep0pt
\item
  \href{https://www.cog-genomics.org/plink2/ld}{v1.90 user manual: LD
  section}
\item
  \texttt{{[}x{]}} create the files needed from the master VCF file
  (tsetseFINAL\_14Oct2014\_f2\_53.recode.renamed\_scaffolds.vcf)

  \begin{itemize}
  \itemsep1pt\parskip0pt\parsep0pt
  \item
    looks like \texttt{plink} now reads VCF (\texttt{v1.90}): will try
    this first.
  \end{itemize}
\item
  \texttt{{[}x{]}} split data into smaller pieces to parallelize the
  \texttt{plink} analysis.

  \begin{itemize}
  \itemsep1pt\parskip0pt\parsep0pt
  \item
    looks like the \texttt{-\/-parallel} flag will allow \texttt{plink}
    to take care of this.
  \end{itemize}
\item
  \texttt{{[}x{]}} start run(s) on \texttt{louise}.
\item
  \texttt{{[} {]}} \emph{try running a \texttt{-\/-blocks} PLINK
  analysis for haplotype blocks to see if its useful?}
\end{itemize}

\textbf{plink commands run and kept:}

\begin{Shaded}
\begin{Highlighting}[]
\KeywordTok{plink} \NormalTok{--vcf tsetseFINAL_14Oct2014_f2_53.recode.renamed_scaffolds.vcf --allow-extra-chr \textbackslash{}}
    \KeywordTok{--r} \NormalTok{gz with-freqs  \textbackslash{}}
    \KeywordTok{--out} \NormalTok{plink_out/tsetseFINAL_14Oct2014_f2_53.recode.renamed_scaffolds.vcf\textbackslash{}}
    \KeywordTok{/ld/r_none_freqs}

\KeywordTok{plink} \NormalTok{--vcf tsetseFINAL_14Oct2014_f2_53.recode.renamed_scaffolds.vcf --allow-extra-chr \textbackslash{}}
    \KeywordTok{--r} \NormalTok{gz in-phase with-freqs  \textbackslash{}}
    \KeywordTok{--out} \NormalTok{plink_out/tsetseFINAL_14Oct2014_f2_53.recode.renamed_scaffolds.vcf\textbackslash{}}
    \KeywordTok{/ld/r_none_phase_freqs}

\KeywordTok{plink} \NormalTok{--vcf tsetseFINAL_14Oct2014_f2_53.recode.renamed_scaffolds.vcf --allow-extra-chr \textbackslash{}}
    \KeywordTok{--r} \NormalTok{triangle gz  \textbackslash{}}
    \KeywordTok{--out} \NormalTok{plink_out/tsetseFINAL_14Oct2014_f2_53.recode.renamed_scaffolds.vcf\textbackslash{}}
    \KeywordTok{/ld/r_tri}
\end{Highlighting}
\end{Shaded}

\subsubsection{Plot PLINK results}\label{plot-plink-results}

\begin{itemize}
\itemsep1pt\parskip0pt\parsep0pt
\item
  \texttt{{[}x{]}} create \texttt{ipython} notebook file

  \begin{itemize}
  \itemsep1pt\parskip0pt\parsep0pt
  \item
    \href{http://nbviewer.ipython.org/github/xguse/ipy_notebooks/blob/master/YALE/ddrad58/2015-01-02_Plot_PLINK_results.ipynb}{YALE/ddrad58/2015-01-02\_Plot\_PLINK\_results.ipynb}
  \item
    {[}2015-01-13{]}:
    \href{http://nbviewer.ipython.org/github/xguse/ipy_notebooks/blob/master/YALE/ddrad58/2015-01-05_Plot_PLINK_results.ipynb}{YALE/ddrad58/2015-01-05\_Plot\_PLINK\_results.ipynb}
  \end{itemize}
\item
  \texttt{{[} {]}} write code to plot
\end{itemize}

\subsection{TODO for Gisella}\label{todo-for-gisella}

\begin{itemize}
\itemsep1pt\parskip0pt\parsep0pt
\item
  \texttt{{[} {]}} re-read grant bit about bioinformatics and think
  about how to use Hongyu Zhao.
\end{itemize}

\begin{center}\rule{0.5\linewidth}{\linethickness}\end{center}

\section{2015-01-03 (Saturday)}\label{saturday}

\subsection{Linkage Analysis}\label{linkage-analysis-1}

\subsubsection{Plot PLINK results}\label{plot-plink-results-1}

\begin{itemize}
\itemsep1pt\parskip0pt\parsep0pt
\item
  \texttt{{[}x{]}} \textbf{WHAT} should be plotted?

  \begin{itemize}
  \itemsep1pt\parskip0pt\parsep0pt
  \item
    \texttt{{[}x{]}} what \emph{exactly is} the \(r\) value telling us?

    \begin{itemize}
    \itemsep1pt\parskip0pt\parsep0pt
    \item
      \texttt{{[}x{]}} does it already take into account the distance?

      \begin{itemize}
      \itemsep1pt\parskip0pt\parsep0pt
      \item
        according to
        \href{http://en.wikipedia.org/wiki/Linkage_disequilibrium\#Definition}{Wikipedia},
        \(r\) is simply the correlation coefficient between pairs of
        loci: \[r=\frac{D}{\sqrt{p_1p_2q_1q_2}}\]
      \end{itemize}
    \end{itemize}
  \item
    It seems like plotting \(\frac{r}{l_{a} - l_{b}}\) (\(r\) divided by
    distance) \textbf{is} warranted where:

    \begin{itemize}
    \itemsep1pt\parskip0pt\parsep0pt
    \item
      \(l_{a}\) is location of SNP\(_{a}\)
    \item
      \(l_{b}\) is location of SNP\(_{b}\)
    \end{itemize}
  \end{itemize}
\item
  \texttt{{[} {]}} write code to plot
\end{itemize}

\paragraph{Questions for Andrea}\label{questions-for-andrea}

\begin{itemize}
\itemsep1pt\parskip0pt\parsep0pt
\item
  Some MAFs are zero which causes the LD (\(r\)) to fail.
  \href{http://localhost:8888/jupiter/notebooks/YALE/ddrad58/2015-01-02_Plot_PLINK_results.ipynb\#LD-as-r-for-Scaffold0:}{link}
\end{itemize}

\begin{center}\rule{0.5\linewidth}{\linethickness}\end{center}

\section{2015-01-05 (Monday)}\label{monday}

\subsection{Linkage Analysis}\label{linkage-analysis-2}

\subsubsection{Plot PLINK results}\label{plot-plink-results-2}

\begin{itemize}
\itemsep1pt\parskip0pt\parsep0pt
\item
  met with Andrea after showing her what I had and specifically the MAF
  = 0 for about half the \texttt{scaffold0} comparisons.

  \begin{itemize}
  \itemsep1pt\parskip0pt\parsep0pt
  \item
    \textbf{(see meeting notes for more details)}
  \end{itemize}
\end{itemize}

\subsubsection{Re-Filter original VCF}\label{re-filter-original-vcf}

\begin{itemize}
\itemsep1pt\parskip0pt\parsep0pt
\item
  the incorrect (\texttt{-\/-min-allele}/\texttt{-\/-max-allele}) filter
  was used to generate:
  \texttt{tsetseFINAL\_14Oct2014\_f2\_53.recode.vcf}
\item
  the correct filter is \texttt{-\/-maf}.
\item
  I am doing it myself with MAF = 0.05 (see below).
\item
  \textbf{retained only 47.7\% sites}
\item
  \textbf{I will be re-running my PLINK LD analysis just in case.}
\end{itemize}

\begin{Shaded}
\begin{Highlighting}[]
\KeywordTok{wd238} \NormalTok{at compute-1-4 in ~GENOMES/glossina_fuscipes/annotations/SNPs (py278) }
\NormalTok{$ }\KeywordTok{vcftools} \NormalTok{\textbackslash{}}
    \KeywordTok{--vcf} \NormalTok{tsetseFINAL_14Oct2014_f2_53.recode.renamed_scaffolds.vcf \textbackslash{}}
    \KeywordTok{--maf} \NormalTok{0.05 \textbackslash{}}
    \KeywordTok{--out} \NormalTok{tsetseFINAL_14Oct2014_f2_53.recode.renamed_scaffolds.maf0_05 \textbackslash{}}
    \KeywordTok{--recode}

\KeywordTok{VCFtools} \NormalTok{- v0.1.12b}
\KeywordTok{(C)} \KeywordTok{Adam} \NormalTok{Auton and Anthony Marcketta 2009}

\KeywordTok{Parameters} \NormalTok{as interpreted:}
    \KeywordTok{--vcf} \NormalTok{tsetseFINAL_14Oct2014_f2_53.recode.renamed_scaffolds.vcf}
    \KeywordTok{--maf} \NormalTok{0.05}
    \KeywordTok{--out} \NormalTok{tsetseFINAL_14Oct2014_f2_53.recode.renamed_scaffolds.maf0_05}
    \KeywordTok{--recode}

\KeywordTok{After} \NormalTok{filtering, kept 53 out of 53 Individuals}
\KeywordTok{Outputting} \NormalTok{VCF file...}
\KeywordTok{After} \NormalTok{filtering, kept 73297 out of a possible 153650 Sites}
\KeywordTok{Run} \NormalTok{Time = 21.00 seconds}
\end{Highlighting}
\end{Shaded}

\subsubsection{PLINK - rerun}\label{plink---rerun}

\begin{Shaded}
\begin{Highlighting}[]
\KeywordTok{plink} \NormalTok{--vcf tsetseFINAL_14Oct2014_f2_53.recode.renamed_scaffolds.maf0_05.vcf \textbackslash{}}
\KeywordTok{--allow-extra-chr} \NormalTok{\textbackslash{}}
\KeywordTok{--r} \NormalTok{gz with-freqs dprime \textbackslash{}}
\KeywordTok{--out} \NormalTok{plink_out/tsetseFINAL_14Oct2014_f2_53.recode.renamed_scaffolds.maf0_05.vcf/ld/r_none_freqs_dprime}
\end{Highlighting}
\end{Shaded}

\subsection{Recover dead positives}\label{recover-dead-positives}

\subsubsection{Dissections}\label{dissections}

\begin{itemize}
\itemsep1pt\parskip0pt\parsep0pt
\item
  Prepped for dissections and pre-filled the worksheets
\item
  but we are out of the 1.5 ml tubes that I bought for this and I will
  have to go get some more tomorrow morning.
\end{itemize}

\begin{center}\rule{0.5\linewidth}{\linethickness}\end{center}

\section{2015-01-06 (Tuesday)}\label{tuesday}

\subsection{Linkage Analysis}\label{linkage-analysis-3}

\begin{itemize}
\itemsep1pt\parskip0pt\parsep0pt
\item
  emailed Dan about looking over the results.
\item
  Should probably run them by Jeff if he has time too.
\end{itemize}

\subsubsection{PLINK \texttt{-\/-make-bed}}\label{plink---make-bed}

\begin{verbatim}
wd238 at compute-1-4 in ~GENOMES/glossina_fuscipes/annotations/SNPs (py278) 
$ plink --vcf tsetseFINAL_14Oct2014_f2_53.recode.renamed_scaffolds.maf0_05.vcf \
> --allow-extra-chr \
> --maf 0.05 \
> --make-bed \
> --out tsetseFINAL_14Oct2014_f2_53.recode.renamed_scaffolds.maf0_05.plink

PLINK v1.90b2o 64-bit (25 Nov 2014)        https://www.cog-genomics.org/plink2
(C) 2005-2014 Shaun Purcell, Christopher Chang   GNU General Public License v3
Logging to tsetseFINAL_14Oct2014_f2_53.recode.renamed_scaffolds.maf0_05.plink.log.
48251 MB RAM detected; reserving 24125 MB for main workspace.
--vcf: 73k variants complete.
tsetseFINAL_14Oct2014_f2_53.recode.renamed_scaffolds.maf0_05.plink-temporary.bed
+
tsetseFINAL_14Oct2014_f2_53.recode.renamed_scaffolds.maf0_05.plink-temporary.bim
+
tsetseFINAL_14Oct2014_f2_53.recode.renamed_scaffolds.maf0_05.plink-temporary.fam
written.
73297 variants loaded from .bim file.
53 people (0 males, 0 females, 53 ambiguous) loaded from .fam.
Ambiguous sex IDs written to
tsetseFINAL_14Oct2014_f2_53.recode.renamed_scaffolds.maf0_05.plink.nosex .
Using 1 thread (no multithreaded calculations invoked).
Before main variant filters, 53 founders and 0 nonfounders present.
Calculating allele frequencies... done.
Total genotyping rate is 0.965098.
0 variants removed due to MAF threshold(s) (--maf/--max-maf).
73297 variants and 53 people pass filters and QC.
Note: No phenotypes present.
--make-bed to
tsetseFINAL_14Oct2014_f2_53.recode.renamed_scaffolds.maf0_05.plink.bed +
tsetseFINAL_14Oct2014_f2_53.recode.renamed_scaffolds.maf0_05.plink.bim +
tsetseFINAL_14Oct2014_f2_53.recode.renamed_scaffolds.maf0_05.plink.fam ...
done.
\end{verbatim}

\subsubsection{PLINK \texttt{-\/-blocks}}\label{plink---blocks}

\begin{itemize}
\itemsep1pt\parskip0pt\parsep0pt
\item
  running with \texttt{-\/-blocks} option to look at estimated haplotype
  blocks
\end{itemize}

\begin{verbatim}
wd238 at compute-1-4 in ~GENOMES/glossina_fuscipes/annotations/SNPs (py278) 
$ plink --vcf tsetseFINAL_14Oct2014_f2_53.recode.renamed_scaffolds.maf0_05.vcf \
> --allow-extra-chr \
> --blocks no-pheno-req no-small-max-span \
> --out plink_out/tsetseFINAL_14Oct2014_f2_53.recode.renamed_scaffolds.maf0_05.vcf\
    /ld/blocks_nophenoreq_nosmallmaxspan

PLINK v1.90b2o 64-bit (25 Nov 2014)        https://www.cog-genomics.org/plink2
(C) 2005-2014 Shaun Purcell, Christopher Chang   GNU General Public License v3
Logging to plink_out/tsetseFINAL_14Oct2014_f2_53.recode.renamed_scaffolds.maf0_05.vcf\
    /ld/blocks_nophenoreq_nosmallmaxspan.log.
48251 MB RAM detected; reserving 24125 MB for main workspace.
--vcf: 73k variants complete.

...

73297 variants loaded from .bim file.
53 people (0 males, 0 females, 53 ambiguous) loaded from .fam.
Ambiguous sex IDs written to
plink_out/tsetseFINAL_14Oct2014_f2_53.recode.renamed_scaffolds.maf0_05.vcf\ 
    /ld/blocks_nophenoreq_nosmallmaxspan.nosex
.
Using 1 thread (no multithreaded calculations invoked).
Before main variant filters, 53 founders and 0 nonfounders present.
Calculating allele frequencies... done.
Total genotyping rate is 0.965098.
73297 variants and 53 people pass filters and QC.
Note: No phenotypes present.
--blocks: 8040 haploblocks written to
plink_out/tsetseFINAL_14Oct2014_f2_53.recode.renamed_scaffolds.maf0_05.vcf\
    /ld/blocks_nophenoreq_nosmallmaxspan.blocks
.
Extra block details written to
plink_out/tsetseFINAL_14Oct2014_f2_53.recode.renamed_scaffolds.maf0_05.vcf\
    /ld/blocks_nophenoreq_nosmallmaxspan.blocks.det
.
Longest span: 199.985kb.
\end{verbatim}

\subsubsection{Plot PLINK results}\label{plot-plink-results-3}

\begin{itemize}
\itemsep1pt\parskip0pt\parsep0pt
\item
  cleaned up a few things
\item
  added residual plots following the regplots
\end{itemize}

\subsection{Recover dead positives}\label{recover-dead-positives-1}

\begin{itemize}
\itemsep1pt\parskip0pt\parsep0pt
\item
  need to meet with Kirsten (emailed her to schedule a time)

  \begin{itemize}
  \itemsep1pt\parskip0pt\parsep0pt
  \item
    subject: ``\emph{Short meeting to talk about the dead positives
    screen?}''
  \end{itemize}
\end{itemize}

\subsubsection{Dissections}\label{dissections-1}

\begin{itemize}
\itemsep1pt\parskip0pt\parsep0pt
\item
  getting more:

  \begin{itemize}
  \itemsep1pt\parskip0pt\parsep0pt
  \item
    tubes
  \item
    PBS
  \item
    Pens
  \end{itemize}
\item
  need to get receipt(s) from Kirsten regarding the dissection dish
  order

  \begin{itemize}
  \itemsep1pt\parskip0pt\parsep0pt
  \item
    emailed with subject: ``\emph{did you ever send me the receipt(s)
    for the stuff you ordered for the dissections over the internet?}''
  \end{itemize}
\end{itemize}

\subsection{Bonizzoni \emph{et al}: Insecticide
Resistance}\label{bonizzoni-et-al-insecticide-resistance}

\begin{itemize}
\itemsep1pt\parskip0pt\parsep0pt
\item
  I am SUPER late on this!
\end{itemize}

\begin{center}\rule{0.5\linewidth}{\linethickness}\end{center}

\section{2015-01-07 (Wednesday)}\label{wednesday}

\subsection{Bonizzoni \emph{et al}: Insecticide
Resistance}\label{bonizzoni-et-al-insecticide-resistance-1}

\textbf{Status:} COMPLETE

\begin{itemize}
\itemsep1pt\parskip0pt\parsep0pt
\item
  finished reviewing the main text
\item
  emailed it to her
\item
  will not be going over the legends or figs
\end{itemize}

\subsection{Meeting with Serap Aksoy}\label{meeting-with-serap-aksoy}

\textbf{Time:} 10:00 AM to 11:30AM

\subsubsection{Discussed}\label{discussed}

\begin{itemize}
\itemsep1pt\parskip0pt\parsep0pt
\item
  how to log Iowa tsetse samples
\item
  student to do much f the logging after a spreadsheet is devised
\item
  location of the other RNA midguts

  \begin{itemize}
  \itemsep1pt\parskip0pt\parsep0pt
  \item
    she said she thought they didnt get any but then thought she
    remembered that Brian tried extracting RNA from at least a few
    infected midguts with no success.
  \item
    **She said she needs the carcasses of the infected flies too which I
    did not remember (\textbf{I need to bring this up with Gisella bc
    this is a major reduction of our expected infected
    material\ldots{}}).

    \begin{itemize}
    \itemsep1pt\parskip0pt\parsep0pt
    \item
      \emph{this doesn't really make sense anyway since i dont think we
      preserved the bodies for RNA.}
    \end{itemize}
  \end{itemize}
\item
  having me send an ad or two for a postdoc position for her lab to my
  friends
\end{itemize}

\subsubsection{Action items}\label{action-items}

\textbf{Status:} IN PROGRESS

\begin{itemize}
\itemsep1pt\parskip0pt\parsep0pt
\item
  \texttt{{[}x{]}} create simple excel sheet to track Iowa samples

  \begin{itemize}
  \itemsep1pt\parskip0pt\parsep0pt
  \item
    \texttt{{[}x{]}} email sheet to Brian and Serap

    \begin{itemize}
    \itemsep1pt\parskip0pt\parsep0pt
    \item
      subject: \emph{Spreadsheet to record Iowa sample materials}
    \end{itemize}
  \item
    \textbf{NOTE:} sheet is a google sheet named
    \href{https://docs.google.com/spreadsheets/d/1SeoKnRQ0djB1xjyVGQy-NikNFTzQ-wL-hXCwUZAoNqw/edit?usp=sharing}{Iowa\_tsetse\_material\_inventory}
  \end{itemize}
\item
  \texttt{{[}x{]}} locate extra RNA midguts in our freezers

  \begin{itemize}
  \itemsep1pt\parskip0pt\parsep0pt
  \item
    \texttt{{[}x{]}} email Aksoy, Brian, Michelle to schedule pickup

    \begin{itemize}
    \itemsep1pt\parskip0pt\parsep0pt
    \item
      subject: \emph{Many more midguts and heads for RNA}
    \end{itemize}
  \end{itemize}
\item
  \texttt{{[} {]}} send feelers and ads to friends about postdoc
  position in her lab
\item
  \texttt{{[} {]}} contact Gisella about Serap wanting the
  carcasses\ldots{}
\end{itemize}

\subsection{Meeting with Andrea}\label{meeting-with-andrea}

\subsubsection{Discussed}\label{discussed-1}

\begin{itemize}
\itemsep1pt\parskip0pt\parsep0pt
\item
  problem re-running the figure generation R script
\item
  couldn't open the PNG writer bc no X11 on the cluster and
  \texttt{ssh -Y} wasn't working even though it did last time\ldots{}
\item
  I added some code to the R script to specify `png(type=``cairo'')
\item
  waiting to hear the outcome

  \begin{itemize}
  \itemsep1pt\parskip0pt\parsep0pt
  \item
    \emph{program ran but output was not what was expected: many more
    graphs than last time}
  \item
    I expect user error
  \end{itemize}
\end{itemize}

\subsection{Recover dead positives}\label{recover-dead-positives-2}

\begin{itemize}
\itemsep1pt\parskip0pt\parsep0pt
\item
  Meeting with Kirstin tomorrow at 1 or 2 PM
\end{itemize}

\begin{center}\rule{0.5\linewidth}{\linethickness}\end{center}

\section{2015-01-08 (Thursday)}\label{thursday}

\subsection{Admin stuff}\label{admin-stuff}

\subsubsection{Serap's postdoc
advertisement}\label{seraps-postdoc-advertisement}

\begin{itemize}
\itemsep1pt\parskip0pt\parsep0pt
\item
  posted to facebook
\end{itemize}

\subsubsection{Lab meetings}\label{lab-meetings}

\begin{itemize}
\itemsep1pt\parskip0pt\parsep0pt
\item
  email from Jeff:

  \begin{itemize}
  \itemsep1pt\parskip0pt\parsep0pt
  \item
    subject: \emph{lab meetings}
  \item
    body:
  \end{itemize}

  \begin{quote}
  Gus,
  \end{quote}

  \begin{quote}
  Any lab meetings set up? We should grab Aris and Yiota who might give
  a joint one. Then also the new Anthropology guy, Eduardo
  Fernandez-Duque.
  \end{quote}

  \begin{quote}
  Jeff
  \end{quote}
\item
  doodle poll:

  \begin{itemize}
  \itemsep1pt\parskip0pt\parsep0pt
  \item
    \href{http://doodle.com/hc4r8gdi6wnse425}{link} sent to
    \texttt{pc\_labs}
  \item
    not certain if Maggie is on that list yet
  \item
    emailed Carol for her email in case not

    \begin{itemize}
    \itemsep1pt\parskip0pt\parsep0pt
    \item
      She \textbf{WAS} on the list when I sent the link
    \item
      Carol replied with current list: recorded below
    \end{itemize}
  \end{itemize}
\end{itemize}

\subsubsection{Current \texttt{pc\_labs}
list}\label{current-pcux5flabs-list}

\begin{Shaded}
\begin{Highlighting}[]
\CommentTok{#separator-CACCONE#}
\KeywordTok{adalgisa.caccone@yale.edu}
\KeywordTok{carol.mariani@yale.edu}
\KeywordTok{danielle.edwards@yale.edu}
\KeywordTok{nphavill@fs.fed.us}
\KeywordTok{jrichardson@providence.edu}
\KeywordTok{giovanna.carpi@yale.edu}
\KeywordTok{katharine.walter@yale.edu}
\KeywordTok{gus.dunn@yale.edu}

\CommentTok{#separator-POWELL#}
\KeywordTok{jeffrey.powell@yale.edu}
\KeywordTok{kirstin.dion@yale.edu}
\KeywordTok{andrea.gloria-soria@yale.edu}
\KeywordTok{b.evans@yale.edu}
\KeywordTok{joshua.richardson@yale.edu}

\CommentTok{#separator-TEMP-ROTATION-UNDERGRAD#}
\KeywordTok{christian.hernandez@yale.edu}
\KeywordTok{elaine.guevara@yale.edu}
\KeywordTok{andres.valdivieso@yale.edu}
\KeywordTok{mkcorley@gmail.com}
\KeywordTok{alexis.halyard@yale.edu}
\KeywordTok{pkotsakiozi@hotmail.com}
\KeywordTok{aristeidis.parmakelis@yale.edu}
\end{Highlighting}
\end{Shaded}

\subsection{Recover dead positives}\label{recover-dead-positives-3}

\subsubsection{Meeting with Kirstin}\label{meeting-with-kirstin}

\begin{itemize}
\itemsep1pt\parskip0pt\parsep0pt
\item
  see:
  \href{file:///home/gus/Dropbox/repos/git/markdown-docs/notes/meetings/Kirsten-2015-01-08/Kirsten-2015-01-08.md}{Kirsten-2015-01-08.md}
\item
  sent above for confirmation or amendment to Kirstin

  \begin{itemize}
  \itemsep1pt\parskip0pt\parsep0pt
  \item
    She approves
  \end{itemize}
\end{itemize}

\begin{center}\rule{0.5\linewidth}{\linethickness}\end{center}

\section{2015-01-09 (Friday)}\label{friday-1}

\subsection{Sarah Licensing Exam}\label{sarah-licensing-exam}

\begin{itemize}
\itemsep1pt\parskip0pt\parsep0pt
\item
  I was taking care of the kids all morning
\end{itemize}

\subsection{New project brainstorming}\label{new-project-brainstorming}

\subsection{PLINK: Fst}\label{plink-fst}

\begin{itemize}
\itemsep1pt\parskip0pt\parsep0pt
\item
  defining population ID file for:

  \begin{itemize}
  \itemsep1pt\parskip0pt\parsep0pt
  \item
    \texttt{tsetseFINAL\_14Oct2014\_f2\_53.recode.renamed\_scaffolds.maf0\_05.vcf}
  \item
    \texttt{tsetseFINAL\_14Oct2014\_f2\_53.recode.renamed\_scaffolds.maf0\_05.vcf.popdef}
  \end{itemize}
\end{itemize}

\begin{verbatim}
plink --bfile tsetseFINAL_14Oct2014_f2_53.recode.renamed_scaffolds.maf0_05.plink \
--allow-extra-chr \
--within tsetseFINAL_14Oct2014_f2_53.recode.renamed_scaffolds.maf0_05.vcf.popdef \
--fst \
--out plink_out/tsetseFINAL_14Oct2014_f2_53.recode.renamed_scaffolds.maf0_05.vcf/fst/out
\end{verbatim}

\begin{itemize}
\itemsep1pt\parskip0pt\parsep0pt
\item
  This keeps giving me errors:
\end{itemize}

\begin{verbatim}
Warning: No samples named in --within file remain in the current analysis.
Using 1 thread (no multithreaded calculations invoked).
Before main variant filters, 53 founders and 0 nonfounders present.
Calculating allele frequencies... done.
Total genotyping rate is 0.965098.
73297 variants and 53 people pass filters and QC.
Note: No phenotypes present.
Error: --fst requires at least two nonempty clusters.
\end{verbatim}

\begin{center}\rule{0.5\linewidth}{\linethickness}\end{center}

\section{2015-01-10 (Saturday)}\label{saturday-1}

\subsection{PLINK: Fst}\label{plink-fst-1}

\begin{itemize}
\itemsep1pt\parskip0pt\parsep0pt
\item
  Still getting errors
\item
  looks like its bc there is no sex info attached to the samples
\item
  looking for other was to do this stuff: found EggLib-py
\end{itemize}

\subsection{Install EggLib}\label{install-egglib}

\begin{itemize}
\itemsep1pt\parskip0pt\parsep0pt
\item
  installation needs \texttt{bio++}
\item
  see tomorrow
\end{itemize}

\subsection{Install \texttt{Bio++}
(\texttt{bpp})}\label{install-bio-bpp}

\begin{itemize}
\itemsep1pt\parskip0pt\parsep0pt
\item
  install script
  (\href{http://biopp.univ-montp2.fr/Download/bpp-setup.sh}{bpp-setup.sh})
  obtained from
  \href{http://biopp.univ-montp2.fr/wiki/index.php/Installation}{bio++
  website}.

  \begin{itemize}
  \itemsep1pt\parskip0pt\parsep0pt
  \item
    altered install script to fit system (\texttt{louise}) and renamed
    \href{/home/gus/remote_mounts/louise/scripts/installs/install_bpp_2.2.0.sh}{install\_bpp\_2.2.0.sh}.
  \end{itemize}
\item
  see tomorrow
\end{itemize}

\begin{center}\rule{0.5\linewidth}{\linethickness}\end{center}

\section{2015-01-11 (Sunday)}\label{sunday}

\subsection{Install EggLib}\label{install-egglib-1}

\begin{itemize}
\itemsep1pt\parskip0pt\parsep0pt
\item
  finished \texttt{bio++}install
\item
  installing other things (actually may just module-ize versions already
  installed on \texttt{louise}:

  \begin{itemize}
  \itemsep1pt\parskip0pt\parsep0pt
  \item
    \texttt{{[}x{]}} gsl (already installed)
  \item
    \texttt{{[}x{]}} clustalw (linked to
    \texttt{louise/\textasciitilde{}MAIN\_APPS/clustalw/clustalw-2.0.12-linux-i686-libcppstatic})
  \item
    \texttt{{[}x{]}} muscle
  \item
    \texttt{{[}x{]}} paml
  \item
    \texttt{{[}x{]}} phyml
  \item
    \texttt{{[}x{]}} primer3
  \item
    \texttt{{[}x{]}} phylip
  \end{itemize}
\end{itemize}

\subsection{Install \texttt{Bio++}
(\texttt{bpp})}\label{install-bio-bpp-1}

\begin{itemize}
\itemsep1pt\parskip0pt\parsep0pt
\item
  had to ammend the install script to include the \texttt{bpp} install
  location in \texttt{\$PATH} so it can use itself to install/build
  parts of itself
\item
  as far as I can tell the only things that fail to install now are the
  GUI-based stuff that needs Qt. As I am using this on th cluster I dont
  need/want these so I am going to proceed as if this succeeded.
\end{itemize}

\begin{center}\rule{0.5\linewidth}{\linethickness}\end{center}

\section{2015-01-12 (Monday)}\label{monday-1}

\subsection{Install EggLib}\label{install-egglib-2}

\begin{itemize}
\itemsep1pt\parskip0pt\parsep0pt
\item
  finished installing external helper programs:

  \begin{itemize}
  \itemsep1pt\parskip0pt\parsep0pt
  \item
    clustalw
  \item
    phylip
  \end{itemize}
\item
  made a \texttt{modules} file for the whole group:
  \texttt{egglib\_helpers}
\end{itemize}

\subsection{Install \texttt{Bio++}
(\texttt{bpp})}\label{install-bio-bpp-2}

\begin{itemize}
\itemsep1pt\parskip0pt\parsep0pt
\item
  forgot to write a \texttt{modules} file for this.
\item
  doing it now
\end{itemize}

\begin{center}\rule{0.5\linewidth}{\linethickness}\end{center}

\section{2015-01-13 (Tuesday)}\label{tuesday-1}

\subsection{Recover dead positives}\label{recover-dead-positives-4}

\begin{itemize}
\itemsep1pt\parskip0pt\parsep0pt
\item
  meeting
\item
  see meeting notes
  \href{file:///home/gus/Dropbox/repos/git/markdown-docs/notes/meetings/gisella_kirsten_2015-01-14/gisella_kirsten_2015-01-14.md}{gisella\_kirsten\_2015-01-14.md}
\end{itemize}

\subsection{Maps stuff}\label{maps-stuff}

\begin{itemize}
\itemsep1pt\parskip0pt\parsep0pt
\item
  Need to update our Northern Uganda map with the latest Village
  location data
\item
  \texttt{Spartan.utils.maps.gps}

  \begin{itemize}
  \itemsep1pt\parskip0pt\parsep0pt
  \item
    coding the functions to take all trap GPS coords for a village and
    return one GPS coord set for each that represents the central
    tendency for simplified plotting
  \end{itemize}
\end{itemize}

\subsection{TsetseSampleDB}\label{tsetsesampledb}

\begin{itemize}
\itemsep1pt\parskip0pt\parsep0pt
\item
  adding names/village codes to the list of village-to-code map
  (\href{file:///home/gus/Dropbox/repos/git/TsetseCheckout/TsetseCheckout/data/village_id_map.csv}{village\_id\_map.csv})
\end{itemize}

\subsection{Helping Aris}\label{helping-aris}

\begin{itemize}
\itemsep1pt\parskip0pt\parsep0pt
\item
  getting \texttt{mpirun mrbayes} to run on \texttt{grace}
\item
  took \textasciitilde{} 1 hour.
\end{itemize}

\begin{center}\rule{0.5\linewidth}{\linethickness}\end{center}

\section{2015-01-15 (Thursday)}\label{thursday-1}

\subsection{Science Fair}\label{science-fair}

\begin{itemize}
\itemsep1pt\parskip0pt\parsep0pt
\item
  8 to 12:30
\end{itemize}

\subsection{Meeting about Kenya
Tsetse}\label{meeting-about-kenya-tsetse}

Members:

\begin{itemize}
\itemsep1pt\parskip0pt\parsep0pt
\item
  Gisella
\item
  Serap
\item
  Michelle
\item
  Brian
\item
  Gus
\end{itemize}

Notes (bad bc they were taken with my phone):

\begin{itemize}
\itemsep1pt\parskip0pt\parsep0pt
\item
  Gpd samples
\item
  DNA samples
\item
  What column data will be needed for this Gpd Kenya collection
\item
  Re-circulate collection template excel and protocol etc
\end{itemize}

\begin{center}\rule{0.5\linewidth}{\linethickness}\end{center}

\section{2015-01-16 (Friday)}\label{friday-2}

\subsection{Collection Spreadsheet
review}\label{collection-spreadsheet-review}

\begin{itemize}
\itemsep1pt\parskip0pt\parsep0pt
\item
  point is to make sure we can use this for the Kenya ``simpler''
\item
  after speaking with Gisella, I am going to add a few of the
  ``simpler'' column heading to the normal collection spreadsheet and
  write a bunch of notes explaining that not everything needs to be
  filled in for everything.
\item
  collection spreadsheet:
  \href{file:///home/gus/Dropbox/uganda\%20data/collection_sheet_templates/Example_collection_template_kenya.xls}{Example\_collection\_template\_kenya.xls}

  \begin{itemize}
  \itemsep1pt\parskip0pt\parsep0pt
  \item
    \textbf{status:} finished
  \end{itemize}
\item
  summary spreadsheet:
  \href{file:///home/gus/Dropbox/uganda\%20data/collection_sheet_templates/Example_summary_template_kenya.xls}{Example\_summary\_template\_kenya.xls}

  \begin{itemize}
  \itemsep1pt\parskip0pt\parsep0pt
  \item
    \textbf{status:} in progress
  \end{itemize}
\end{itemize}

\subsubsection{Email explanation}\label{email-explanation}

\textbf{Subject:}

\textbf{Body:}

\subsection{Updating maps: current trap
locations}\label{updating-maps-current-trap-locations}

\begin{itemize}
\itemsep1pt\parskip0pt\parsep0pt
\item
  created new
  spreadsheet:\href{file:///home/gus/Dropbox/uganda\%20data/collection_meta_data/meta_data.ods}{collection\_meta\_data/meta\_data.ods}
  to store current state of stuff like the village-to-symbol map, etc.
\item
  \texttt{{[}} \texttt{{]}} collecting trap GPS data to file:
  \href{file://CREATE_ME}{TsetseCheckout/data/village\_id\_map.csv}
\item
  \texttt{{[}} \texttt{{]}} collecting all village-to-symbol maps that I
  have to
  \href{file:///home/gus/Dropbox/repos/git/TsetseCheckout/TsetseCheckout/data/village_id_map.csv}{meta\_data.ods}
\end{itemize}

\subsubsection{\texttt{spartan} dev: GPS
stuff}\label{spartan-dev-gps-stuff}

\begin{itemize}
\itemsep1pt\parskip0pt\parsep0pt
\item
  pycharm and ipython
\end{itemize}

\subsection{Phone for Dan}\label{phone-for-dan}

I emailed Dan the following:

\begin{quote}
\textbf{Subject:} Phone call for you

\textbf{Body:} Just fielded a call for you from Karan(Karen?) Peart from
the Yale Public affairs and communications office.

She would like you to call her back at your earliest convenience
(432-1326).

Gus
\end{quote}

\begin{center}\rule{0.5\linewidth}{\linethickness}\end{center}

\section{2015-01-18 (Sunday)}\label{sunday-1}

\subsection{Sarah is sick}\label{sarah-is-sick}

\begin{itemize}
\itemsep1pt\parskip0pt\parsep0pt
\item
  short day: 10am to 1:45pm
\end{itemize}

\subsection{Updating maps: current trap
locations}\label{updating-maps-current-trap-locations-1}

\begin{itemize}
\itemsep1pt\parskip0pt\parsep0pt
\item
  ipython:
  \href{file:///home/gus/Dropbox/common/ipy_notebooks/YALE/maps_stuff/2015-01-16_convert_fall2014_trap_gps_village_names.ipynb}{2015-01-16\_convert\_fall2014\_trap\_gps\_village\_names.ipynb}
\end{itemize}

\subsubsection{\texttt{spartan} dev: GPS
stuff}\label{spartan-dev-gps-stuff-1}

\begin{itemize}
\itemsep1pt\parskip0pt\parsep0pt
\item
  working on teaching \texttt{GPSCoordTree} how to get mean coordinates
\end{itemize}

\begin{center}\rule{0.5\linewidth}{\linethickness}\end{center}

\section{2015-01-19 (Monday)}\label{monday-2}

\subsection{Andrea: quick chat}\label{andrea-quick-chat}

\begin{itemize}
\itemsep1pt\parskip0pt\parsep0pt
\item
  wants to re-run the ddRAD pipeline since I(we) found some issues with
  at least one of the command line runs' options.
\item
  I agree
\item
  \texttt{{[}} \texttt{{]}} \#todo: I am installing
  \href{http://creskolab.uoregon.edu/stacks/}{Stacks 1.24} for her on
  \texttt{louise} and trying to set up the web-based analysis part
\item
  should not change TOO much about the results and will end up being
  MUCH more replicatable due to the use of publicly accessible data from
  vectorbase.
\item
  we can continue to work with the current data until the new set is
  done and just adjust the work to accommodate the new stuff at the end.
\end{itemize}

\subsection{Updating maps: current trap
locations}\label{updating-maps-current-trap-locations-2}

\subsubsection{\texttt{spartan} dev: GPS
stuff}\label{spartan-dev-gps-stuff-2}

\begin{itemize}
\itemsep1pt\parskip0pt\parsep0pt
\item
  working on teaching \texttt{GPSCoordTree} how to get mean coordinates
\end{itemize}

\subsection{Install Stacks}\label{install-stacks}

\begin{itemize}
\itemsep1pt\parskip0pt\parsep0pt
\item
  \href{http://creskolab.uoregon.edu/stacks/manual/\#install}{installation
  guide}
\end{itemize}

\subsection{Collection Spreadsheet
review}\label{collection-spreadsheet-review-1}

\begin{itemize}
\item
  adding explanation text to the
  \href{file:///home/gus/Dropbox/repos/git/markdown-docs/protocols/fly_collection_basic/fly_collection_basic.md}{fly\_collection\_basic.md}
  document.
\item
  \textbf{STATUS:}

  \begin{itemize}
  \itemsep1pt\parskip0pt\parsep0pt
  \item
    having issues getting validation and drop-down lists to carry over
    into ``empty'' rows
  \item
    plan to fix it by copying a template row into like 1000 rows
  \item
    still need to execute the above tomorrow bc computer is acting a
    fool and I have to go home to sick Sarah and Liam.
  \end{itemize}
\end{itemize}

\begin{center}\rule{0.5\linewidth}{\linethickness}\end{center}

\section{2015-01-20 (Tuesday)}\label{tuesday-2}

\subsection{Sarah still sick}\label{sarah-still-sick}

\begin{itemize}
\itemsep1pt\parskip0pt\parsep0pt
\item
  stayed home to help with particularly crazy morning
\end{itemize}

\subsection{Meeting with Gisella and Andrea: ddRAD
paper}\label{meeting-with-gisella-and-andrea-ddrad-paper}

\begin{itemize}
\itemsep1pt\parskip0pt\parsep0pt
\item
  Met at 10:30 AM
\item
  summarized in
  \href{file:///home/gus/Dropbox/repos/git/markdown-docs/notes/meetings/gisella_andrea_2015-01-20/gisella_andrea_2015-01-20.md}{gisella\_andrea\_2015-01-20.md}.
\end{itemize}

\subsection{ddRAD todos}\label{ddrad-todos}

\begin{itemize}
\itemsep1pt\parskip0pt\parsep0pt
\item
  \texttt{{[}in progress{]}} read Mark's tryp paper for the LD stuff
\item
  \texttt{{[}¤{]}} email Mark to have a short sit-down to go over my
  results and ask about his work
\end{itemize}

\subsection{Install Stacks}\label{install-stacks-1}

\begin{itemize}
\itemsep1pt\parskip0pt\parsep0pt
\item
  \href{http://creskolab.uoregon.edu/stacks/manual/\#install}{installation
  guide}
\end{itemize}

\subsubsection{Prerequisites}\label{prerequisites}

\textbf{Visualization:}

\begin{itemize}
\itemsep1pt\parskip0pt\parsep0pt
\item
  \texttt{{[} {]}} DB2 Pear Module:
  \url{http://pear.php.net/package/MDB2/}
\item
  \texttt{{[} {]}} MDB2 MySQL driver:
  \url{http://pear.php.net/package/MDB2_Driver_mysql/}
\item
  \texttt{{[} {]}} PHP
\item
  \texttt{{[} {]}} MySQL
\item
  \texttt{{[} {]}} Perl DBI module installed with the MySQL driver
  \href{http://search.cpan.org/dist/DBD-mysql/}{CPAN/dist/DBD-mysql/}
\end{itemize}

\textbf{Spreadsheet export:}

\begin{itemize}
\itemsep1pt\parskip0pt\parsep0pt
\item
  \texttt{{[} {]}} Perl module:
  \href{http://search.cpan.org/~jmcnamara/Spreadsheet-WriteExcel-2.37/}{\texttt{Spreadsheet::WriteExcel}}
\end{itemize}

\textbf{Performance improvement:}

\begin{itemize}
\itemsep1pt\parskip0pt\parsep0pt
\item
  \texttt{{[}X{]}} \texttt{samtools} for reading BAM files (already
  installed)
\item
  \texttt{{[} {]}} Google's \texttt{SparseHash} class to lower memory
  usage \url{http://code.google.com/p/sparsehash/}
\end{itemize}

\subsubsection{Stacks}\label{stacks}

\textbf{INSTALLATION PROBLEMS NOTES:}

\begin{itemize}
\itemsep1pt\parskip0pt\parsep0pt
\item
  Having issues getting the samtools includes and libs configured for
  the \texttt{./configure} command.
\item
  plan to build on \texttt{jupiter} using the ARCH ABS and copy the
  include/lib directories to \texttt{louise}
\end{itemize}

\textbf{download location:}

\begin{itemize}
\item
  status: \emph{\texttt{in progress}}
\item
  downloaded
  \href{http://creskolab.uoregon.edu/stacks/source/stacks-1.24.tar.gz}{stacks-1.24.tar.gz}
  to \texttt{gus@louise/src}.
\end{itemize}

\textbf{install script:}

\begin{itemize}
\item
  status: \emph{\texttt{in progress}}
\item
  \href{file:///home/gus/remote_mounts/louise/scripts/installs/install_XXXXXXX}{\texttt{gus@louise/scripts/installs/install\_XXXX}}
\end{itemize}

\textbf{module file:}

\begin{itemize}
\item
  status: \emph{\texttt{in progress}}
\item
  \href{/home/gus/remote_mounts/louise/.local/environment-modules/Modules/3.2.10/my_modulefiles/XXXXX/XXXXX}{\texttt{gus@louise/.local/environment-modules/Modules/3.2.10/my\_modulefiles/XXXXX/XXXXX}}
\end{itemize}

\textbf{software root:}

\begin{itemize}
\item
  status: \emph{\texttt{in progress}}
\item
  \href{/home/gus/remote_mounts/louise/.local/easybuild/software/XXXXX/XXXXX}{\texttt{gus@louise/home/gus/remote\_mounts/louise/.local/easybuild/software/XXXXX/XXXXX}}
\end{itemize}

\subsection{Install \texttt{SparseHash}}\label{install-sparsehash}

\textbf{download location:}

\begin{itemize}
\item
  status: \emph{\texttt{complete}}
\item
  downloaded
  \href{https://sparsehash.googlecode.com/files/sparsehash-2.0.2.tar.gz}{sparsehash-2.0.2.tar.gz}
  to \texttt{gus@louise/src}.
\end{itemize}

\textbf{install script:}

\begin{itemize}
\item
  status: \emph{\texttt{written and run}}
\item
  \href{file:///home/gus/remote_mounts/louise/scripts/installs/install_sparsehash_2.0.2.sh}{\texttt{gus@louise/scripts/installs/install\_sparsehash\_2.0.2.sh}}
\end{itemize}

\textbf{module file:}

\begin{itemize}
\item
  status: \emph{\texttt{complete but not tested}}
\item
  \href{/home/gus/remote_mounts/louise/.local/environment-modules/Modules/3.2.10/my_modulefiles/sparsehash/2.0.2}{\texttt{gus@louise/.local/environment-modules/Modules/3.2.10/my\_modulefiles/sparsehash/2.0.2}}
\end{itemize}

\textbf{software root:}

\begin{itemize}
\item
  status: \emph{\texttt{installed}}
\item
  \href{/home/gus/remote_mounts/louise/.local/easybuild/software/sparsehash/2.0.2}{\texttt{gus@louise/home/gus/remote\_mounts/louise/.local/easybuild/software/sparsehash/2.0.2}}
\end{itemize}

\begin{center}\rule{0.5\linewidth}{\linethickness}\end{center}

\section{2015-01-21 (Wednesday)}\label{wednesday-1}

\subsection{Family still VERY sick}\label{family-still-very-sick}

\begin{itemize}
\itemsep1pt\parskip0pt\parsep0pt
\item
  stayed home with Liam while Sarah took Clementine and herself to the
  doctor
\item
  got to work at 12:30
\item
  had to go home so Sarah could sleep bc Clem was not letting her
\item
  went home at 1:30
\item
  came back at 3:30
\item
  home again at 6:00
\item
  work again at 8:30
\end{itemize}

\subsection{Manual install of
Samtools/htslib}\label{manual-install-of-samtoolshtslib}

\subsubsection{\texttt{htslib} (built with Arch ABS on
\texttt{jupiter})}\label{htslib-built-with-arch-abs-on-jupiter}

ABANDONING THIS METHOD FOR NOW. TOO MANY PROBLEMS WITH INTEGRATING
CERTAIN \texttt{INCLUDE} AND \texttt{LIB} DIRECTRIES WHEN COMPILING
ACCROSS DEPENDENCIES. TRYING \texttt{EASYBUILD} AND ITS \emph{TOOLCHAIN}
PARADIGM FOR NOW.

\textbf{ABS build:}

\begin{verbatim}
cd /home/gus/remote_mounts/louise/src/ABS/
tar -xf htslib.tar.gz
cd htslib
makepkg
\end{verbatim}

\textbf{Install script:}

\begin{itemize}
\itemsep1pt\parskip0pt\parsep0pt
\item
  \href{file:///home/gus/remote_mounts/louise/scripts/installs/install_htslib_1.1.sh}{\texttt{gus@louise/scripts/installs/install\_htslib\_1.1.sh}}
\end{itemize}

\subsection{EasyBuild installs}\label{easybuild-installs}

\subsubsection{EasyBuild 1.16.1}\label{easybuild-1.16.1}

\textbf{Install script:}

\begin{itemize}
\itemsep1pt\parskip0pt\parsep0pt
\item
  \href{file:///home/gus/remote_mounts/louise/scripts/installs/install_easybuild_1.16.1.sh}{\texttt{gus@louise/scripts/installs/install\_easybuild\_1.16.1.sh}}
\end{itemize}

\subsubsection{Install
\texttt{samtools-1.1}}\label{install-samtools-1.1}

\begin{Shaded}
\begin{Highlighting}[]
\KeywordTok{wd238} \NormalTok{at compute-21-15 in ~ (py278) }
\NormalTok{$ }\KeywordTok{md} \NormalTok{load EasyBuild/1.16.1}

\KeywordTok{wd238} \NormalTok{at compute-21-15 in ~ (py278) }
\NormalTok{$ }\KeywordTok{eb} \NormalTok{SAMtools-1.1-goolf-1.4.10.eb --try-toolchain=goolf,1.4.10-no-OFED --robot}

\KeywordTok{...}
\end{Highlighting}
\end{Shaded}

\begin{itemize}
\itemsep1pt\parskip0pt\parsep0pt
\item
  this will be installing the whole toolchain and all \texttt{samtools}
  dependencies so it was executed in a \texttt{screen}.
\item
  I am going home to sleep while this works
  \texttt{{[}2015-01-21 21:33{]}}.
\item
  emailing Andrea first to let her know its not gonna be ready when I
  told her.
\end{itemize}

\textbf{STATUS: \texttt{{[}2015-01-22 07:52{]}}}

\begin{itemize}
\itemsep1pt\parskip0pt\parsep0pt
\item
  Build seems to have \textbf{SUCCEEDED}
\end{itemize}

\begin{center}\rule{0.5\linewidth}{\linethickness}\end{center}

\section{2015-01-22 (Thursday)}\label{thursday-2}

\subsection{Collection documentation
files}\label{collection-documentation-files}

\begin{itemize}
\item
  fixed/kludged the collection template to keep cell-notes and
  verification by including ``dummy'' entries up to 500.

  \begin{itemize}
  \itemsep1pt\parskip0pt\parsep0pt
  \item
    collection template:
    \href{file:///home/gus/Dropbox/uganda\%20data/collection_sheet_templates/Example_collection_template_kenya.xlsx}{Example\_collection\_template\_kenya.xlsx}
  \item
    summary template:
    \href{file:///home/gus/Dropbox/uganda\%20data/collection_sheet_templates/Example_summary_template_kenya.xlsx}{Example\_summary\_template\_kenya.xlsx}
  \end{itemize}
\item
  amended and committed v0.2.1 of
  \href{file:///home/gus/Dropbox/repos/git/markdown-docs/protocols/fly_collection_basic/fly_collection_basic.md}{fly\_collection\_basic.md}
  to the repo with custom message:

  \begin{quote}
  ``protocols/fly\_collection\_basic/fly\_collection\_basic.md commited
  at version: v0.2.1''
  \end{quote}
\end{itemize}

\subsection{EasyBuild installs}\label{easybuild-installs-1}

\subsubsection{Install \texttt{GSL-1.16}}\label{install-gsl-1.16}

\begin{verbatim}
wd238 at compute-21-15 in ~ (py278) 
$ eb GSL-1.16-goolf-1.4.10.eb --try-toolchain=goolf,1.4.10-no-OFED --robot
\end{verbatim}

\textbf{STATUS:}

\begin{verbatim}
== COMPLETED: Installation ended successfully
== Results of the build can be found in the log file \
    /home2/wd238/.local/easybuild/software/GSL/1.16-goolf-1.4.10-no-OFED/\
    easybuild/easybuild-GSL-1.16-20150122.075231.log
== Build succeeded for 1 out of 1
\end{verbatim}

\subsubsection{Install
\texttt{sparsehash-2.0.2}}\label{install-sparsehash-2.0.2}

\begin{verbatim}
wd238 at compute-21-15 in ~ (py278)
eb google-sparsehash-2.0.2-goolf-1.4.10.eb --try-toolchain=goolf,1.4.10-no-OFED --robot
\end{verbatim}

\textbf{STATUS:}

\begin{verbatim}
== COMPLETED: Installation ended successfully
== Results of the build can be found in the log file \
    /home2/wd238/.local/easybuild/software/google-sparsehash/ \
    2.0.2-goolf-1.4.10-no-OFED/easybuild/ \
    easybuild-google-sparsehash-2.0.2-20150122.080547.log
== Build succeeded for 1 out of 1
\end{verbatim}

\subsubsection{Install \texttt{Stacks-1.03}}\label{install-stacks-1.03}

\textbf{NOTE:} this is not for use per se but to generate the
\texttt{config} and \texttt{module} files so that I can modify them for
the latest version of \texttt{Stacks} and install \emph{THAT} version.

\begin{verbatim}
wd238 at compute-21-15 in ~ (py278) 
$ eb Stacks-1.03-goolf-1.4.10.eb --try-toolchain=goolf,1.4.10-no-OFED --robot
\end{verbatim}

\textbf{STATUS:}

\begin{verbatim}
== COMPLETED: Installation ended successfully
== Results of the build can be found in the log file \
    /home2/wd238/.local/easybuild/software/Stacks/ \
    1.03-goolf-1.4.10-no-OFED/easybuild/ \
    easybuild-Stacks-1.03-20150122.081337.log
== Build succeeded for 1 out of 1
\end{verbatim}

\subsubsection{Install \texttt{Stacks-1.24}}\label{install-stacks-1.24}

\textbf{easyconfig file:}

\begin{itemize}
\itemsep1pt\parskip0pt\parsep0pt
\item
  altered the one generated when building \texttt{Stacks-1.03}

  \begin{itemize}
  \itemsep1pt\parskip0pt\parsep0pt
  \item
    \href{file:///home/gus/remote_mounts/louise/.local/easybuild/ebfiles_repo/Stacks/Stacks-1.03-goolf-1.4.10-no-OFED.eb}{gus@louise/.local/easybuild/ebfiles\_repo/Stacks/Stacks-1.03-goolf-1.4.10-no-OFED.eb}
  \end{itemize}
\item
  \href{file:///home/gus/remote_mounts/louise/scripts/installs/easybuild/Stacks-1.24-goolf-1.4.10-no-OFED.eb}{gus@louise/scripts/installs/easybuild/Stacks-1.24-goolf-1.4.10-no-OFED.eb}
\end{itemize}

\paragraph{Attempt 01}\label{attempt-01}

\begin{verbatim}
wd238 at compute-21-15 in ~ (py278)
eb Stacks-1.24-goolf-1.4.10-no-OFED.eb --try-toolchain=goolf,1.4.10-no-OFED --robot
\end{verbatim}

\textbf{STATUS:} FAILED

\begin{itemize}
\itemsep1pt\parskip0pt\parsep0pt
\item
  couldn't find \texttt{Stacks-1.24-goolf-1.4.10-no-OFED.eb}
\item
  basically expected.
\end{itemize}

\paragraph{Attempt 02}\label{attempt-02}

\begin{verbatim}
wd238 at compute-21-15 in ~ (py278)
eb $HOME/.local/easybuild/ebfiles_repo/Stacks/Stacks-1.24-goolf-1.4.10-no-OFED.eb \
    --try-toolchain=goolf,1.4.10-no-OFED --robot
\end{verbatim}

\textbf{STATUS:} FAILED

\begin{itemize}
\itemsep1pt\parskip0pt\parsep0pt
\item
  error log:
  \href{file:///home/gus/remote_mounts/louise/scripts/installs/easybuild/failure_logs/easybuild-Stacks-1.24-20150122.090021.pPmjV.log}{gus@louise/scripts/installs/easybuild/failure\_logs/easybuild-Stacks-1.24-20150122.090021.pPmjV.log}
\item
  looks like it cant find \texttt{sparsehash} for the linking
\item
  will add \texttt{samtools} and \texttt{sparsehash} to the
  \texttt{easyconfig} file as dependencies and or build dependencies.
\end{itemize}

\paragraph{Attempt 03}\label{attempt-03}

\begin{itemize}
\itemsep1pt\parskip0pt\parsep0pt
\item
  added the below to the \texttt{easyconfig} file:

  \begin{itemize}
  \itemsep1pt\parskip0pt\parsep0pt
  \item
    \texttt{builddependencies = {[}(\textquotesingle{}SAMtools\textquotesingle{}, \textquotesingle{}1.1\textquotesingle{}), (\textquotesingle{}google-sparsehash\textquotesingle{}, \textquotesingle{}2.0.2\textquotesingle{}){]}}
  \end{itemize}
\end{itemize}

\begin{verbatim}
wd238 at compute-21-15 in ~ (py278)
eb $HOME/.local/easybuild/ebfiles_repo/Stacks/Stacks-1.24-goolf-1.4.10-no-OFED.eb \
    --try-toolchain=goolf,1.4.10-no-OFED --robot

== temporary log file in case of crash /tmp/easybuild-0T7khn/easybuild-O8eCl9.log
ERROR: EasyBuild crashed with an error \
    (at easybuild/software/EasyBuild/1.16.1/lib/python2.7/site-packages/\
    easybuild_framework-1.16.1-py2.7.egg/easybuild/tools/robot.py:232 in \
    resolve_dependencies): Irresolvable dependencies encountered: \
    SAMtools/1.1-goolf-1.4.10-no-OFED, google-sparsehash/2.0.2-goolf-1.4.10-no-OFED
\end{verbatim}

\textbf{STATUS:} FAILED

\begin{itemize}
\itemsep1pt\parskip0pt\parsep0pt
\item
  error log:
  \href{file:///home/gus/remote_mounts/louise/scripts/installs/easybuild/failure_logs/easybuild-O8eCl9.log}{gus@louise/scripts/installs/easybuild/failure\_logs/easybuild-O8eCl9.log}
\end{itemize}

\paragraph{IRC session with
author/devs}\label{irc-session-with-authordevs}

\begin{itemize}
\itemsep1pt\parskip0pt\parsep0pt
\item
  one problem was that I dont need to keep using
  \texttt{-\/-try-toolchain=goolf,1.4.10-no-OFED} since the local
  easyconfig (\texttt{Stacks-1.24-goolf-1.4.10-no-OFED.eb}) being passed
  to \texttt{eb} already defines the toolchain.
\item
  the build still fails however
\end{itemize}

\subsubsection{Install \texttt{zlib-1.2.8}}\label{install-zlib-1.2.8}

\textbf{NOTE:}

\begin{itemize}
\itemsep1pt\parskip0pt\parsep0pt
\item
  this is because things seem to need it when building
  \texttt{stacks-1.24}?
\item
  doesn't seem like this was the case?
\end{itemize}

\begin{verbatim}
wd238 at compute-21-15 in ~ (py278) 
$ eb zlib-1.2.8-goolf-1.4.10.eb --try-toolchain=goolf,1.4.10-no-OFED --robot
\end{verbatim}

\textbf{STATUS:} SUCCESSFUL

\begin{verbatim}
== COMPLETED: Installation ended successfully
== Results of the build can be found in the log file \
    /home2/wd238/.local/easybuild/software/zlib/\
    1.2.8-goolf-1.4.10-no-OFED/easybuild/\
    easybuild-zlib-1.2.8-20150122.115448.log
== Build succeeded for 1 out of 1
\end{verbatim}

\begin{center}\rule{0.5\linewidth}{\linethickness}\end{center}

\section{2015-01-23 (Friday)}\label{friday-3}

\subsection{Meeting with Alexis}\label{meeting-with-alexis}

\begin{itemize}
\itemsep1pt\parskip0pt\parsep0pt
\item
  10:00 to 12:20
\item
  talked about overall project and helped her come up with stuff to talk
  about for 2 minutes next Tuesday.
\item
  Gisella was supposed to be here but double booked the time so will
  meet with Alexis individually.
\item
  Gave Alexis my email and asked her to email me so that i would get
  hers
\item
  so far {[}17:00{]} has not emailed me.
\end{itemize}

\subsection{EasyBuild installs}\label{easybuild-installs-2}

\begin{itemize}
\itemsep1pt\parskip0pt\parsep0pt
\item
  forked and cloned
  \href{https://github.com/xguse/easybuild-easyconfigs}{easybuild-easyconfigs}
  git repo
\end{itemize}

\subsection{Doc appointment}\label{doc-appointment}

\begin{itemize}
\itemsep1pt\parskip0pt\parsep0pt
\item
  left desk around 14:00 and got back around 15:20
\end{itemize}

\begin{center}\rule{0.5\linewidth}{\linethickness}\end{center}

\section{2015-01-24 (Saturday)}\label{saturday-2}

\subsection{Sarah sprained/broke? her
ankle}\label{sarah-sprainedbroke-her-ankle}

\begin{itemize}
\itemsep1pt\parskip0pt\parsep0pt
\item
  Sarah slipped while trying to shovel snow (?! WHY ?!)
\item
  had to go back home soon after arrival
\end{itemize}

\begin{center}\rule{0.5\linewidth}{\linethickness}\end{center}

\section{2015-01-25 (Sunday)}\label{sunday-2}

\subsection{EasyBuild installs}\label{easybuild-installs-3}

\begin{itemize}
\item
  adding new forked git-repo of easyconfigs to \texttt{easybuild}
  through environment variables in my
  \href{file:///home/gus/remote_mounts/louise/.zshrc}{gus@louise/.zshrc}.
\item
  \texttt{make check} failing was caused by the easyconfig file setting
  \texttt{runtest=\textquotesingle{}check\textquotesingle{}}.

  \begin{itemize}
  \itemsep1pt\parskip0pt\parsep0pt
  \item
    I removed this
  \end{itemize}
\item
  install still fails but seems to try to repeat itself and fails the
  SECOND TIME?

  \begin{itemize}
  \item
\begin{verbatim}
...
== building and installing Stacks/1.24-goolf-1.4.10-no-OFED...
== fetching files...
== creating build dir, resetting environment...
== unpacking...
== patching...
== preparing...
== configuring...
== building...
== testing...
== installing...
== creating build dir, resetting environment...
== unpacking...
== patching...
== preparing...
== configuring...
== building...
== FAILED: Installation ended unsuccessfully...
...
\end{verbatim}
  \end{itemize}
\item
  checking the logs seems to suggest that the BAM include files are
  still not working

  \begin{itemize}
  \itemsep1pt\parskip0pt\parsep0pt
  \item
    testing by removing the reqs from the configure script

    \begin{itemize}
    \itemsep1pt\parskip0pt\parsep0pt
    \item
      NO sparsehash and NO bam: \textbf{SUCCEEDS}
    \item
      YES sparsehash and NO bam: \textbf{SUCCEEDS}
    \item
      NO sparsehash and YES bam: \textbf{FAILS}
    \end{itemize}
  \end{itemize}
\item
  \textbf{TO TRY TOMORROW:}

  \begin{itemize}
  \itemsep1pt\parskip0pt\parsep0pt
  \item
    `clone' environment from one of the ``test\_reports'' in
    \href{file:///home/gus/remote_mounts/louise/scripts/installs/easybuild/failure_logs/}{failure\_logs}
    and try to run \texttt{./configure; make; make install;} manually.
  \end{itemize}
\end{itemize}

\begin{center}\rule{0.5\linewidth}{\linethickness}\end{center}

\section{2015-01-26 (Monday)}\label{monday-3}

\subsection{Carl Zimmer Writing
Workshop}\label{carl-zimmer-writing-workshop}

\begin{itemize}
\itemsep1pt\parskip0pt\parsep0pt
\item
  10:00 to 12:00
\item
  Notes made in notebook to be transfered here when I have time
  (blizzard approaching)
\end{itemize}

\begin{center}\rule{0.5\linewidth}{\linethickness}\end{center}

\section{2015-01-27 (Tuesday)}\label{tuesday-3}

\subsection{SNOW-pocalypse}\label{snow-pocalypse}

\begin{itemize}
\itemsep1pt\parskip0pt\parsep0pt
\item
  was told to stay home by Yale
\item
  came in by mistake but left soon after realizing
\end{itemize}

\begin{center}\rule{0.5\linewidth}{\linethickness}\end{center}

\section{2015-01-28 (Wednesday)}\label{wednesday-2}

\subsection{EasyBuild installs}\label{easybuild-installs-4}

\subsubsection{Stacks}\label{stacks-1}

\paragraph{Stacks no BAM}\label{stacks-no-bam}

\begin{itemize}
\itemsep1pt\parskip0pt\parsep0pt
\item
  testing to make sure it works
\item
  abandoning this bc we decided that we dont need to run this step over
  right now
\end{itemize}

\paragraph{Stacks yes BAM}\label{stacks-yes-bam}

\begin{itemize}
\itemsep1pt\parskip0pt\parsep0pt
\item
  still not building correctly
\end{itemize}

\subsection{tmux}\label{tmux}

\begin{itemize}
\itemsep1pt\parskip0pt\parsep0pt
\item
  starting point
  \href{https://github.com/fgeorgatos/easybuild.experimental/blob/539bd104d158c9f41b45d60115f6bf1b7155e11e/contrib/pkgsrc/20141219/t/tmux-1.9a-goolf-1.4.10.eb}{easyconfig}
\item
  abandoning this for now
\item
  simply not crucial
\end{itemize}

\subsection{ddRAD stuff}\label{ddrad-stuff}

\begin{itemize}
\itemsep1pt\parskip0pt\parsep0pt
\item
  \texttt{{[}X{]}} email Mark about Tryp LD analysis in his paper

  \begin{itemize}
  \itemsep1pt\parskip0pt\parsep0pt
  \item
    \texttt{{[}X{]}} follow up with him on this ``Re: Short meeting to
    chat about genomic scale LD analysis?''
  \end{itemize}
\item
  \texttt{{[}X{]}} read Tryp paper for same
\item
  \texttt{{[}X{]}} install latest \texttt{Stacks} version on
  \texttt{louise} and make runable by Andrea
\item
  \texttt{{[}X{]}} return \texttt{hapFLK} script to original code and
  copy altered version to new name

  \begin{itemize}
  \itemsep1pt\parskip0pt\parsep0pt
  \item
    \texttt{{[}X{]}} let Andrea know (\emph{acknowledged})
  \end{itemize}
\item
  \texttt{{[}ip{]}} Generate descriptive statistics and figures of the
  LD results as a whole rather than by contig where possible
\end{itemize}

\begin{center}\rule{0.5\linewidth}{\linethickness}\end{center}

\section{2015-01-29 (Thursday)}\label{thursday-3}

\subsection{ddRAD stuff}\label{ddrad-stuff-1}

\begin{itemize}
\itemsep1pt\parskip0pt\parsep0pt
\item
  \texttt{{[}ip{]}} Generate descriptive statistics and figures of the
  LD results as a whole rather than by contig where possible
\item
  \href{file:///home/gus/Dropbox/common/ipy_notebooks/YALE/ddrad58/2015-01-28_Plot_PLINK_results_cumulative.ipynb}{2015-01-28\_Plot\_PLINK\_results\_cumulative.ipynb}
\end{itemize}

\subsection{Robert's stuff}\label{roberts-stuff}

\begin{itemize}
\itemsep1pt\parskip0pt\parsep0pt
\item
  \texttt{{[}-ip-{]}} Pick out 26 flies (13 M, 13 F) from each area we
  want to look at for Robert's work in March

  \begin{itemize}
  \itemsep1pt\parskip0pt\parsep0pt
  \item
    sent first set to Kirstin and Alexis

    \begin{itemize}
    \itemsep1pt\parskip0pt\parsep0pt
    \item
      13 M/13 F from Oyam/Kole trip in 2014-07 (see Table
      \ref{2015-01-29T1})
    \end{itemize}
  \end{itemize}
\item
  \texttt{{[} {]}} make sure we have an updated map with all the
  villages from \emph{Spring and Summer 2014}
\item
  \texttt{{[}X{]}} Meet with Gisella to pick out which locations
  Robert's data will come from while looking at the updated map.

  \begin{itemize}
  \itemsep1pt\parskip0pt\parsep0pt
  \item
    \texttt{{[} {]}} she told me to pick the areas and give her a
    table/report on which and why. The issues to consider are:

    \begin{itemize}
    \itemsep1pt\parskip0pt\parsep0pt
    \item
      wide representation of population areas in the North
    \item
      allows temporal comparisons as well
    \end{itemize}
  \end{itemize}
\end{itemize}

\begin{longtable}[c]{@{}cccccc@{}}
\caption{Samples given to Alexis for DNA extraction from a single leg.
\label{2015-01-29T1}}\tabularnewline
\toprule
Collection Date & Species & Sex & Teneral & Village & Fly\tabularnewline
\midrule
\endfirsthead
\toprule
Collection Date & Species & Sex & Teneral & Village & Fly\tabularnewline
\midrule
\endhead
2014-07-15 & \emph{G. f. fuscipes} & F & NT & OD & 017\tabularnewline
2014-07-15 & \emph{G. f. fuscipes} & F & NT & OCA & 031\tabularnewline
2014-07-15 & \emph{G. f. fuscipes} & F & NT & OCA & 039\tabularnewline
2014-07-16 & \emph{G. f. fuscipes} & F & NT & AKA & 045\tabularnewline
2014-07-16 & \emph{G. f. fuscipes} & F & NT & AKA & 052\tabularnewline
2014-07-16 & \emph{G. f. fuscipes} & F & NT & AKA & 056\tabularnewline
2014-07-16 & \emph{G. f. fuscipes} & F & NT & AKA & 062\tabularnewline
2014-07-16 & \emph{G. f. fuscipes} & F & NT & AKA & 068\tabularnewline
2014-07-16 & \emph{G. f. fuscipes} & F & NT & OCA & 092\tabularnewline
2014-07-16 & \emph{G. f. fuscipes} & F & NT & OCA & 100\tabularnewline
2014-07-16 & \emph{G. f. fuscipes} & F & NT & ACA & 120\tabularnewline
2014-07-16 & \emph{G. f. fuscipes} & F & NT & OD & 137\tabularnewline
2014-07-16 & \emph{G. f. fuscipes} & F & NT & OD & 149\tabularnewline
2014-07-15 & \emph{G. f. fuscipes} & M & NT & OD & 020\tabularnewline
2014-07-15 & \emph{G. f. fuscipes} & M & NT & OCA & 022\tabularnewline
2014-07-15 & \emph{G. f. fuscipes} & M & NT & OCA & 025\tabularnewline
2014-07-15 & \emph{G. f. fuscipes} & M & NT & OCA & 026\tabularnewline
2014-07-15 & \emph{G. f. fuscipes} & M & NT & OCA & 035\tabularnewline
2014-07-16 & \emph{G. f. fuscipes} & M & NT & AKA & 049\tabularnewline
2014-07-16 & \emph{G. f. fuscipes} & M & NT & AKA & 063\tabularnewline
2014-07-16 & \emph{G. f. fuscipes} & M & NT & OCA & 095\tabularnewline
2014-07-16 & \emph{G. f. fuscipes} & M & NT & ACA & 116\tabularnewline
2014-07-16 & \emph{G. f. fuscipes} & M & NT & ACA & 125\tabularnewline
2014-07-16 & \emph{G. f. fuscipes} & M & NT & ACA & 129\tabularnewline
2014-07-16 & \emph{G. f. fuscipes} & M & NT & OCA & 146\tabularnewline
2014-07-16 & \emph{G. f. fuscipes} & M & NT & OD & 155\tabularnewline
\bottomrule
\end{longtable}

\subsection{Rob H Jobs}\label{rob-h-jobs}

\subsubsection{Iowa samples}\label{iowa-samples}

\begin{itemize}
\itemsep1pt\parskip0pt\parsep0pt
\item
  \texttt{{[}X{]}} catalog the boxes that we got here at \textbf{ESC}.
\item
  \texttt{{[}X{]}} email Gisella about getting Rob over to \textbf{EPH}
  to start on their samples.
\end{itemize}

\subsubsection{\emph{G. f. fuscipes} samples
2014}\label{g.-f.-fuscipes-samples-2014}

\begin{itemize}
\itemsep1pt\parskip0pt\parsep0pt
\item
  \texttt{{[}X{]}} organize boxes of \emph{carcass} and \emph{midgut}
  samples from 2014-03 to 2014-08 in Gisella freezer by month.
\item
  \texttt{{[}-ip-{]}} standardize the collection spreadsheets to prepare
  for automated import to \texttt{TsetseSampleDB}.
\end{itemize}

\subsection{Updating maps: current trap
locations}\label{updating-maps-current-trap-locations-3}

\subsubsection{\texttt{spartan} dev: GPS
stuff}\label{spartan-dev-gps-stuff-3}

\begin{itemize}
\itemsep1pt\parskip0pt\parsep0pt
\item
  \texttt{{[}-ip-{]}} teaching \texttt{GPSCoordTree} how to get mean
  coordinates

  \begin{itemize}
  \itemsep1pt\parskip0pt\parsep0pt
  \item
    \texttt{{[}X{]}} make \texttt{GPSCoord} hashable
  \item
    \texttt{{[}-ip-{]}} fix \texttt{GPSCoordTree.\_add\_levels()}:
    getting an empty list somewhere o something that is causing a
    \texttt{None} to be returned.
  \end{itemize}
\item
  still not fixed but trying a new tactic

  \begin{itemize}
  \itemsep1pt\parskip0pt\parsep0pt
  \item
    converting \texttt{GPSCoordTree} to use autovivification trick with
    an extra key at each node that holds links to all \texttt{gps\_objs}
    found below it.
  \item
    see
    \href{http://nbviewer.ipython.org/github/xguse/ipy_notebooks/blob/master/YALE/ddrad58/2014-12-26_functional_annotation_table_generator.ipynb}{2014-12-26\_functional\_annotation\_table\_generator.ipynb}
    for example of autovivfication method.
  \end{itemize}
\end{itemize}

\end{document}
