\documentclass[letterpaper]{scrartcl}
\usepackage{lmodern}
\usepackage{amssymb,amsmath}
\usepackage{ifxetex,ifluatex}
\usepackage{fixltx2e} % provides \textsubscript
\ifnum 0\ifxetex 1\fi\ifluatex 1\fi=0 % if pdftex
  \usepackage[T1]{fontenc}
  \usepackage[utf8]{inputenc}
\else % if luatex or xelatex
  \ifxetex
    \usepackage{mathspec}
    \usepackage{xltxtra,xunicode}
  \else
    \usepackage{fontspec}
  \fi
  \defaultfontfeatures{Mapping=tex-text,Scale=MatchLowercase}
  \newcommand{\euro}{€}
\fi
% use upquote if available, for straight quotes in verbatim environments
\IfFileExists{upquote.sty}{\usepackage{upquote}}{}
% use microtype if available
\IfFileExists{microtype.sty}{%
\usepackage{microtype}
\UseMicrotypeSet[protrusion]{basicmath} % disable protrusion for tt fonts
}{}
\usepackage{color}
\usepackage{fancyvrb}
\newcommand{\VerbBar}{|}
\newcommand{\VERB}{\Verb[commandchars=\\\{\}]}
\DefineVerbatimEnvironment{Highlighting}{Verbatim}{commandchars=\\\{\}}
% Add ',fontsize=\small' for more characters per line
\newenvironment{Shaded}{}{}
\newcommand{\KeywordTok}[1]{\textcolor[rgb]{0.00,0.44,0.13}{\textbf{{#1}}}}
\newcommand{\DataTypeTok}[1]{\textcolor[rgb]{0.56,0.13,0.00}{{#1}}}
\newcommand{\DecValTok}[1]{\textcolor[rgb]{0.25,0.63,0.44}{{#1}}}
\newcommand{\BaseNTok}[1]{\textcolor[rgb]{0.25,0.63,0.44}{{#1}}}
\newcommand{\FloatTok}[1]{\textcolor[rgb]{0.25,0.63,0.44}{{#1}}}
\newcommand{\CharTok}[1]{\textcolor[rgb]{0.25,0.44,0.63}{{#1}}}
\newcommand{\StringTok}[1]{\textcolor[rgb]{0.25,0.44,0.63}{{#1}}}
\newcommand{\CommentTok}[1]{\textcolor[rgb]{0.38,0.63,0.69}{\textit{{#1}}}}
\newcommand{\OtherTok}[1]{\textcolor[rgb]{0.00,0.44,0.13}{{#1}}}
\newcommand{\AlertTok}[1]{\textcolor[rgb]{1.00,0.00,0.00}{\textbf{{#1}}}}
\newcommand{\FunctionTok}[1]{\textcolor[rgb]{0.02,0.16,0.49}{{#1}}}
\newcommand{\RegionMarkerTok}[1]{{#1}}
\newcommand{\ErrorTok}[1]{\textcolor[rgb]{1.00,0.00,0.00}{\textbf{{#1}}}}
\newcommand{\NormalTok}[1]{{#1}}
\usepackage{graphicx}
\makeatletter
\def\maxwidth{\ifdim\Gin@nat@width>\linewidth\linewidth\else\Gin@nat@width\fi}
\def\maxheight{\ifdim\Gin@nat@height>\textheight\textheight\else\Gin@nat@height\fi}
\makeatother
% Scale images if necessary, so that they will not overflow the page
% margins by default, and it is still possible to overwrite the defaults
% using explicit options in \includegraphics[width, height, ...]{}
\setkeys{Gin}{width=\maxwidth,height=\maxheight,keepaspectratio}
\ifxetex
  \usepackage[setpagesize=false, % page size defined by xetex
              unicode=false, % unicode breaks when used with xetex
              xetex]{hyperref}
\else
  \usepackage[unicode=true]{hyperref}
\fi
\hypersetup{breaklinks=true,
            bookmarks=true,
            pdfauthor={Gus Dunn},
            pdftitle={Daily Records},
            colorlinks=true,
            citecolor=blue,
            urlcolor=blue,
            linkcolor=magenta,
            pdfborder={0 0 0}}
\urlstyle{same}  % don't use monospace font for urls
\setlength{\parindent}{0pt}
\setlength{\parskip}{6pt plus 2pt minus 1pt}
\setlength{\emergencystretch}{3em}  % prevent overfull lines
\setcounter{secnumdepth}{5}

\title{Daily Records\\\vspace{0.5em}{\large Caccone PostDoc}}
\author{Gus Dunn}
\date{February, 2015}
\usepackage{bbding}
\usepackage[T1]{fontenc}
\usepackage{lxfonts}

% blockquote
\usepackage{xcolor}
\definecolor{greyborder}{RGB}{221,221,221}
\definecolor{greytext}{RGB}{119,119,119}
\usepackage{mdframed}
\newmdenv[rightline=false,bottomline=false,topline=false,linewidth=3pt,linecolor=greyborder,skipabove=\parskip]{blockquote}
\renewenvironment{quote}{\begin{blockquote}\list{}{\rightmargin=0em\leftmargin=0em}%
\item\relax\color{greytext}\ignorespaces}{\unskip\unskip\endlist\end{blockquote}}


\begin{document}
\maketitle

{
\hypersetup{linkcolor=black}
\setcounter{tocdepth}{3}
\tableofcontents
}
\begin{center}\rule{0.5\linewidth}{\linethickness}\end{center}

\section{2015-02-01 (Sunday)}\label{sunday}

\subsection{Updating maps: current trap
locations}\label{updating-maps-current-trap-locations}

\subsubsection{\texttt{spartan} dev: GPS
stuff}\label{spartan-dev-gps-stuff}

\begin{itemize}
\itemsep1pt\parskip0pt\parsep0pt
\item
  writing autovivification version of
  \texttt{GPSCoordTree.\_grow\_branch}.
\end{itemize}

\begin{center}\rule{0.5\linewidth}{\linethickness}\end{center}

\section{2015-02-02 (Monday)}\label{monday}

\subsection{Updating maps: current trap
locations}\label{updating-maps-current-trap-locations-1}

\subsubsection{\texttt{spartan} dev: GPS
stuff}\label{spartan-dev-gps-stuff-1}

\begin{itemize}
\itemsep1pt\parskip0pt\parsep0pt
\item
  testing
  (\href{file:///home/gus/Dropbox/repos/git/spartan/src/spartan/tests/test_utils_maps_gps.py}{test\_utils\_maps\_gps.py}):

  \begin{itemize}
  \itemsep1pt\parskip0pt\parsep0pt
  \item
    \texttt{{[}x{]}} \texttt{GPSCoordTree.\_grow\_branch}
  \item
    \texttt{{[}x{]}} \texttt{GPSCoordTree.\_get\_subtree}
  \item
    \texttt{{[}x{]}} \texttt{GPSCoordTree.mean}
  \end{itemize}
\end{itemize}

\subsection{Creating Uganda Data Repo}\label{creating-uganda-data-repo}

\begin{itemize}
\item
  \textbf{local location:}

  \href{file:///home/gus/Dropbox/uganda_data/data_repos/field_data}{/home/gus/Dropbox/uganda\_data/data\_repos/field\_data}
\item
  \textbf{github address:}
  \url{https://github.com/CacconeLabYale/field_data.git}
\end{itemize}

\begin{center}\rule{0.5\linewidth}{\linethickness}\end{center}

\section{2015-02-03 (Tuesday)}\label{tuesday}

\subsection{Updating maps: current trap
locations}\label{updating-maps-current-trap-locations-2}

\begin{itemize}
\itemsep1pt\parskip0pt\parsep0pt
\item
  established comprehensive lists of village-ID-map and trap GPS
  locations for Uganda:

  \begin{itemize}
  \item
    \textbf{village-ID-map:}

    \href{file:///home/gus/Dropbox/uganda_data/data_repos/field_data/locations/names/uganda_village_id_map.csv}{field\_data/locations/names/uganda\_village\_id\_map.csv}
  \item
    \textbf{trap GPS coords:}

    \href{file:///home/gus/Dropbox/uganda_data/data_repos/field_data/locations/gps/traps/uganda_traps_gps.csv}{field\_data/locations/gps/traps/uganda\_traps\_gps.csv}
  \end{itemize}
\end{itemize}

\begin{center}\rule{0.5\linewidth}{\linethickness}\end{center}

\section{2015-02-04 (Wednesday)}\label{wednesday}

\subsection{General ToDo}\label{general-todo}

\begin{itemize}
\itemsep1pt\parskip0pt\parsep0pt
\item
  \texttt{{[}x{]}} email to confirm HR got my letter
\item
  \texttt{{[}x{]}} meet with Gisella and Andrea {[}1130{]}

  \begin{itemize}
  \itemsep1pt\parskip0pt\parsep0pt
  \item
    \texttt{{[}X{]}} write up notes from meeting:
    \href{file:///home/gus/Dropbox/repos/git/markdown-docs/notes/meetings/gisella_andrea_2015-02-04/gisella_andrea_2015-02-04.pdf}{gisella\_andrea\_2015-02-04.pdf}
  \end{itemize}
\item
  \texttt{{[}x{]}} Talk to Ben E about the MAD idea.
\item
  \texttt{{[}x{]}} create git repo for this paper
\item
  \texttt{{[} {]}} begin development of the MAD idea
\item
  \texttt{{[}X{]}} install LDna and R-studio
\item
  \texttt{{[}X{]}} Located space to move the EPH \emph{G. pallidipes}
  samples here to ESC with Rob
\end{itemize}

\subsection{ddRAD stuff}\label{ddrad-stuff}

\subsubsection{LD: detect `outlier'
SNP-pairs}\label{ld-detect-outlier-snp-pairs}

\begin{itemize}
\itemsep1pt\parskip0pt\parsep0pt
\item
  \textbf{I propose this method:}

  \begin{enumerate}
  \def\labelenumi{\arabic{enumi}.}
  \itemsep1pt\parskip0pt\parsep0pt
  \item
    for each distance group: collect \(r^2\) from \(\pm \sim5\) bp
    distance window

    \begin{enumerate}
    \def\labelenumii{\alph{enumii}.}
    \itemsep1pt\parskip0pt\parsep0pt
    \item
      across genome
    \item
      across scaffold
    \end{enumerate}
  \item
    calculate modified z-score (based on \emph{median absolute
    deviation} rather than standard deviation: \textbf{MAD is more
    robust than SD for HTS-type data})
  \item
    flag any SNP-pair with \(z \geq 3.5\)
  \item
    \emph{possibly randomize data and calculate FDR to evaluate
    performance.}

    \begin{enumerate}
    \def\labelenumii{\alph{enumii}.}
    \itemsep1pt\parskip0pt\parsep0pt
    \item
      perhaps vary the window from step 1 to use FDR to chose window
      that minimizes FDR.
    \end{enumerate}
  \end{enumerate}
\item
  \textbf{Ben E's thoughts:}

  \begin{itemize}
  \itemsep1pt\parskip0pt\parsep0pt
  \item
    basically: this is probably a waste of time and energy

    \begin{itemize}
    \itemsep1pt\parskip0pt\parsep0pt
    \item
      other more sophisticated methods have already been applied to this
      data with not much significance detected
    \item
      why do we expect this work to yield better/more results?
    \end{itemize}
  \end{itemize}
\item
  \textbf{Gisella's thoughts:}

  \begin{itemize}
  \itemsep1pt\parskip0pt\parsep0pt
  \item
    still should do it bc we will need it when we have more data
  \end{itemize}
\end{itemize}

\subsubsection{Install LDna}\label{install-ldna}

\begin{itemize}
\item
  github page:
  \href{https://github.com/petrikemppainen/LDna}{github.com/petrikemppainen/LDna}
\item
  paper reference:
  \url{http://onlinelibrary.wiley.com/doi/10.1111/1755-0998.12369/abstract}
\item
  installed \texttt{devtools} with \texttt{RStudio} gui:
  \textbf{{[}successful{]}}
\item
  installed \texttt{LDna} with \texttt{devtools}:
  \textbf{{[}successful{]}}

\begin{Shaded}
\begin{Highlighting}[]
\NormalTok{devtools::}\KeywordTok{install_github}\NormalTok{(}\StringTok{"petrikemppainen/LDna"}\NormalTok{)}
\end{Highlighting}
\end{Shaded}

  \begin{itemize}
  \itemsep1pt\parskip0pt\parsep0pt
  \item
    documentation:
    \href{file:///home/gus/R/x86_64-unknown-linux-gnu-library/3.1/LDna/html/00Index.html}{LDna/html/00Index.html}
  \end{itemize}
\end{itemize}

\subsubsection{LDna notes}\label{ldna-notes}

\begin{itemize}
\item
  operates on:

  \begin{quote}
  Lower diagonal matrix of pairwise LD values, \(r^2\) is strongly
  recommended
  \end{quote}
\item
  the code below should generate what I want (\textbf{I think}):
\end{itemize}

\begin{Shaded}
\begin{Highlighting}[]
\KeywordTok{plink} \NormalTok{--vcf tsetseFINAL_14Oct2014_f2_53.recode.renamed_scaffolds.maf0_05.vcf \textbackslash{}}
\KeywordTok{--allow-extra-chr} \NormalTok{\textbackslash{}}
\KeywordTok{--r2} \NormalTok{bin  \textbackslash{}}
\KeywordTok{--out} \NormalTok{plink_out/tsetseFINAL_14Oct2014_f2_53.recode.renamed_scaffolds.maf0_05.vcf/ld/r2_bin}
\end{Highlighting}
\end{Shaded}

\subsubsection{PLINK run for LDna}\label{plink-run-for-ldna}

\begin{itemize}
\itemsep1pt\parskip0pt\parsep0pt
\item
  ran the command below:
\end{itemize}

\begin{verbatim}
wd238 at compute-23-2 in ~GENOMES/glossina_fuscipes/annotations/SNPs (py278)
$ plink --vcf tsetseFINAL_14Oct2014_f2_53.recode.renamed_scaffolds.maf0_05.vcf \
> --allow-extra-chr \
> --r2 bin  \
> --out plink_out/tsetseFINAL_14Oct2014_f2_53.recode.renamed_scaffolds.maf0_05.vcf/ld/r2_bin
\end{verbatim}

\begin{itemize}
\itemsep1pt\parskip0pt\parsep0pt
\item
  waiting for it to finish: \textbf{{[}failed{]}}
\end{itemize}

\subsubsection{Louise Scratch Request
Email}\label{louise-scratch-request-email}

\begin{quote}
\textbf{netid:} wd238 \textbf{group:} caccone \textbf{anticipated
usage:}

\begin{itemize}
\item
  \textasciitilde{} 100G
\item
  \textless{} 100 files \textbf{purpose of usage:}
\item
  running \texttt{plink} \emph{all\_v\_all} linkage disequilibrium
  calculations on \textasciitilde{}40K SNPs
\item
  current attempt (documented below) gave a write failure which I think
  may be bc of some rather large tmp files generated during the process?
\item
  Does bumping up against our space quota have hard/immediate
  consequences like that?
\end{itemize}

\textbf{error log:}

\begin{verbatim}
wd238 at compute-23-2 in ~GENOMES/glossina_fuscipes/annotations/SNPs (py278)
$ plink --vcf tsetseFINAL_14Oct2014_f2_53.recode.renamed_scaffolds.maf0_05.vcf \
> --allow-extra-chr \
> --r2 bin  \
> --out plink_out/tsetseFINAL_14Oct2014_f2_53.recode.renamed_scaffolds.maf0_05.vcf/ld/r2_bin
PLINK v1.90b2o 64-bit (25 Nov 2014)        https://www.cog-genomics.org/plink2
(C) 2005-2014 Shaun Purcell, Christopher Chang   GNU General Public License v3
Logging to plink_out/tsetseFINAL_14Oct2014_f2_53.recode.renamed_scaffolds.maf0_05.vcf/ld/r2_bin.log.
516842 MB RAM detected; reserving 258421 MB for main workspace.
--vcf: 73k variants complete.
plink_out/tsetseFINAL_14Oct2014_f2_53.recode.renamed_scaffolds.maf0_05.vcf/ld/r2_bin-temporary.bed
+
plink_out/tsetseFINAL_14Oct2014_f2_53.recode.renamed_scaffolds.maf0_05.vcf/ld/r2_bin-temporary.bim
+
plink_out/tsetseFINAL_14Oct2014_f2_53.recode.renamed_scaffolds.maf0_05.vcf/ld/r2_bin-temporary.fam
written.
73297 variants loaded from .bim file.
53 people (0 males, 0 females, 53 ambiguous) loaded from .fam.
Ambiguous sex IDs written to
plink_out/tsetseFINAL_14Oct2014_f2_53.recode.renamed_scaffolds.maf0_05.vcf/ld/r2_bin.nosex
.
Using up to 63 threads (change this with --threads).
Before main variant filters, 53 founders and 0 nonfounders present.
Calculating allele frequencies... done.
Total genotyping rate is 0.965098.
73297 variants and 53 people pass filters and QC.
Note: No phenotypes present.
--r2 square bin to
plink_out/tsetseFINAL_14Oct2014_f2_53.recode.renamed_scaffolds.maf0_05.vcf/ld/r2_bin.ld.bin
... done.

Error: File write failure.
\end{verbatim}
\end{quote}

\subsubsection{Github repo for this
paper}\label{github-repo-for-this-paper}

\begin{itemize}
\itemsep1pt\parskip0pt\parsep0pt
\item
  github
  page:\\\url{https://github.com/CacconeLabYale/gloria_soria_ddRAD_2015.git}
\end{itemize}

\begin{center}\rule{0.5\linewidth}{\linethickness}\end{center}

\section{2015-02-05 (Thursday)}\label{thursday}

\subsection{Mariangela \texttt{blacktie}
install}\label{mariangela-blacktie-install}

\begin{itemize}
\itemsep1pt\parskip0pt\parsep0pt
\item
  turns out i did NOT send Mariangela install instructions for the
  development branch
\item
  wrote quick install script for her to use and sent it
\end{itemize}

\subsection{MAD idea}\label{mad-idea}

\begin{enumerate}
\def\labelenumi{\arabic{enumi}.}
\itemsep1pt\parskip0pt\parsep0pt
\item
  for each group of SNPs \(x\) bp apart: collect \(r^2\) from
  \(\pm \sim5\) bp distance window around \(x\):

  \begin{enumerate}
  \def\labelenumii{\alph{enumii}.}
  \itemsep1pt\parskip0pt\parsep0pt
  \item
    across genome
  \item
    across scaffold
  \end{enumerate}
\item
  calculate modified z-score (based on \emph{median absolute deviation}
  rather than standard deviation: \textbf{MAD is more robust than SD for
  HTS-type data})
\item
  flag any SNP-pair with \(z \geq 3.5\)
\item
  possibly randomize data and calculate FDR to evaluate performance.

  \begin{enumerate}
  \def\labelenumii{\alph{enumii}.}
  \itemsep1pt\parskip0pt\parsep0pt
  \item
    perhaps vary the window-size from step 1 to use FDR to chose
    window-size that minimizes FDR.
  \end{enumerate}
\end{enumerate}

\subsubsection{Development}\label{development}

\begin{itemize}
\itemsep1pt\parskip0pt\parsep0pt
\item
  ipython notebook:
  \href{file:///home/gus/Dropbox/common/ipy_notebooks/YALE/ddrad58/2015-02-05_MAD_idea.ipynb}{ddrad58/2015-02-05\_MAD\_idea.ipynb}
\end{itemize}

\subsection{\emph{G. pallidipes}}\label{g.-pallidipes}

\begin{itemize}
\itemsep1pt\parskip0pt\parsep0pt
\item
  Rob brought most to ESC this morning
\item
  doesn't expect to need my truck for the rest
\end{itemize}

\begin{center}\rule{0.5\linewidth}{\linethickness}\end{center}

\section{2015-02-06 (Friday)}\label{friday}

\subsection{MAD idea}\label{mad-idea-1}

\subsubsection{Development}\label{development-1}

\begin{itemize}
\item
  LOTS of progress at ipython notebook:
  \href{file:///home/gus/Dropbox/common/ipy_notebooks/YALE/ddrad58/2015-02-05_MAD_idea.ipynb}{ddrad58/2015-02-05\_MAD\_idea.ipynb}
\item
  See notes about plotting median and MAD with bootstrapped CIs near the
  bottom of above (commit dd7fe5da5733406edeaab6ce3c25b523b94552f2)
\end{itemize}

\begin{center}\rule{0.5\linewidth}{\linethickness}\end{center}

\section{2015-02-09 (Monday)}\label{monday-1}

\textbf{Goals:}

\begin{itemize}
\itemsep1pt\parskip0pt\parsep0pt
\item
  \texttt{{[}x{]}} Zimmer Workshop
\item
  \texttt{{[}x{]}} Start Professional Development notebook
\item
  \texttt{{[}x{]}} Find out how to process health reimbursement

  \begin{itemize}
  \itemsep1pt\parskip0pt\parsep0pt
  \item
    \texttt{{[} {]}} Get them ready for mailing

    \begin{itemize}
    \itemsep1pt\parskip0pt\parsep0pt
    \item
      \texttt{{[}x{]}} form
    \item
      \texttt{{[} {]}} receipts
    \end{itemize}
  \item
    \texttt{{[}X{]}} Assemble list of information I need from Sarah and
    send it to her
  \end{itemize}
\item
  \texttt{{[} {]}} Progress on MAD idea
\item
  \texttt{{[} {]}} Generate strategy for the week
\item
  \texttt{{[} {]}} Sketch out abstract for Keystone? meeting
\item
  \texttt{{[} {]}} find out if there is data available on tsetse control
  by area in Uganda

  \begin{itemize}
  \itemsep1pt\parskip0pt\parsep0pt
  \item
    chemicals sold
  \item
    etc
  \end{itemize}
\end{itemize}

\subsection{Health reimbursement}\label{health-reimbursement}

\begin{itemize}
\itemsep1pt\parskip0pt\parsep0pt
\item
  \url{http://yalehealth.yale.edu/claims}
\item
  Supplemental Claim form:
  \url{http://yalehealth.yale.edu/sites/default/files/supplemental_claims_form.pdf}
\item
  pharmacy claim form:
  \url{http://yalehealth.yale.edu/sites/default/files/pharmacy_claim_form_restat_catamaran.pdf}
\end{itemize}

\subsubsection{Instructions for pharmacy
process}\label{instructions-for-pharmacy-process}

\begin{itemize}
\itemsep1pt\parskip0pt\parsep0pt
\item
  from website above
\end{itemize}

\begin{quote}
Include copies of prescription receipts showing the following
information:

\begin{itemize}
\itemsep1pt\parskip0pt\parsep0pt
\item
  Pharmacy Name, Address \& Phone Number
\item
  Patient Name
\item
  Prescription Number
\item
  Prescription Fill Date
\item
  Drug Name, Strength and NDC Code
\item
  Drug Quantity \& Days supply
\item
  Drug Cost
\item
  Amount Paid
\end{itemize}

Please mail the Prescription Drug Claim Form and receipts to:

Restat\\Patient Reimbursement\\11900 W. Lake Park Drive\\Milwaukee, WI
53224

Claims are honored for one year from the date of service. If you haven't
received a response to a claim within 60 days of filing, contact the
Claims Department. You may call sooner to inquire if the claim has been
received and is in process.
\end{quote}

\begin{center}\rule{0.5\linewidth}{\linethickness}\end{center}

\section{2015-02-10 (Tuesday)}\label{tuesday-1}

\textbf{Goals:}

\begin{itemize}
\itemsep1pt\parskip0pt\parsep0pt
\item
  \texttt{{[} {]}} Get pharm claim ready for mailing

  \begin{itemize}
  \itemsep1pt\parskip0pt\parsep0pt
  \item
    \texttt{{[}x{]}} form
  \item
    \texttt{{[} {]}} receipts
  \end{itemize}
\item
  \texttt{{[} {]}} Progress on MAD idea
\item
  \texttt{{[} {]}} Generate strategy for the week
\item
  \texttt{{[} {]}} Sketch out abstract for Keystone? meeting
\item
  \texttt{{[} {]}} find out if there is data available on tsetse control
  by area in Uganda

  \begin{itemize}
  \itemsep1pt\parskip0pt\parsep0pt
  \item
    chemicals sold
  \item
    etc
  \end{itemize}
\item
  \texttt{{[} {]}} figure out how to download zimmer files
\end{itemize}

\subsection{Health reimbursement}\label{health-reimbursement-1}

\begin{itemize}
\itemsep1pt\parskip0pt\parsep0pt
\item
  printed form
\end{itemize}

\subsection{Met with Postdoc applicant
(Christina)}\label{met-with-postdoc-applicant-christina}

\begin{itemize}
\itemsep1pt\parskip0pt\parsep0pt
\item
  had lunch
\end{itemize}

\begin{center}\rule{0.5\linewidth}{\linethickness}\end{center}

\section{2015-02-12 (Thursday)}\label{thursday-1}

\subsection{Health reimbursement}\label{health-reimbursement-2}

\begin{itemize}
\itemsep1pt\parskip0pt\parsep0pt
\item
  Need Catherine's member ID
\end{itemize}

\subsection{MAD idea}\label{mad-idea-2}

\subsubsection{Development}\label{development-2}

\begin{itemize}
\itemsep1pt\parskip0pt\parsep0pt
\item
  \emph{yesterday:}

  \begin{itemize}
  \itemsep1pt\parskip0pt\parsep0pt
  \item
    bootstrap confidence intervals are functional
  \item
    modified z-score is functional
  \item
    used ggplot to provide nice figure showing rough progression of
    z-scored \(r^2\) through distance between snps
  \end{itemize}
\end{itemize}

\begin{center}\rule{0.5\linewidth}{\linethickness}\end{center}

\section{2015-02-13 (Friday)}\label{friday-1}

\subsection{\emph{G. pallidipes} Sample
catalog}\label{g.-pallidipes-sample-catalog}

\subsubsection{Summary table}\label{summary-table}

\begin{itemize}
\item
  data types:

  \begin{itemize}
  \itemsep1pt\parskip0pt\parsep0pt
  \item
    location
  \item
    symbols when present (\emph{I assume you mean location symbol?})
  \item
    number of individuals
  \item
    date range
  \item
    is tissue?
  \item
    is extraction?
  \item
    analysis status
  \end{itemize}
\item
  will be done in \texttt{python} for increased flexibility by
  \textbf{{[}Gus{]}}
\item
  notebook file:
  \href{file:///home/gus/Dropbox/common/ipy_notebooks/YALE/pallidipes_kenya/2015-02-12_sample_catalog_summary.ipynb}{2015-02-12\_sample\_catalog\_summary.ipynb}
\item
  Showed output to Gisella and she signed off on it after asking whether
  I could accommodate GEO COORDS when we get them.
\item
  \textbf{STATUS:} \textbf{{[}completed{]}}
\end{itemize}

\subsubsection{Primers etc}\label{primers-etc}

\begin{itemize}
\itemsep1pt\parskip0pt\parsep0pt
\item
  RobH reports that he and KirstinD found many primers etc that were
  either designed for \emph{G. pallidipes} or shown to work with it in
  the past.
\item
  testing on the primers will begin next week.
\end{itemize}

\subsubsection{Leg extractions}\label{leg-extractions}

\begin{itemize}
\itemsep1pt\parskip0pt\parsep0pt
\item
  Rob did Xymogen extractions on 5 legs
\item
  NanoDrop indicates absorption at 260 but peaks look weird

  \begin{itemize}
  \itemsep1pt\parskip0pt\parsep0pt
  \item
    probably bc the kit leaves EVERYTHING still in solution
  \item
    \texttt{{[} {]}} RobH will check with KirstinD about her extraction
    traces on \emph{G. f. fuscipes} legs
  \end{itemize}
\end{itemize}

\subsection{MAD idea}\label{mad-idea-3}

\subsubsection{Development}\label{development-3}

\begin{itemize}
\itemsep1pt\parskip0pt\parsep0pt
\item
  \textbf{{[}completed{]}:} functions to

  \begin{itemize}
  \itemsep1pt\parskip0pt\parsep0pt
  \item
    update df with \texttt{distance\_bin} and \texttt{mad\_z}
  \item
    plot mad\_z by bins
  \end{itemize}
\item
  \textbf{{[}to do{]}:}

  \begin{itemize}
  \itemsep1pt\parskip0pt\parsep0pt
  \item
    implement printing/saving snp-pairs that pass the z-filter
  \end{itemize}
\end{itemize}

\begin{center}\rule{0.5\linewidth}{\linethickness}\end{center}

\section{2015-02-14 (Saturday)}\label{saturday}

\subsection{MAD idea}\label{mad-idea-4}

\subsubsection{Development}\label{development-4}

\begin{itemize}
\itemsep1pt\parskip0pt\parsep0pt
\item
  implement printing/saving snp-pairs that pass the z-filter
\end{itemize}

\begin{center}\rule{0.5\linewidth}{\linethickness}\end{center}

\section{2015-02-16 (Monday)}\label{monday-2}

\subsection{\emph{G. f. fuscipes}: infection
summaries}\label{g.-f.-fuscipes-infection-summaries}

\begin{itemize}
\itemsep1pt\parskip0pt\parsep0pt
\item
  ipython to get pivot table for infected flies

  \begin{itemize}
  \itemsep1pt\parskip0pt\parsep0pt
  \item
    file:
    \href{file:///home/gus/Dropbox/common/ipy_notebooks/YALE/g_f_fuscipes_general/2015-02-16_g_f_fuscipes_pandas_import.ipynb}{2015-02-16\_g\_f\_fuscipes\_pandas\_import.ipynb}

    \begin{itemize}
    \itemsep1pt\parskip0pt\parsep0pt
    \item
      file of dumped pandas table of collection records for 2014 in hdf5
      format:
    \end{itemize}
  \end{itemize}
\item
  add PCR detected fly statuses to main DB
\end{itemize}

\subsection{\emph{G. pallidipes}: MicroSat extraction
pilot}\label{g.-pallidipes-microsat-extraction-pilot}

\begin{itemize}
\itemsep1pt\parskip0pt\parsep0pt
\item
  RobH spoke with KirstinD about strange NanoDrop traces:

  \begin{itemize}
  \itemsep1pt\parskip0pt\parsep0pt
  \item
    KirstinD: hers looked the same, just used 260/280 values as
    presented
  \item
    likely explanation is that the extraction kit is EXTREMELY dirty by
    design so the spec peaks are shifted around
  \end{itemize}
\item
  RobH is beginning PCRs with ITS primers (same that KirstinD is using
  on the \emph{G. f. fuscipes}) today.
\item
  RobH is researching location names on the SerapA tubes (n
  \textasciitilde{} 6) bc GisellaC is not convinced the sheet SerapA
  included makes since.

  \begin{itemize}
  \itemsep1pt\parskip0pt\parsep0pt
  \item
    RobH will google first
  \item
    GusD will get GIS admin layers to search if google fails
  \end{itemize}
\item
  \href{https://github.com/xguse/blacktie/archive/v0.2.1.2-1.tar.gz}{v0.2.1.2-1.tar.gz}
\end{itemize}

\begin{center}\rule{0.5\linewidth}{\linethickness}\end{center}

\section{2015-02-17 (Tuesday)}\label{tuesday-2}

\subsection{meeting}\label{meeting}

\begin{itemize}
\itemsep1pt\parskip0pt\parsep0pt
\item
  escarpment Nguruman:

  \begin{itemize}
  \itemsep1pt\parskip0pt\parsep0pt
  \item
    GisellaC try to get samples from extremes and in the middle
  \end{itemize}
\item
  \texttt{{[} {]}} GusD send most recent version of protocol to BrianW
\end{itemize}

\section{2015-02-18 (Wednesday)}\label{wednesday-1}

\subsection{\emph{G. pallidipes} status update
meeting}\label{g.-pallidipes-status-update-meeting}

\begin{itemize}
\item
  GusD
\item
  RobH
\item
  KirstinD
\item
  extractions not working for a while with KirstinD
\item
  trouble shooting
\item
  KirstinD moving forward with extractions now
\end{itemize}

\begin{center}\rule{0.5\linewidth}{\linethickness}\end{center}

\section{2015-02-21 (Saturday)}\label{saturday-1}

\textbf{GOALS:}

\begin{itemize}
\itemsep1pt\parskip0pt\parsep0pt
\item
  \texttt{{[}worked on{]}} \emph{G. f. fuscipes} infection
  summaries/maps for GisellaC meeting
\item
  \texttt{{[}no work{]}} script for MariangelaB
\item
  \texttt{{[}small work{]}} \(r^2\) per bin model
\end{itemize}

\subsection{\emph{G. f. fuscipes}: infection
summaries}\label{g.-f.-fuscipes-infection-summaries-1}

\subsubsection{Converting dates to
YYYY-MM-DD}\label{converting-dates-to-yyyy-mm-dd}

\begin{itemize}
\itemsep1pt\parskip0pt\parsep0pt
\item
  \href{file:///home/gus/Documents/YalePostDoc/project_stuff/g_f_fucipes_uganda/collection_data/2014_spring_summer_from_rob.xlsx}{2014\_spring\_summer\_from\_rob.xlsx}

  \begin{itemize}
  \itemsep1pt\parskip0pt\parsep0pt
  \item
    added new function to
    \texttt{TsetseCheckout}:\\\href{file:///home/gus/Dropbox/repos/git/TsetseCheckout/TsetseCheckout/data/utils.py}{TsetseCheckout/data/utils.py:convert\_brit\_dates\_to\_yyyy\_mm\_dd(string)}
  \item
    added new cell magic to ipython to send variable to
    clipboard:\\\href{https://gist.github.com/xguse/a01780ef22cfad8adaf9}{clip\_magic.py}
  \item
    used new function and the cell magic to copy, change, then paste
    back into spreadsheet.
  \end{itemize}
\item
  \href{file:///home/gus/Documents/YalePostDoc/project_stuff/g_f_fucipes_uganda/collection_data/2014_fall_for_pandas.xlsx}{2014\_fall\_for\_pandas.xlsx}

  \begin{itemize}
  \itemsep1pt\parskip0pt\parsep0pt
  \item
    dates already fine
  \end{itemize}
\end{itemize}

\subsubsection{Adding Village names to the spring/summer excel
file}\label{adding-village-names-to-the-springsummer-excel-file}

\begin{itemize}
\item
  \textbf{{[}COMPLETED{]}:} 2015-02-22
\item
  created python hack to use the summary sheet info to generate the
  Village
  rows\\\href{/home/gus/Documents/YalePostDoc/project_stuff/g_f_fucipes_uganda/collection_data/traps_to_villages.py}{YalePostDoc/project\_stuff/g\_f\_fucipes\_uganda/collection\_data/traps\_to\_villages.py}

  \begin{itemize}
  \itemsep1pt\parskip0pt\parsep0pt
  \item
    summary
    sheets:\\\href{file:///home/gus/Dropbox/uganda_data/2014_Collection_Sheets_Spring-Summer/2014_full_surveyreport_20140820/summary\%20survey\%20data.xlsx}{2014\_full\_surveyreport\_20140820/summary
    survey data.xlsx}
  \end{itemize}
\end{itemize}

\subsubsection{ALERT: errors detected in fly name code
combinations}\label{alert-errors-detected-in-fly-name-code-combinations}

\begin{itemize}
\item
  during this process i detected instances where the fly number code
  combinations (example: \texttt{OLW-14 038}) were \textbf{NOT} correct!
\item
  the following IDs illustrate this:

  \begin{itemize}
  \itemsep1pt\parskip0pt\parsep0pt
  \item
    \texttt{OLO-14 033} is Olobo
  \item
    \texttt{OLO-14 034} is Olobo
  \item
    \texttt{OLW-14 035} is Olwi
  \item
    \texttt{OLW-14 036} is Olwi
  \item
    \texttt{OLW-14 037} is Olobo
  \item
    \texttt{OLW-14 038} is Olobo
  \end{itemize}
\item
  additionally, the \texttt{Dissection Data-Kole} sheet has \textbf{ALL}
  fly IDs starting \texttt{KO} regardless of the source village.
\item
  \textbf{RECOMEND \emph{NOT} DEPENDING ON FLY ID FOR VILLAGE SOURCE!}
\end{itemize}

\begin{center}\rule{0.5\linewidth}{\linethickness}\end{center}

\section{2015-02-22 (Sunday)}\label{sunday-1}

\textbf{GOALS:}

\begin{itemize}
\itemsep1pt\parskip0pt\parsep0pt
\item
  \texttt{{[}worked on{]}} \emph{G. f. fuscipes} infection
  summaries/maps for GisellaC meeting
\item
  \texttt{{[}none{]}} script for MariangelaB
\item
  \texttt{{[}none{]}} \(r^2\) per bin model
\end{itemize}

\subsection{\emph{G. f. fuscipes}: infection
summaries}\label{g.-f.-fuscipes-infection-summaries-2}

\subsubsection{HDF5 import and data
cleaning}\label{hdf5-import-and-data-cleaning}

\begin{itemize}
\item
  standardized the spreadsheet column titles by hand to allow import and
  correct dataframe referencing
\item
  file:
  \href{file:///home/gus/Dropbox/common/ipy_notebooks/YALE/g_f_fuscipes_general/2015-02-16_g_f_fuscipes_pandas_import.ipynb}{2015-02-16\_g\_f\_fuscipes\_pandas\_import.ipynb}
\item
  \texttt{recode\_villages(df, map\_func=map\_func)}:

  \begin{itemize}
  \itemsep1pt\parskip0pt\parsep0pt
  \item
    renaming villages to letter codes
  \item
    \textbf{{[}degenerate names discovered{]}} and accommodated
    in\\\href{file:///home/gus/Dropbox/uganda_data/data_repos/field_data/locations/names/uganda_village_id_map.csv}{uganda\_village\_id\_map.csv}
    by mapping the letter code to more than one long form:

    \begin{itemize}
    \itemsep1pt\parskip0pt\parsep0pt
    \item
      AKAYODEBE vs AKAYO-DEBE
    \end{itemize}
  \item
    corrected misspellings of

    \begin{itemize}
    \itemsep1pt\parskip0pt\parsep0pt
    \item
      ``Orubakulemi'' from ``Orubakulem''
    \item
      ``JIAKO'' from ``JAIKO''
    \end{itemize}
  \end{itemize}
\item
  \texttt{recode\_positives(df)}:

  \begin{itemize}
  \itemsep1pt\parskip0pt\parsep0pt
  \item
    recode \texttt{prob}, \texttt{midgut}, \texttt{sal.gland} as
    \texttt{0} or \texttt{1}.
  \item
    \textbf{{[}NOTE{]}} this will change to a trivalent state (class
    \texttt{Tristate}) soon
  \end{itemize}
\item
  \texttt{recode\_tenerals(df)}

  \begin{itemize}
  \itemsep1pt\parskip0pt\parsep0pt
  \item
    implemented but needs conversion to \texttt{Tristate}
  \end{itemize}
\item
  \texttt{recode\_dead(df)}

  \begin{itemize}
  \itemsep1pt\parskip0pt\parsep0pt
  \item
    implemented but needs conversion to \texttt{Tristate}
  \end{itemize}
\item
  \texttt{add\_infection\_state\_col(df)}

  \begin{itemize}
  \itemsep1pt\parskip0pt\parsep0pt
  \item
    implemented but failing to actually alter the dataframe
  \end{itemize}
\item
  \texttt{spartan.utils.misc.Tristate}

  \begin{itemize}
  \itemsep1pt\parskip0pt\parsep0pt
  \item
    implements three state logic that \emph{mostly} supports normal
    boolean arithmetic (just ignoring the \texttt{None} state)
  \end{itemize}
\end{itemize}

\section{2015-02-23 (Monday)}\label{monday-3}

\textbf{GOALS:}

\begin{itemize}
\itemsep1pt\parskip0pt\parsep0pt
\item
  \texttt{{[} {]}} \emph{G. f. fuscipes} infection summaries/maps for
  GisellaC meeting
\item
  \texttt{{[} {]}} script for MariangelaB
\item
  \texttt{{[} {]}} \(r^2\) per bin model
\end{itemize}

\subsection{\emph{G. f. fuscipes}: infection
summaries}\label{g.-f.-fuscipes-infection-summaries-3}

\subsubsection{HDF5 import and data
cleaning}\label{hdf5-import-and-data-cleaning-1}

\begin{itemize}
\itemsep1pt\parskip0pt\parsep0pt
\item
  \texttt{spartan.utils.misc.Tristate}

  \begin{itemize}
  \itemsep1pt\parskip0pt\parsep0pt
  \item
    I found an existing ``Tribool'' class on github and forked
    it:\\\url{https://github.com/xguse/python_tribool}
  \item
    it did not support boolean arithmetic but was much more
    sophisticated in all other ways.
  \item
    I added support for boolean addition but will also add *, -, and /
    before writing the tests and submitting a pull request to upstream.
  \item
    I am now using \texttt{Tribool} instead of \texttt{Tristate}
  \end{itemize}
\item
  running into serious hashable issues \texttt{df.midgut.unique()}
  throws \texttt{\_\_nonezero\_\_}'s \texttt{ValueError}.

  \begin{itemize}
  \itemsep1pt\parskip0pt\parsep0pt
  \item
    possible solutions:

    \begin{itemize}
    \itemsep1pt\parskip0pt\parsep0pt
    \item
      override \texttt{\_\_new\_\_} might allow me to mimic the ``always
      the same mem address'' behavior of \texttt{True} etc?
    \item
      Look into implementation of Factories in Python
    \item
      perhaps a hint in behavior/class code for \texttt{np.NaN}?
    \item
      \textbf{{[}best bet{]}} use \texttt{enum} class
    \end{itemize}
  \end{itemize}
\item
  looking for more fertile ground to cover while I think
\end{itemize}

\end{document}
