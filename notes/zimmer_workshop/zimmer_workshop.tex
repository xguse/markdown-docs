\documentclass[letterpaper]{scrartcl}
\usepackage{lmodern}
\usepackage{amssymb,amsmath}
\usepackage{ifxetex,ifluatex}
\usepackage{fixltx2e} % provides \textsubscript
\ifnum 0\ifxetex 1\fi\ifluatex 1\fi=0 % if pdftex
  \usepackage[T1]{fontenc}
  \usepackage[utf8]{inputenc}
\else % if luatex or xelatex
  \ifxetex
    \usepackage{mathspec}
    \usepackage{xltxtra,xunicode}
  \else
    \usepackage{fontspec}
  \fi
  \defaultfontfeatures{Mapping=tex-text,Scale=MatchLowercase}
  \newcommand{\euro}{€}
\fi
% use upquote if available, for straight quotes in verbatim environments
\IfFileExists{upquote.sty}{\usepackage{upquote}}{}
% use microtype if available
\IfFileExists{microtype.sty}{%
\usepackage{microtype}
\UseMicrotypeSet[protrusion]{basicmath} % disable protrusion for tt fonts
}{}
\usepackage{graphicx}
\makeatletter
\def\maxwidth{\ifdim\Gin@nat@width>\linewidth\linewidth\else\Gin@nat@width\fi}
\def\maxheight{\ifdim\Gin@nat@height>\textheight\textheight\else\Gin@nat@height\fi}
\makeatother
% Scale images if necessary, so that they will not overflow the page
% margins by default, and it is still possible to overwrite the defaults
% using explicit options in \includegraphics[width, height, ...]{}
\setkeys{Gin}{width=\maxwidth,height=\maxheight,keepaspectratio}
\ifxetex
  \usepackage[setpagesize=false, % page size defined by xetex
              unicode=false, % unicode breaks when used with xetex
              xetex]{hyperref}
\else
  \usepackage[unicode=true]{hyperref}
\fi
\hypersetup{breaklinks=true,
            bookmarks=true,
            pdfauthor={Gus Dunn},
            pdftitle={Zimmer},
            colorlinks=true,
            citecolor=blue,
            urlcolor=blue,
            linkcolor=magenta,
            pdfborder={0 0 0}}
\urlstyle{same}  % don't use monospace font for urls
\setlength{\parindent}{0pt}
\setlength{\parskip}{6pt plus 2pt minus 1pt}
\setlength{\emergencystretch}{3em}  % prevent overfull lines
\setcounter{secnumdepth}{5}

\title{Zimmer}
\author{Gus Dunn}
\date{}
\usepackage[T1]{fontenc}
\usepackage{lxfonts}

% blockquote


\begin{document}
\maketitle

{
\hypersetup{linkcolor=black}
\setcounter{tocdepth}{3}
\tableofcontents
}
\section{Carl Zimmer Workshop}\label{carl-zimmer-workshop}

\subsection{Audio and slides were
recorded}\label{audio-and-slides-were-recorded}

\begin{itemize}
\itemsep1pt\parskip0pt\parsep0pt
\item
  link to dropbox should be included in email later from Carl
\end{itemize}

\subsection{outline}\label{outline}

\begin{itemize}
\itemsep1pt\parskip0pt\parsep0pt
\item
  assignment
\item
  becoming public scientist
\item
  when reporter calls
\item
  how to break into the biz
\item
  what IS the biz
\item
  self published scientist
\item
  more
\end{itemize}

\subsection{reflexions on assignments}\label{reflexions-on-assignments}

\begin{itemize}
\itemsep1pt\parskip0pt\parsep0pt
\item
  \textbf{frank:} be willing to ``kill your darlings''

  \begin{itemize}
  \itemsep1pt\parskip0pt\parsep0pt
  \item
    drop things that arent working even if you LOVE it
  \end{itemize}
\item
  \textbf{other:} felt she repeated her self bc of the structure (intro
  up front stuff)

  \begin{itemize}
  \itemsep1pt\parskip0pt\parsep0pt
  \item
    dont give away too many details in the intro
  \item
    hard to find what will grab the reader but not \textbf{NEED} context
    to understand (so you dont have to get into the meat \textbf{twice})
  \item
    tap into reader's existing curiousity/intrest \textbf{not} existing
    expertise (cause that == \textbf{ZERO})
  \end{itemize}
\item
  make sure you are not writing more than one story at once
\item
  what happens when the paper you choose has like 6 cool results and
  maybe a great method too

  \begin{itemize}
  \itemsep1pt\parskip0pt\parsep0pt
  \item
    consider your assignment length
  \item
    whats the \textbf{main} thing you think is interesting or the
    audience might
  \item
    include caveats when needed that show there is more to the paper
    than what you are covering
  \item
    its ok to leave somethings unexhausted
  \end{itemize}
\end{itemize}

\subsection{Life as a public
scientist}\label{life-as-a-public-scientist}

\begin{itemize}
\itemsep1pt\parskip0pt\parsep0pt
\item
  Brian Greene as example
\item
  remmeber that this was the case back in the day
\end{itemize}

\subsection{When reporters show up}\label{when-reporters-show-up}

\begin{itemize}
\itemsep1pt\parskip0pt\parsep0pt
\item
  need to think about what you are going to say before the phone rings
\item
  might feel like minimizing all contact (PhD comics example of sci-news
  cycle)
\item
  problem is that this cycle will happen without you and you dont get to
  influence it if you dont participate
\item
  most important steps are at the beginning

  \begin{itemize}
  \itemsep1pt\parskip0pt\parsep0pt
  \item
    PRESS release (be obnoxious if its not something you can stand
    behind)

    \begin{itemize}
    \itemsep1pt\parskip0pt\parsep0pt
    \item
      use the fact that its your name on the line
    \end{itemize}
  \item
    reporters

    \begin{itemize}
    \itemsep1pt\parskip0pt\parsep0pt
    \item
      be ready with what you want the reporter to leave with
    \item
      think about that 600 word version of your paper and base the convo
      on that
    \end{itemize}
  \end{itemize}
\end{itemize}

\subsection{breaking in:
institutionally}\label{breaking-in-institutionally}

\begin{itemize}
\itemsep1pt\parskip0pt\parsep0pt
\item
  graduate programs
\item
  interships
\item
  AAAS mass media fellows
  \href{aaas.org/programs/education/MassMedia}{AAAS program}
\item
  entry level staf positions

  \begin{itemize}
  \itemsep1pt\parskip0pt\parsep0pt
  \item
    rarer and rarer
  \end{itemize}
\end{itemize}

\subsubsection{Rosie Mestel}\label{rosie-mestel}

\begin{itemize}
\itemsep1pt\parskip0pt\parsep0pt
\item
  BS bio
\item
  PhD UC Davis
\item
  UC Santa Cruz: masters
\item
  Discover assistant editor
\end{itemize}

\subsubsection{Virginia Hughes}\label{virginia-hughes}

\begin{itemize}
\itemsep1pt\parskip0pt\parsep0pt
\item
  BuzzFeed Science
\item
  ex neuroscience Student
\item
  science editor, \emph{BuzzFeed}
\item
  hireing eight staff writers
\item
  150 million views a month
\item
  tease them about lists/cats/etc

  \begin{itemize}
  \itemsep1pt\parskip0pt\parsep0pt
  \item
    trash pays for other stuff
  \end{itemize}
\end{itemize}

\subsection{State of (Science) Classic Journalism
Sources}\label{state-of-science-classic-journalism-sources}

\begin{itemize}
\item
  its how you thought

  \begin{itemize}
  \itemsep1pt\parskip0pt\parsep0pt
  \item
    bad
  \end{itemize}
\item
\end{itemize}

\subsection{BuzFeed types}\label{buzfeed-types}

\begin{itemize}
\itemsep1pt\parskip0pt\parsep0pt
\item
  also Vox
\end{itemize}

\subsection{Beyond the written word}\label{beyond-the-written-word}

\begin{itemize}
\itemsep1pt\parskip0pt\parsep0pt
\item
  RadioLab

  \begin{itemize}
  \itemsep1pt\parskip0pt\parsep0pt
  \item
    focuses on stories that have good sound potential (obvi)
  \item
    podcast-based popularity
  \end{itemize}
\end{itemize}

\subsection{The self-published
Scientist}\label{the-self-published-scientist}

\begin{itemize}
\itemsep1pt\parskip0pt\parsep0pt
\item
  where do you get information on specific scientific issues slide:

  \begin{itemize}
  \itemsep1pt\parskip0pt\parsep0pt
  \item
    christie wilcox
  \item
    internet is main source of scientific information on specific issues
  \end{itemize}
\item
  A mesh not a pipeline:

  \begin{itemize}
  \itemsep1pt\parskip0pt\parsep0pt
  \item
    comments
  \item
    own blog posts

    \begin{itemize}
    \itemsep1pt\parskip0pt\parsep0pt
    \item
      youre wrong
    \item
      that makes me thing of \textbf{this}
    \item
      etc
    \end{itemize}
  \item
    facebook, twitter, g+ (ok maybe not that last one)
  \end{itemize}
\item
  \textbf{CONSISTENCY IS PARAMOUNT}

  \begin{itemize}
  \itemsep1pt\parskip0pt\parsep0pt
  \item
    you are building a community that goes away if you are silent
  \end{itemize}
\end{itemize}

\subsubsection{Tara Smith PhD}\label{tara-smith-phd}

\begin{itemize}
\itemsep1pt\parskip0pt\parsep0pt
\item
  Aetiology (science blogs network)
\item
  both science content and editorializing stuff in the news
\item
  measles etc
\end{itemize}

\subsubsection{Vincent Racaniello}\label{vincent-racaniello}

\begin{itemize}
\itemsep1pt\parskip0pt\parsep0pt
\item
  columbia uni
\item
  blog
\item
  podcasts
\item
  uses this as a platform for his views
\end{itemize}

\subsubsection{Henry Reich}\label{henry-reich}

\begin{itemize}
\itemsep1pt\parskip0pt\parsep0pt
\item
  Minute Physics
\item
  now his actual \textbf{JOB}
\end{itemize}

\subsubsection{CreatureCast}\label{creaturecast}

\begin{itemize}
\itemsep1pt\parskip0pt\parsep0pt
\item
  \href{http://creaturecast.org}{CreatureCast}
\item
  from zoology class projects \textbf{now part of the New York Times}
  (sorta)
\item
  short animated videos
\end{itemize}

\subsection{Resources}\label{resources}

\begin{itemize}
\itemsep1pt\parskip0pt\parsep0pt
\item
  Pinker's \emph{``The Sense of Style''}
\item
  Cornelia Dean \emph{``Am I making Myself Clear''}

  \begin{itemize}
  \itemsep1pt\parskip0pt\parsep0pt
  \item
    how to talk to journalists/public
  \end{itemize}
\item
  Randy Olson \emph{``Don't be Such a Scientist''}

  \begin{itemize}
  \itemsep1pt\parskip0pt\parsep0pt
  \item
    former marine biologist, now filmmaker
  \item
    how not to make may errors that scientists tent to when talking to
    \emph{normal} people
  \end{itemize}
\end{itemize}

\subsection{Embrace Your Inner
Chimera}\label{embrace-your-inner-chimera}

\subsection{Questions}\label{questions}

\subsubsection{about long vs short form}\label{about-long-vs-short-form}

\begin{itemize}
\itemsep1pt\parskip0pt\parsep0pt
\item
  Carl does not think that the shortification is a long term issue
\item
  the Atlantic
\item
  the Adivist(?)

  \begin{itemize}
  \itemsep1pt\parskip0pt\parsep0pt
  \item
    20K words
  \item
    too long for mag
  \item
    too short for book
  \end{itemize}
\end{itemize}

\subsubsection{Places tio write sometimes but not your own second job
like
Blog}\label{places-tio-write-sometimes-but-not-your-own-second-job-like-blog}

\begin{itemize}
\itemsep1pt\parskip0pt\parsep0pt
\item
  op/ed
\item
  but in order to get published even by these people you need to
  practice this skill, maybe a lot
\item
  so why not blog it?
\end{itemize}

\subsubsection{Advice for writing Journal articles that Carl would like
to
read}\label{advice-for-writing-journal-articles-that-carl-would-like-to-read}

\begin{itemize}
\itemsep1pt\parskip0pt\parsep0pt
\item
  \textbf{important:} lay out problem in the beginning in very clear way

  \begin{itemize}
  \itemsep1pt\parskip0pt\parsep0pt
  \item
    abandon the right amount of expertise
  \item
    remember that \emph{hopefully} journalist may also be reading
  \end{itemize}
\item
  you can sort of diagnose problems likely to be in the rest pf the
  paper by the quality of the introduction
\item
  discussion:

  \begin{itemize}
  \itemsep1pt\parskip0pt\parsep0pt
  \item
    reader should understand \textbf{why} one paragraph follows the last
  \item
    even -- \textbf{especially} -- a non-specialist should be able to
  \end{itemize}
\end{itemize}

\end{document}
