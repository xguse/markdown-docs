\documentclass[letterpaper]{scrartcl}
\usepackage{lmodern}
\usepackage{amssymb,amsmath}
\usepackage{ifxetex,ifluatex}
\usepackage{fixltx2e} % provides \textsubscript
\ifnum 0\ifxetex 1\fi\ifluatex 1\fi=0 % if pdftex
  \usepackage[T1]{fontenc}
  \usepackage[utf8]{inputenc}
\else % if luatex or xelatex
  \ifxetex
    \usepackage{mathspec}
    \usepackage{xltxtra,xunicode}
  \else
    \usepackage{fontspec}
  \fi
  \defaultfontfeatures{Mapping=tex-text,Scale=MatchLowercase}
  \newcommand{\euro}{€}
\fi
% use upquote if available, for straight quotes in verbatim environments
\IfFileExists{upquote.sty}{\usepackage{upquote}}{}
% use microtype if available
\IfFileExists{microtype.sty}{%
\usepackage{microtype}
\UseMicrotypeSet[protrusion]{basicmath} % disable protrusion for tt fonts
}{}
\usepackage{graphicx}
\makeatletter
\def\maxwidth{\ifdim\Gin@nat@width>\linewidth\linewidth\else\Gin@nat@width\fi}
\def\maxheight{\ifdim\Gin@nat@height>\textheight\textheight\else\Gin@nat@height\fi}
\makeatother
% Scale images if necessary, so that they will not overflow the page
% margins by default, and it is still possible to overwrite the defaults
% using explicit options in \includegraphics[width, height, ...]{}
\setkeys{Gin}{width=\maxwidth,height=\maxheight,keepaspectratio}
\ifxetex
  \usepackage[setpagesize=false, % page size defined by xetex
              unicode=false, % unicode breaks when used with xetex
              xetex]{hyperref}
\else
  \usepackage[unicode=true]{hyperref}
\fi
\hypersetup{breaklinks=true,
            bookmarks=true,
            pdfauthor={Gus Dunn},
            pdftitle={General Brainstorming},
            colorlinks=true,
            citecolor=blue,
            urlcolor=blue,
            linkcolor=magenta,
            pdfborder={0 0 0}}
\urlstyle{same}  % don't use monospace font for urls
\setlength{\parindent}{0pt}
\setlength{\parskip}{6pt plus 2pt minus 1pt}
\setlength{\emergencystretch}{3em}  % prevent overfull lines
\setcounter{secnumdepth}{5}

\title{General Brainstorming\\\vspace{0.5em}{\large Caccone PostDoc}}
\author{Gus Dunn}
\date{2015: Spring}
\usepackage{bbding}
\usepackage[T1]{fontenc}
\usepackage{lxfonts}

\begin{document}
\maketitle

{
\hypersetup{linkcolor=black}
\setcounter{tocdepth}{3}
\tableofcontents
}
\begin{center}\rule{0.5\linewidth}{\linethickness}\end{center}

\section{New Project Storming}\label{new-project-storming}

\textbf{2015-01-09 (Friday)}

\subsection{Topics}\label{topics}

\begin{itemize}
\itemsep1pt\parskip0pt\parsep0pt
\item
  What are control efforts doing to the local populations of \emph{G. f.
  fuscipes}?
\end{itemize}

\subsection{White board}\label{white-board}

\begin{itemize}
\itemsep1pt\parskip0pt\parsep0pt
\item
  is there a climate change angle that we can probe with this?
\item
  proximity to ``urban'' areas or high traffic areas (major
  roads/parks)?
\item
  what population structure model is best for our populations?

  \begin{itemize}
  \itemsep1pt\parskip0pt\parsep0pt
  \item
    island
  \item
    hierarchical
  \item
    \emph{etc}
  \end{itemize}
\end{itemize}

\end{document}
