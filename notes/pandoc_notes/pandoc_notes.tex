\documentclass[letterpaper]{scrartcl}
\usepackage{concmath}
\usepackage{amssymb,amsmath}
\usepackage{ifxetex,ifluatex}
\usepackage{fixltx2e} % provides \textsubscript
\ifnum 0\ifxetex 1\fi\ifluatex 1\fi=0 % if pdftex
  \usepackage[T1]{fontenc}
  \usepackage[utf8]{inputenc}
\else % if luatex or xelatex
  \ifxetex
    \usepackage{mathspec}
    \usepackage{xltxtra,xunicode}
  \else
    \usepackage{fontspec}
  \fi
  \defaultfontfeatures{Mapping=tex-text,Scale=MatchLowercase}
  \newcommand{\euro}{€}
\fi
% use upquote if available, for straight quotes in verbatim environments
\IfFileExists{upquote.sty}{\usepackage{upquote}}{}
% use microtype if available
\IfFileExists{microtype.sty}{%
\usepackage{microtype}
\UseMicrotypeSet[protrusion]{basicmath} % disable protrusion for tt fonts
}{}
\usepackage{color}
\usepackage{fancyvrb}
\newcommand{\VerbBar}{|}
\newcommand{\VERB}{\Verb[commandchars=\\\{\}]}
\DefineVerbatimEnvironment{Highlighting}{Verbatim}{commandchars=\\\{\}}
% Add ',fontsize=\small' for more characters per line
\usepackage{framed}
\definecolor{shadecolor}{RGB}{48,48,48}
\newenvironment{Shaded}{\begin{snugshade}}{\end{snugshade}}
\newcommand{\KeywordTok}[1]{\textcolor[rgb]{0.94,0.87,0.69}{{#1}}}
\newcommand{\DataTypeTok}[1]{\textcolor[rgb]{0.87,0.87,0.75}{{#1}}}
\newcommand{\DecValTok}[1]{\textcolor[rgb]{0.86,0.86,0.80}{{#1}}}
\newcommand{\BaseNTok}[1]{\textcolor[rgb]{0.86,0.64,0.64}{{#1}}}
\newcommand{\FloatTok}[1]{\textcolor[rgb]{0.75,0.75,0.82}{{#1}}}
\newcommand{\CharTok}[1]{\textcolor[rgb]{0.86,0.64,0.64}{{#1}}}
\newcommand{\StringTok}[1]{\textcolor[rgb]{0.80,0.58,0.58}{{#1}}}
\newcommand{\CommentTok}[1]{\textcolor[rgb]{0.50,0.62,0.50}{{#1}}}
\newcommand{\OtherTok}[1]{\textcolor[rgb]{0.94,0.94,0.56}{{#1}}}
\newcommand{\AlertTok}[1]{\textcolor[rgb]{1.00,0.81,0.69}{{#1}}}
\newcommand{\FunctionTok}[1]{\textcolor[rgb]{0.94,0.94,0.56}{{#1}}}
\newcommand{\RegionMarkerTok}[1]{\textcolor[rgb]{0.80,0.80,0.80}{{#1}}}
\newcommand{\ErrorTok}[1]{\textcolor[rgb]{0.76,0.75,0.62}{{#1}}}
\newcommand{\NormalTok}[1]{\textcolor[rgb]{0.80,0.80,0.80}{{#1}}}
\usepackage{graphicx}
\makeatletter
\def\maxwidth{\ifdim\Gin@nat@width>\linewidth\linewidth\else\Gin@nat@width\fi}
\def\maxheight{\ifdim\Gin@nat@height>\textheight\textheight\else\Gin@nat@height\fi}
\makeatother
% Scale images if necessary, so that they will not overflow the page
% margins by default, and it is still possible to overwrite the defaults
% using explicit options in \includegraphics[width, height, ...]{}
\setkeys{Gin}{width=\maxwidth,height=\maxheight,keepaspectratio}
\ifxetex
  \usepackage[setpagesize=false, % page size defined by xetex
              unicode=false, % unicode breaks when used with xetex
              xetex]{hyperref}
\else
  \usepackage[unicode=true]{hyperref}
\fi
\hypersetup{breaklinks=true,
            bookmarks=true,
            pdfauthor={Gus Dunn},
            pdftitle={Pandoc Notes},
            colorlinks=true,
            citecolor=blue,
            urlcolor=blue,
            linkcolor=magenta,
            pdfborder={0 0 0}}
\urlstyle{same}  % don't use monospace font for urls
\setlength{\parindent}{0pt}
\setlength{\parskip}{6pt plus 2pt minus 1pt}
\setlength{\emergencystretch}{3em}  % prevent overfull lines
\setcounter{secnumdepth}{5}

\title{Pandoc Notes}
\author{Gus Dunn}
\date{}

\begin{document}
\maketitle

{
\hypersetup{linkcolor=black}
\setcounter{tocdepth}{3}
\tableofcontents
}
\section{Code syntax highlighting}\label{code-syntax-highlighting}

\subsection{Basic example from
\href{http://johnmacfarlane.net}{johnmacfarlane.net}}\label{basic-example-from-johnmacfarlane.net}

Source:
\href{http://johnmacfarlane.net/pandoc/demo/example18f.html}{johnmacfarlane.net/pandoc/demo/example18f.html}

\begin{quote}
Here's what a delimited code block looks like:

\begin{Shaded}
\begin{Highlighting}[]
\CommentTok{-- | Inefficient quicksort in haskell.}
\OtherTok{qsort ::} \NormalTok{(}\DataTypeTok{Enum} \NormalTok{a) }\OtherTok{=>} \NormalTok{[a] }\OtherTok{->} \NormalTok{[a]}
\NormalTok{qsort []     }\FunctionTok{=} \NormalTok{[]}
\NormalTok{qsort (x}\FunctionTok{:}\NormalTok{xs) }\FunctionTok{=} \NormalTok{qsort (filter (}\FunctionTok{<} \NormalTok{x) xs) }\FunctionTok{++} \NormalTok{[x] }\FunctionTok{++}
              \NormalTok{qsort (filter (}\FunctionTok{>=} \NormalTok{x) xs) }
\end{Highlighting}
\end{Shaded}

Here's some python, with numbered lines (specify \{.python
.numberLines\}):

\begin{Shaded}
\begin{Highlighting}[numbers=left,,]
\KeywordTok{class} \NormalTok{FSM(}\DataTypeTok{object}\NormalTok{):}

\CommentTok{"""This is a Finite State Machine (FSM).}
\CommentTok{"""}

\KeywordTok{def} \OtherTok{__init__}\NormalTok{(}\OtherTok{self}\NormalTok{, initial_state, memory=}\OtherTok{None}\NormalTok{):}

   \CommentTok{"""This creates the FSM. You set the initial state here. The "memory"}
\CommentTok{   attribute is any object that you want to pass along to the action}
\CommentTok{   functions. It is not used by the FSM. For parsing you would typically}
\CommentTok{   pass a list to be used as a stack. """}

   \CommentTok{# Map (input_symbol, current_state) --> (action, next_state).}
   \OtherTok{self}\NormalTok{.state_transitions = \{\}}
   \CommentTok{# Map (current_state) --> (action, next_state).}
   \OtherTok{self}\NormalTok{.state_transitions_any = \{\}}
   \OtherTok{self}\NormalTok{.default_transition = }\OtherTok{None}
   \NormalTok{...}
\end{Highlighting}
\end{Shaded}
\end{quote}

\subsection{Highlighting languages/styles
available}\label{highlighting-languagesstyles-available}

\begin{itemize}
\itemsep1pt\parskip0pt\parsep0pt
\item
  based on excerpts from \texttt{Highlighting.hs} on MacFarlane's github
  repo:
  \href{https://github.com/jgm/pandoc-highlight/blob/master/Text/Pandoc/Highlighting.hs}{\texttt{jgm/pandoc-highlight}}
\end{itemize}

\subsubsection{Languages}\label{languages}

\begin{Shaded}
\begin{Highlighting}[]

\FunctionTok{...}

\OtherTok{langsList ::} \NormalTok{[(}\DataTypeTok{String}\NormalTok{, }\DataTypeTok{String}\NormalTok{)]}
\NormalTok{langsList }\FunctionTok{=}    \NormalTok{[(}\StringTok{"ada"}\NormalTok{,}\StringTok{"Ada"}\NormalTok{)}
               \NormalTok{,(}\StringTok{"java"}\NormalTok{,}\StringTok{"Java"}\NormalTok{)}
               \NormalTok{,(}\StringTok{"prolog"}\NormalTok{,}\StringTok{"Prolog"}\NormalTok{)}
               \NormalTok{,(}\StringTok{"python"}\NormalTok{,}\StringTok{"Python"}\NormalTok{)}
               \NormalTok{,(}\StringTok{"gnuassembler"}\NormalTok{,}\StringTok{"Assembler"}\NormalTok{)}
               \NormalTok{,(}\StringTok{"commonlisp"}\NormalTok{,}\StringTok{"Lisp"}\NormalTok{)}
               \NormalTok{,(}\StringTok{"r"}\NormalTok{,}\StringTok{"R"}\NormalTok{)}
               \NormalTok{,(}\StringTok{"awk"}\NormalTok{,}\StringTok{"Awk"}\NormalTok{)}
               \NormalTok{,(}\StringTok{"bash"}\NormalTok{,}\StringTok{"bash"}\NormalTok{)}
               \NormalTok{,(}\StringTok{"makefile"}\NormalTok{,}\StringTok{"make"}\NormalTok{)}
               \NormalTok{,(}\StringTok{"c"}\NormalTok{,}\StringTok{"C"}\NormalTok{)}
               \NormalTok{,(}\StringTok{"matlab"}\NormalTok{,}\StringTok{"Matlab"}\NormalTok{)}
               \NormalTok{,(}\StringTok{"ruby"}\NormalTok{,}\StringTok{"Ruby"}\NormalTok{)}
               \NormalTok{,(}\StringTok{"cpp"}\NormalTok{,}\StringTok{"C++"}\NormalTok{)}
               \NormalTok{,(}\StringTok{"ocaml"}\NormalTok{,}\StringTok{"Caml"}\NormalTok{)}
               \NormalTok{,(}\StringTok{"modula2"}\NormalTok{,}\StringTok{"Modula-2"}\NormalTok{)}
               \NormalTok{,(}\StringTok{"sql"}\NormalTok{,}\StringTok{"SQL"}\NormalTok{)}
               \NormalTok{,(}\StringTok{"eiffel"}\NormalTok{,}\StringTok{"Eiffel"}\NormalTok{)}
               \NormalTok{,(}\StringTok{"tcl"}\NormalTok{,}\StringTok{"tcl"}\NormalTok{)}
               \NormalTok{,(}\StringTok{"erlang"}\NormalTok{,}\StringTok{"erlang"}\NormalTok{)}
               \NormalTok{,(}\StringTok{"verilog"}\NormalTok{,}\StringTok{"Verilog"}\NormalTok{)}
               \NormalTok{,(}\StringTok{"fortran"}\NormalTok{,}\StringTok{"Fortran"}\NormalTok{)}
               \NormalTok{,(}\StringTok{"vhdl"}\NormalTok{,}\StringTok{"VHDL"}\NormalTok{)}
               \NormalTok{,(}\StringTok{"pascal"}\NormalTok{,}\StringTok{"Pascal"}\NormalTok{)}
               \NormalTok{,(}\StringTok{"perl"}\NormalTok{,}\StringTok{"Perl"}\NormalTok{)}
               \NormalTok{,(}\StringTok{"xml"}\NormalTok{,}\StringTok{"XML"}\NormalTok{)}
               \NormalTok{,(}\StringTok{"haskell"}\NormalTok{,}\StringTok{"Haskell"}\NormalTok{)}
               \NormalTok{,(}\StringTok{"php"}\NormalTok{,}\StringTok{"PHP"}\NormalTok{)}
               \NormalTok{,(}\StringTok{"xslt"}\NormalTok{,}\StringTok{"XSLT"}\NormalTok{)}
               \NormalTok{,(}\StringTok{"html"}\NormalTok{,}\StringTok{"HTML"}\NormalTok{)}
               \NormalTok{]}

\FunctionTok{...}
\end{Highlighting}
\end{Shaded}

\subsubsection{Styles}\label{styles}

Looks like only these?

\begin{itemize}
\itemsep1pt\parskip0pt\parsep0pt
\item
  pygments
\item
  espresso
\item
  zenburn
\item
  tango
\item
  kate
\item
  monochrom
\item
  haddock
\end{itemize}

\begin{Shaded}
\begin{Highlighting}[]

\KeywordTok{module} \DataTypeTok{Text.Pandoc.Highlighting} \NormalTok{( languages}
                                \NormalTok{, languagesByExtension}
                                \NormalTok{, highlight}
                                \NormalTok{, formatLaTeXInline}
                                \NormalTok{, formatLaTeXBlock}
                                \NormalTok{, styleToLaTeX}
                                \NormalTok{, formatHtmlInline}
                                \NormalTok{, formatHtmlBlock}
                                \NormalTok{, styleToCss}
                                \NormalTok{, pygments}
                                \NormalTok{, espresso}
                                \NormalTok{, zenburn}
                                \NormalTok{, tango}
                                \NormalTok{, kate}
                                \NormalTok{, monochrome}
                                \NormalTok{, haddock}
                                \NormalTok{, }\DataTypeTok{Style}
                                \NormalTok{, fromListingsLanguage}
                                \NormalTok{, toListingsLanguage}
                                \NormalTok{) }\KeywordTok{where}
\end{Highlighting}
\end{Shaded}

\subsubsection{\texttt{highlighting-kate}}\label{highlighting-kate}

Actually it looks like any highlighting style supported by
\href{http://kate-editor.org/}{Kate} (\emph{as long as you have it
installed properly?}) should be supported.

From
\href{https://benjeffrey.com/pandoc-syntax-highlighting-css}{benjeffrey.com}:

\begin{quote}
\textbf{The highlighting-kate Package}
\end{quote}

\begin{quote}
Unless you dig into Pandoc's Haddock documentation, you won't find much
information on the internet telling you what the generated markup
classes (.kw, .co, .ot, etc.) mean, or how you can customize them.
\end{quote}

\begin{quote}
It turns out that Pandoc relies on another one of John Macfarlane's
creations in order to mark up code syntax, the highlighting-kate
package,
\end{quote}

\begin{quote}
\begin{quote}
\emph{a syntax highlighting library with support for nearly one hundred
languages. The syntax parsers are automatically generated from Kate
syntax descriptions (http://kate-editor.org/), so any syntax supported
by Kate can be added.}
\end{quote}
\end{quote}

\begin{quote}
\begin{quote}
-highlighting-kate package description
\end{quote}
\end{quote}

\begin{quote}
So it turns out that the syntax tokens are based on definitions provided
by the KDE text editor, Kate.
\end{quote}

\subsubsection{\texttt{Pygments} styles for
highlighting}\label{pygments-styles-for-highlighting}

For this you will need to write a filter that sends every code block to
\texttt{pygments} and makes sure the result ends up in the document.

\subsection{Invoking the \texttt{highlighting-kate}
styles}\label{invoking-the-highlighting-kate-styles}

You add \texttt{-\/-highlight-style \textless{}style-name\textgreater{}}
to the command line or in your \texttt{Makefile}.

\end{document}
