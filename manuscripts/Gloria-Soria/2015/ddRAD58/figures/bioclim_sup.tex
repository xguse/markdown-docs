\documentclass[letterpaper]{scrartcl}
\usepackage{lmodern}
\usepackage{amssymb,amsmath}
\usepackage{ifxetex,ifluatex}
\usepackage{fixltx2e} % provides \textsubscript
\ifnum 0\ifxetex 1\fi\ifluatex 1\fi=0 % if pdftex
  \usepackage[T1]{fontenc}
  \usepackage[utf8]{inputenc}
\else % if luatex or xelatex
  \ifxetex
    \usepackage{mathspec}
    \usepackage{xltxtra,xunicode}
  \else
    \usepackage{fontspec}
  \fi
  \defaultfontfeatures{Mapping=tex-text,Scale=MatchLowercase}
  \newcommand{\euro}{€}
\fi
% use upquote if available, for straight quotes in verbatim environments
\IfFileExists{upquote.sty}{\usepackage{upquote}}{}
% use microtype if available
\IfFileExists{microtype.sty}{%
\usepackage{microtype}
\UseMicrotypeSet[protrusion]{basicmath} % disable protrusion for tt fonts
}{}
\usepackage[margin=1in]{geometry}
\usepackage{longtable,booktabs}
\usepackage{graphicx}
\makeatletter
\def\maxwidth{\ifdim\Gin@nat@width>\linewidth\linewidth\else\Gin@nat@width\fi}
\def\maxheight{\ifdim\Gin@nat@height>\textheight\textheight\else\Gin@nat@height\fi}
\makeatother
% Scale images if necessary, so that they will not overflow the page
% margins by default, and it is still possible to overwrite the defaults
% using explicit options in \includegraphics[width, height, ...]{}
\setkeys{Gin}{width=\maxwidth,height=\maxheight,keepaspectratio}
\ifxetex
  \usepackage[setpagesize=false, % page size defined by xetex
              unicode=false, % unicode breaks when used with xetex
              xetex]{hyperref}
\else
  \usepackage[unicode=true]{hyperref}
\fi
\hypersetup{breaklinks=true,
            bookmarks=true,
            pdfauthor={},
            pdftitle={},
            colorlinks=true,
            citecolor=blue,
            urlcolor=blue,
            linkcolor=magenta,
            pdfborder={0 0 0}}
\urlstyle{same}  % don't use monospace font for urls
\setlength{\parindent}{0pt}
\setlength{\parskip}{6pt plus 2pt minus 1pt}
\setlength{\emergencystretch}{3em}  % prevent overfull lines
\setcounter{secnumdepth}{5}

\date{}
\usepackage{fontspec}
\setmainfont{Linux Libertine O}

% blockquote


\begin{document}

{
\hypersetup{linkcolor=black}
\setcounter{tocdepth}{3}
\tableofcontents
}
\section{Table of bioclims used:}\label{table-of-bioclims-used}

\begin{longtable}[c]{@{}lllllllllllllll@{}}
\toprule
Site & bio12 & bio13 & bio14 & bio15 & bio18 & bio19 & bio2 & bio3 &
bio6 & bio7 & bio4 & bio5 & bio8 & bio9\tabularnewline
\midrule
\endhead
KG & 1918.0 & 302.0 & 79.0 & 45.0 & 436.0 & 306.0 & 103.0 & 83.0 & 153.0
& 124.0 & 525.0 & 277.0 & 219.0 & 208.0\tabularnewline
OT & 1312.0 & 194.0 & 17.0 & 51.0 & 134.0 & 447.0 & 129.0 & 79.0 & 165.0
& 163.0 & 1015.0 & 328.0 & 222.0 & 245.0\tabularnewline
MS & 1330.0 & 174.0 & 33.0 & 41.0 & 186.0 & 388.0 & 119.0 & 81.0 & 166.0
& 146.0 & 795.0 & 312.0 & 223.0 & 238.0\tabularnewline
NB & 1322.0 & 202.0 & 40.0 & 40.0 & 214.0 & 310.0 & 121.0 & 82.0 & 166.0
& 147.0 & 691.0 & 313.0 & 234.0 & 238.0\tabularnewline
\bottomrule
\end{longtable}

bio2 = Mean Diurnal Range (Mean of monthly (max temp - min temp))\\bio3
= Isothermality (BIO2/BIO7) (* 100)\\bio4 = Temperature Seasonality
(standard deviation *100)\\bio5 = Max Temperature of Warmest Month\\bio6
= Min Temperature of Coldest Month\\bio7 = Temperature Annual Range
(BIO5-BIO6)\\bio8 = Mean Temperature of Wettest Quarter\\bio9 = Mean
Temperature of Driest Quarter\\bio12 = Annual Precipitation\\bio13 =
Precipitation of Wettest Month\\bio14 = Precipitation of Driest
Month\\bio15 = Precipitation Seasonality (Coefficient of
Variation)\\bio18 = Precipitation of Warmest Quarter\\bio19 =
Precipitation of Coldest Quarter

Temperature values are in degrees C * 10.\\Precipitation values are in
mm.

\newpage

\section{Principal components of bioclim
data:}\label{principal-components-of-bioclim-data}

\begin{figure}[htbp]
\centering
\includegraphics{/home/gus/MEGAsync/projects/ddRAD_phase2/repos/ddRAD_phase2/notebook/static/media/gs_2015/bioclim_pca_1x3.png}
\caption{\textbf{Principal components one and three.} Principal
components one and three. Components one and three plotted against each
other illustrate that the MS site groups apart from the others along
component three's axis. Investigation of component three's loading
values reveals that the bioclims most responsible for this are bio8,
bio15. and bio13.}
\end{figure}

\end{document}
