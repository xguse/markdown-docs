\documentclass[letterpaper]{scrartcl}
\usepackage{fourier}
\usepackage{amssymb,amsmath}
\usepackage{ifxetex,ifluatex}
\usepackage{fixltx2e} % provides \textsubscript
\ifnum 0\ifxetex 1\fi\ifluatex 1\fi=0 % if pdftex
  \usepackage[T1]{fontenc}
  \usepackage[utf8]{inputenc}
\else % if luatex or xelatex
  \ifxetex
    \usepackage{mathspec}
    \usepackage{xltxtra,xunicode}
  \else
    \usepackage{fontspec}
  \fi
  \defaultfontfeatures{Mapping=tex-text,Scale=MatchLowercase}
  \newcommand{\euro}{€}
\fi
% use upquote if available, for straight quotes in verbatim environments
\IfFileExists{upquote.sty}{\usepackage{upquote}}{}
% use microtype if available
\IfFileExists{microtype.sty}{%
\usepackage{microtype}
\UseMicrotypeSet[protrusion]{basicmath} % disable protrusion for tt fonts
}{}
\usepackage[margin=1in]{geometry}
\usepackage{graphicx}
\makeatletter
\def\maxwidth{\ifdim\Gin@nat@width>\linewidth\linewidth\else\Gin@nat@width\fi}
\def\maxheight{\ifdim\Gin@nat@height>\textheight\textheight\else\Gin@nat@height\fi}
\makeatother
% Scale images if necessary, so that they will not overflow the page
% margins by default, and it is still possible to overwrite the defaults
% using explicit options in \includegraphics[width, height, ...]{}
\setkeys{Gin}{width=\maxwidth,height=\maxheight,keepaspectratio}
\ifxetex
  \usepackage[setpagesize=false, % page size defined by xetex
              unicode=false, % unicode breaks when used with xetex
              xetex]{hyperref}
\else
  \usepackage[unicode=true]{hyperref}
\fi
\hypersetup{breaklinks=true,
            bookmarks=true,
            pdfauthor={Gus Dunn},
            pdftitle={RE java and snpEff},
            colorlinks=true,
            citecolor=blue,
            urlcolor=blue,
            linkcolor=magenta,
            pdfborder={0 0 0}}
\urlstyle{same}  % don't use monospace font for urls
\setlength{\parindent}{0pt}
\setlength{\parskip}{6pt plus 2pt minus 1pt}
\setlength{\emergencystretch}{3em}  % prevent overfull lines
\setcounter{secnumdepth}{0}

\title{RE java and snpEff}
\author{Gus Dunn}
\date{2014-12-18}

\begin{document}
\maketitle

\section{2014-12-18}\label{section}

\subsection{First of all:}\label{first-of-all}

I am exceedingly proud of you! You totally rocked this email! You gave
me like 99\% of what I needed to think about this problem effectively
without me having to ask a single question! I hope that you notice your
improved computer literacy and allow yourself to feel some pride as
well. You deserve it.

\subsection{So on to the \texttt{snpEff}
stuff:}\label{so-on-to-the-snpeff-stuff}

\begin{quote}
when i run ``which java'' i get:
\end{quote}

\begin{verbatim}
mariangelabonizzoni@dhcp-v106-224 17:31:49 ~:
which java
/usr/bin/java
\end{verbatim}

\begin{quote}
when i call \texttt{java}, i get the help menu.
\end{quote}

\begin{quote}
When i call \texttt{java -version}, i get the java version as:
\end{quote}

\begin{verbatim}
mariangelabonizzoni@dhcp-v106-224 17:31:53 ~:
java -version
java version "1.8.0_25"
Java(TM) SE Runtime Environment (build 1.8.0_25-b17)
Java HotSpot(TM) 64-Bit Server VM (build 25.25-b02, mixed mode)
\end{verbatim}

\begin{quote}
When i call \texttt{which snpEff}, i get:
\end{quote}

\begin{verbatim}
which snpEff
/usr/local/bin/snpEff
\end{verbatim}

\begin{quote}
snpEff is supposely a java program to be run through
\texttt{java -jar snpEff.jar}, but when i do it, i always get the
message:
\end{quote}

\begin{verbatim}
mariangelabonizzoni@dhcp-v106-224 17:33:18 ~:
java -jar snpEff.jar
Error: Unable to access jarfile snpEff.jar
\end{verbatim}

\begin{quote}
I tried to run this command with many different variants, like going
into /user/local/bin or from /user/bin or using instead of sniper.jar
the whole path to the program, but i always get the same message.

interestingly, when i simply type \texttt{snpEff}, i get the help menu.
As follows
\end{quote}

\begin{verbatim}
snpEff
Error: Missing command

snpEff version SnpEff 3.6c (build 2014-05-20), by Pablo Cingolani
Usage: snpEff [command] [options] [files]

Run 'java -jar snpEff.jar command' for help on each specific command

Available commands:
        [eff]                        : Calculate effect of variants. Default: eff
(no command or 'eff').
        build                        : Build a SnpEff database.
        XXXXXXXXXXX
        -ud , -upDownStreamLen <int> : Set upstream downstream interval length
(in bases)
\end{verbatim}

\subsubsection{My reply:}\label{my-reply}

Yes.

I am fairly certain that the correct way to call snpEff is to omit the
\texttt{java -jar} call and simply call \texttt{snpEff} directly. The
reason for this is that it has been installed in \texttt{/usr/local/bin}
which is almost always in the user's \texttt{\$PATH} variable by
default. The way that it is installed the \texttt{java -jar} portion is
implied and the computer takes care of it behind the scenes.

The help text is an unfortunate confusion: for \textbf{for sure}. But
again, I am quite proud that you basically figured this out without my
help at all. Very nice work, Dr!

\subsection{The genome data files:}\label{the-genome-data-files}

\begin{quote}
the big problem that i have is that i have to build a genomic database
because the organism that i want to study is not supported (it is an
asian mosquito). i downloaded the required genome sequences and
annotation file in GTF from vectorbase and i followed (i thik i did) the
instruction in the ``create database'' from the snpEff menu page. but i
keep getting the same message as: genome not found. see below. i tried i
do not how amny different options to change the name of the files, to
run the comand from different folders, to change the configuration file
for snapped, i alwasy get the same message. like the genome file is not
seen. it is driving me creasy!!!! any idea? I wrote to the snpEff
developers, but thye have not had the curtesy to reply yet. any
suggestion would be great!!! thanks
\end{quote}

\begin{verbatim}
snpEff build -gtf22 /Volumes/Seagate_Exp_1/snpEff/data/AsinC2.1
java.lang.RuntimeException: Property:
'/Volumes/Seagate_Exp_1/snpEff/data/AsinC2.1.genome' not found
        at ca.mcgill.mcb.pcingola.interval.Genome.<init>(Genome.java:92)
        at ca.mcgill.mcb.pcingola.snpEffect.Config.readGenomeConfig(Config.java:513)
        at ca.mcgill.mcb.pcingola.snpEffect.Config.readConfig(Config.java:476)
        at ca.mcgill.mcb.pcingola.snpEffect.Config.init(Config.java:377)
        at ca.mcgill.mcb.pcingola.snpEffect.Config.<init>(Config.java:99)
        at
ca.mcgill.mcb.pcingola.snpEffect.commandLine.SnpEff.loadConfig(SnpEff.java:236)
        at
ca.mcgill.mcb.pcingola.snpEffect.commandLine.SnpEffCmdBuild.run(SnpEffCmdBuild.java:256)
        at ca.mcgill.mcb.pcingola.snpEffect.commandLine.SnpEff.run(SnpEff.java:685)
        at ca.mcgill.mcb.pcingola.snpEffect.commandLine.SnpEff.main(SnpEff.java:118)
\end{verbatim}

\subsubsection{My reply:}\label{my-reply-1}

Ok. This is a bit more complicated. I am going to have to try looking
through the \texttt{snpEff.config} file that they mention. It looks like
this file is how you set up all the locations of things that you will
need to organize to let \texttt{snpEff} know how to build its databases.
Did you configure this file yet? If so, would you send me the file so I
have a place to start?

Gus

\end{document}
