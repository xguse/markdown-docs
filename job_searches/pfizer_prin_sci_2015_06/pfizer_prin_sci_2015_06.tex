\documentclass[letterpaper]{scrartcl}
\usepackage{lmodern}
\usepackage{amssymb,amsmath}
\usepackage{ifxetex,ifluatex}
\usepackage{fixltx2e} % provides \textsubscript
\ifnum 0\ifxetex 1\fi\ifluatex 1\fi=0 % if pdftex
  \usepackage[T1]{fontenc}
  \usepackage[utf8]{inputenc}
\else % if luatex or xelatex
  \ifxetex
    \usepackage{mathspec}
    \usepackage{xltxtra,xunicode}
  \else
    \usepackage{fontspec}
  \fi
  \defaultfontfeatures{Mapping=tex-text,Scale=MatchLowercase}
  \newcommand{\euro}{€}
\fi
% use upquote if available, for straight quotes in verbatim environments
\IfFileExists{upquote.sty}{\usepackage{upquote}}{}
% use microtype if available
\IfFileExists{microtype.sty}{%
\usepackage{microtype}
\UseMicrotypeSet[protrusion]{basicmath} % disable protrusion for tt fonts
}{}
\usepackage[margin=1in]{geometry}
\usepackage{graphicx}
\makeatletter
\def\maxwidth{\ifdim\Gin@nat@width>\linewidth\linewidth\else\Gin@nat@width\fi}
\def\maxheight{\ifdim\Gin@nat@height>\textheight\textheight\else\Gin@nat@height\fi}
\makeatother
% Scale images if necessary, so that they will not overflow the page
% margins by default, and it is still possible to overwrite the defaults
% using explicit options in \includegraphics[width, height, ...]{}
\setkeys{Gin}{width=\maxwidth,height=\maxheight,keepaspectratio}
\ifxetex
  \usepackage[setpagesize=false, % page size defined by xetex
              unicode=false, % unicode breaks when used with xetex
              xetex]{hyperref}
\else
  \usepackage[unicode=true]{hyperref}
\fi
\hypersetup{breaklinks=true,
            bookmarks=true,
            pdfauthor={Gus Dunn},
            pdftitle={Master Skills List},
            colorlinks=true,
            citecolor=blue,
            urlcolor=blue,
            linkcolor=magenta,
            pdfborder={0 0 0}}
\urlstyle{same}  % don't use monospace font for urls
\setlength{\parindent}{0pt}
\setlength{\parskip}{6pt plus 2pt minus 1pt}
\setlength{\emergencystretch}{3em}  % prevent overfull lines
\setcounter{secnumdepth}{5}

\title{Master Skills List}
\author{Gus Dunn}
\date{2015-04-07}
\usepackage{fontspec}
\setmainfont{Linux Libertine O}

% blockquote


\begin{document}
\maketitle

{
\hypersetup{linkcolor=black}
\setcounter{tocdepth}{3}
\tableofcontents
}
\section{Skills}\label{skills}

\subsection{Hard}\label{hard}

\subsection{Soft}\label{soft}

\section{Listing Text}\label{listing-text}

\subsection{Org Marketing Statement}\label{org-marketing-statement}

All over the world, Pfizer colleagues are working together to positively
impact health for everyone, everywhere. Each position at Pfizer touches
and contributes to the success of our business and our world. That's
why, as one of the global leaders in the biopharmaceutical industry,
Pfizer is committed to seeking out inspired new talent who share our
core values and mission of making the world a healthier place.

\subsection{Role Description}\label{role-description}

As part of the Computational Sciences Center of Emphasis, apply and
possibly develop advanced analysis and mining approaches to inform
target / biomarker identification and compound selection. The successful
candidate will be responsible for performing \textbf{integrative
analysis} relating molecular entities, in-vitro/in-vivo/clinical
biological activities, and complex phenotypes in a quantitative way and
deliver testable hypotheses in high quality presentations/publications.
The ideal candidate will have demonstrated expertise with data mining
methods upon large-scale, multidimensional molecular datasets, and
possess expertise in mining structured data and unstructured/text
information and be able to identify gaps in current methodologies and
define requirements for solutions to address these gaps.

\subsection{Responsibilities}\label{responsibilities}

\begin{itemize}
\itemsep1pt\parskip0pt\parsep0pt
\item
  Perform computational analysis efforts of high-dimensional datasets
  from model systems, experimental medicine and clinical trials to
  identify novel targets, biomarkers, mechanisms of resistance, and
  effective therapeutic combinations through the translation of
  bioinformatics data.
\item
  Deliver testable hypotheses/insights from complex multidimenstional
  data to inform target/compound selection, biomarker identification and
  patient stratification in high quality presentations \& publications.
\item
  Identify and validate innovative approaches to improve quality and
  efficiency of hypothesis generation from experimental data\\
\item
  Maintain current awareness of emerging approaches and methods in
  computational biology and provide ad-hoc support to cross-disciplinary
  project teams.
\end{itemize}

\subsection{Qualifications}\label{qualifications}

Education and Experience: PhD in Computer Science, Mathematics,
Statistics, Biological Sciences, relevant natural sciences required, 1-2
years relevant experience applying quatitative approaches to solving
biological problems, ideally in a pharmaceutical, biotech or comparable
context.

\subsection{Technical Skills:}\label{technical-skills}

\begin{itemize}
\itemsep1pt\parskip0pt\parsep0pt
\item
  Demonstrated expertise in delivering insights/hypotheses from complex
  multi-dimensional biological data in a biomedical context.
\item
  Demonstrated experience applying, defining and validating
  computational approaches to deliver insights/hypotheses, e\_g\_
  multivariate, Bayesian and machine learning approaches.
\item
  Pharmaceutically relevant experience or formal training in
  computational chemistry/biology, computer science, physics, applied
  mathematics, or statistics
\item
  Demonstrated ability for sound experimental design for in-silico
  experimentation/workflows required, in addition to ability to
  effectively interface with Research Unit biologist to
  communicate/discuss results, ideas, and follow-up experiments.
\item
  In depth knowledge of relevant public and proprietary databases,
  methods and tools.
\item
  Exceptional communication skills (oral and written) as demonstrated by
  publications \& presentations.
\end{itemize}

\section{Resume}\label{resume}

\subsection{Contact information}\label{contact-information}

W. Augustine Dunn III US Citizen 132 Nicoll St FL 1 New Haven CT, 06511
email: gus.dunn@yale.edu \textbar{} primary phone: 770-312-9544

\subsection{Objective or Summary
(optional)}\label{objective-or-summary-optional}

\subsection{HIGHLIGHTS/SKILLS/QUALIFICATIONS/PROFILE}\label{highlightsskillsqualificationsprofile}

\subsubsection{Qualifications}\label{qualifications-1}

\subsubsection{Technical expertise}\label{technical-expertise}

\subsection{Education}\label{education}

Degree: Doctor of Philosophy in Biological Sciences\\University:
University of California, Irvine Irvine, CA\\Graduation: March
2014\\Adviser: Dr.~Anthony James\\Co-Adviser: Dr.~Xiaohui Xie\\Topic:
\emph{Comparative transcriptomics of blood-feeding in midguts of three
disease-vector mosquito species.}

Degree: Bachelor of Science\\University: University of Georgia Athens,
GA\\Graduation: 2003\\Major: Biology

\subsection{Experience}\label{experience}

\subsection{Awards/Honors/Fellowships/Etc}\label{awardshonorsfellowshipsetc}

\begin{itemize}
\item
  \texttt{2011} The Pacific-Southwest Regional Center of Excellence for
  Biodefense and Emerging Infectious Diseases Annual Meeting Travel
  Award
\item
  \texttt{2010/2011} President of UCI's IGB Biomedical Informatics
  Training fellows
\item
  \texttt{2009-2012} Biomedical Informatics Training fellow (NIH/NLM
  5T15LM007443)
\end{itemize}

\subsection*{Activities/Interests}\label{activitiesinterests}
\addcontentsline{toc}{subsection}{Activities/Interests}

\end{document}
