\documentclass[letterpaper]{scrartcl}
\usepackage{lmodern}
\usepackage{amssymb,amsmath}
\usepackage{ifxetex,ifluatex}
\usepackage{fixltx2e} % provides \textsubscript
\ifnum 0\ifxetex 1\fi\ifluatex 1\fi=0 % if pdftex
  \usepackage[T1]{fontenc}
  \usepackage[utf8]{inputenc}
\else % if luatex or xelatex
  \ifxetex
    \usepackage{mathspec}
    \usepackage{xltxtra,xunicode}
  \else
    \usepackage{fontspec}
  \fi
  \defaultfontfeatures{Mapping=tex-text,Scale=MatchLowercase}
  \newcommand{\euro}{€}
\fi
% use upquote if available, for straight quotes in verbatim environments
\IfFileExists{upquote.sty}{\usepackage{upquote}}{}
% use microtype if available
\IfFileExists{microtype.sty}{%
\usepackage{microtype}
\UseMicrotypeSet[protrusion]{basicmath} % disable protrusion for tt fonts
}{}
\usepackage[margin=1in]{geometry}
\usepackage{graphicx}
\makeatletter
\def\maxwidth{\ifdim\Gin@nat@width>\linewidth\linewidth\else\Gin@nat@width\fi}
\def\maxheight{\ifdim\Gin@nat@height>\textheight\textheight\else\Gin@nat@height\fi}
\makeatother
% Scale images if necessary, so that they will not overflow the page
% margins by default, and it is still possible to overwrite the defaults
% using explicit options in \includegraphics[width, height, ...]{}
\setkeys{Gin}{width=\maxwidth,height=\maxheight,keepaspectratio}
\ifxetex
  \usepackage[setpagesize=false, % page size defined by xetex
              unicode=false, % unicode breaks when used with xetex
              xetex]{hyperref}
\else
  \usepackage[unicode=true]{hyperref}
\fi
\hypersetup{breaklinks=true,
            bookmarks=true,
            pdfauthor={Gus Dunn},
            pdftitle={ddRAD Phase 2},
            colorlinks=true,
            citecolor=blue,
            urlcolor=blue,
            linkcolor=magenta,
            pdfborder={0 0 0}}
\urlstyle{same}  % don't use monospace font for urls
\setlength{\parindent}{0pt}
\setlength{\parskip}{6pt plus 2pt minus 1pt}
\setlength{\emergencystretch}{3em}  % prevent overfull lines
\setcounter{secnumdepth}{5}

\title{ddRAD Phase 2}
\author{Gus Dunn}
\date{2015-04-30}
\usepackage{fontspec}
\setmainfont{Linux Libertine O}

% blockquote


\begin{document}
\maketitle

\section{Main costs and limiting
factors}\label{main-costs-and-limiting-factors}

After talking with AndreaG and JoshuaR, I believe that the main cost
determinate will simply be the sequencing. There is plenty of the
special primer material still available which would be the other main
cost. The reagents, as the PLOS paper points out are very cheap: NEB
enzymes/buffers plus high-fi Taq reagents.

The current cost to do a similar type of sequencing strategy according
to the
\href{http://medicine.yale.edu/keck/ycga/services/illuminaprices.aspx}{YCGA
website} is \$1887 for HiSeq 2000 or 2500: 75 bp, paired-end sequencing
(1 lane, 2 x 75).

\section{How many individuals to do}\label{how-many-individuals-to-do}

AndreaG did 24 per lane, and according to the results in our current
paper:

\begin{quote}
Downstream analysis of the ddRAD reads indicates that we achieved an
average of 32X coverage for each individual fly.
\end{quote}

I am going to refer to the depth-coverage per individual as DCI from now
on.

Our paper doesn't yet mention how many RAD-regions (genomic chunks) we
got. It probably should as this is talked about in the PLOS ddRAD paper
and is related to the proportion of the genome we can \emph{see} in our
data. I asked AndreaG, and she didn't remember of the top of her head
but was going to look it up. However, since we do not intend to deviate
from her protocol, I only bring it up in the interest of completeness.

At current prices and using AndreaG's 24 individuals per plate, 5 lanes
would cost \$9,435 and yield 120 individuals. If we could tolerate lower
than 32X DCI, we can expand the number of individuals. Assuming a linear
relationship between individuals per lane and DCI: allowing our DCI to
drop to 19.2X (40 flies per lane) would increase our yield to 200
individuals: a DCI of 15.36X (50 flies per lane) yields 250 individuals.
And of course adding another lane would bump that to 300 and change the
cost to \$11,322.

Since we want to shoot for at least \textasciitilde{}100 infected
individuals, we should plan to sequence \emph{at least}
\textasciitilde{}200 flies. We can do that easily as long as we can
tolerate DCI \textasciitilde{}15X and/or add an additional lane(s).

\section{Minimum depth-coverage per
individual}\label{minimum-depth-coverage-per-individual}

I am still looking for a theoretical minimal DCI threshold needed in
order to draw solid conclusions from our data. I am opening up
correspondence with Deren Eaton (pyrad -- upstairs neighbor) and
probably Brant K. Peterson (corresponding author on the PLOS paper)
regarding this. However, I do not think that 15X is very close to this
limit. Especially since we will have \textasciitilde{}15X * 200 or more
individuals when it is all said and done.

\section{Planing collecting efforts}\label{planing-collecting-efforts}

I am worried that our existing specimens stores from the south may not
represent comparable breadth or depth as compared to our existing
specimens from the northern and to some degree western stores,
particularly when considering infections. I think that as the project
continues we need to insist on some level of reconnaissance protocols
(regularly executed low-volume surveying expeditions that cover a wide
swath of our population landscapes) that will allow us to \emph{swiftly}
(\textbf{in the same season}) deploy high-volume collection expeditions
to regions where sufficient infections are detected.

I am double checking our numbers, but I would currently focus that type
of reconnaissance in the north-west and south and send high-volume
expeditions to the one that looks most promising for this season. Then
repeat or adjust recon focus-regions for Fall according to to the
results of the Summer.

\section{ddRAD phase 2 experimental
goals}\label{ddrad-phase-2-experimental-goals}

Once we decide how many flies we plan to sequence, I will propose to you
the populations and fly numbers to represent each to make sure that the
project's goals are covered. From there, I will give you a set of
milestones and completion goals. After I run through the process in my
own hands and get the preliminary quality control results from the
single and double digestion trails, I will adjust the completion goals
based on real-life.

\end{document}
