\documentclass[letterpaper]{report}
\usepackage{bookman}
\usepackage{amssymb,amsmath}
\usepackage{ifxetex,ifluatex}
\usepackage{fixltx2e} % provides \textsubscript
\ifnum 0\ifxetex 1\fi\ifluatex 1\fi=0 % if pdftex
  \usepackage[T1]{fontenc}
  \usepackage[utf8]{inputenc}
\else % if luatex or xelatex
  \ifxetex
    \usepackage{mathspec}
    \usepackage{xltxtra,xunicode}
  \else
    \usepackage{fontspec}
  \fi
  \defaultfontfeatures{Mapping=tex-text,Scale=MatchLowercase}
  \newcommand{\euro}{€}
\fi
% use upquote if available, for straight quotes in verbatim environments
\IfFileExists{upquote.sty}{\usepackage{upquote}}{}
% use microtype if available
\IfFileExists{microtype.sty}{%
\usepackage{microtype}
\UseMicrotypeSet[protrusion]{basicmath} % disable protrusion for tt fonts
}{}
\usepackage{graphicx}
\makeatletter
\def\maxwidth{\ifdim\Gin@nat@width>\linewidth\linewidth\else\Gin@nat@width\fi}
\def\maxheight{\ifdim\Gin@nat@height>\textheight\textheight\else\Gin@nat@height\fi}
\makeatother
% Scale images if necessary, so that they will not overflow the page
% margins by default, and it is still possible to overwrite the defaults
% using explicit options in \includegraphics[width, height, ...]{}
\setkeys{Gin}{width=\maxwidth,height=\maxheight,keepaspectratio}
\ifxetex
  \usepackage[setpagesize=false, % page size defined by xetex
              unicode=false, % unicode breaks when used with xetex
              xetex]{hyperref}
\else
  \usepackage[unicode=true]{hyperref}
\fi
\hypersetup{breaklinks=true,
            bookmarks=true,
            pdfauthor={Robert Opiro, Augustine Dunn},
            pdftitle={Tsetse R01 Progress Report},
            colorlinks=true,
            citecolor=blue,
            urlcolor=blue,
            linkcolor=magenta,
            pdfborder={0 0 0}}
\urlstyle{same}  % don't use monospace font for urls
\setlength{\parindent}{0pt}
\setlength{\parskip}{6pt plus 2pt minus 1pt}
\setlength{\emergencystretch}{3em}  % prevent overfull lines
\setcounter{secnumdepth}{0}

\title{Tsetse R01 Progress Report\\\vspace{0.5em}{\large Sampling and Databasing}}
\author{Robert Opiro, Augustine Dunn}
\date{October 14, 2014}

\begin{document}
\maketitle

{
\hypersetup{linkcolor=black}
\setcounter{tocdepth}{3}
\tableofcontents
}
\section{Sampling}\label{Sampling}

\subsection{Study sites}\label{study-sites}

The surveys were done in the districts of Kole, Oyam, Nwoya, Amuru,
Adjumani, Moyo and Arua in northern/Northwest Uganda. Additional
information on tsetse population distribution was obtained from the
District Entomology Offices of the relevant districts.

\subsection{Data collection}\label{data-collection}

Trapping for tsetse flies were carried out using biconicals traps
({Challier, A and Laveissiere} 1973). The coordinates for each trap site
were taken using a hand-held GPS. Vegetation types and human activities
at the trapping sites were also recorded. Each village is at least 5km
apart; a single village is taken to be a trapping site (with a number of
traps deployed in each).

\subsection{Dissection and
examination}\label{dissection-and-examination}

Trapped flies were identified, sexed, counted, recorded and transported
to a field dissection site. Live flies were dissected and examined
microscopically to determine the presence/absence of trypanosomes in the
midguts/salivary glands. The midguts, fly carcass, reproductive parts,
and heads were then preserved in parafilm-sealed and labeled cryo-tubes
in either 90\% ethanol or RNA-preservation solution for further
molecular studies.

\section{Results}\label{results}

\subsection{Kole District
(\texttt{2014-03-22 to 2014-03-30})}\label{kole-district-2014-03-22-to-2014-03-30}

Five villages were surveyed (Olepo {[}OLE{]}, Mwanya {[}MWA{]},
Akayo-debe {[}AKA{]}, Aputu-Lwaa {[}APU{]}, and Ocala {[}OCA{]}) with a
total of 40 traps. 1227 \emph{Gff} were captured (564 M and 663 F) and
yielded five infected individuals (1.2\% estimated infection rate).

\subsection{Oyam District
(\texttt{2014-05-17 to 2014-05-22})}\label{oyam-district-2014-05-17-to-2014-05-22}

Nine villages were surveyed (Ocala {[}OCA{]}, Odworo {[}OD{]}, Alege
{[}ALE{]}, Acankoma {[}ACA{]}, Oguk {[}OGU{]}, Agoba B {[}AG{]},
Abok{[}ABO{]}, Ocol {[}OCL{]} and Opuyu {[}OPU{]}) with 32 traps. 715
\emph{Gff} were captured (298 M and 417 F) and yielded 10 infected
individuals (3.0\% estimated infection rate).

\subsection{Oyam and Kole Districts
(\texttt{2014-07-14 to 2014-07-21})}\label{oyam-and-kole-districts-2014-07-14-to-2014-07-21}

This survey targeted sites that produced infected flies from the
previous surveys. The field team deployed 27 traps across four villages
that were divided between the two districts:

\textbf{Oyam:}

\begin{itemize}
\itemsep1pt\parskip0pt\parsep0pt
\item
  Ocala {[}OCA{]}
\item
  Odworo {[}OD{]}
\item
  Acankoma {[}ACA{]}
\end{itemize}

\textbf{Kole:}

\begin{itemize}
\itemsep1pt\parskip0pt\parsep0pt
\item
  Akayodebe {[}AKA{]}
\end{itemize}

\includegraphics{/home/gus/Dropbox/uganda data/2014_Collection_Sheets_Spring-Summer/2014_full_surveyreport_20140820/Village_map_2014-10-21T20:42:32Z.png}\{\#id\}

\section*{References}\label{references}
\addcontentsline{toc}{section}{References}

{Challier, A and Laveissiere}, C. 1973. ``Un nouveau pie'ge pour la
capture des glossines (Glossina: Diptera, Muscidae): description et
essais sur le terrain.'' \emph{Cah ORSTOM Ser Ent Med Parasitol} 11:
251--62.

\end{document}
