\documentclass[letterpaper]{report}
\usepackage{bookman}
\usepackage{amssymb,amsmath}
\usepackage{ifxetex,ifluatex}
\usepackage{fixltx2e} % provides \textsubscript
\ifnum 0\ifxetex 1\fi\ifluatex 1\fi=0 % if pdftex
  \usepackage[T1]{fontenc}
  \usepackage[utf8]{inputenc}
\else % if luatex or xelatex
  \ifxetex
    \usepackage{mathspec}
    \usepackage{xltxtra,xunicode}
  \else
    \usepackage{fontspec}
  \fi
  \defaultfontfeatures{Mapping=tex-text,Scale=MatchLowercase}
  \newcommand{\euro}{€}
\fi
% use upquote if available, for straight quotes in verbatim environments
\IfFileExists{upquote.sty}{\usepackage{upquote}}{}
% use microtype if available
\IfFileExists{microtype.sty}{%
\usepackage{microtype}
\UseMicrotypeSet[protrusion]{basicmath} % disable protrusion for tt fonts
}{}
\usepackage[margin=1in]{geometry}
\usepackage{longtable,booktabs}
\usepackage{graphicx}
\makeatletter
\def\maxwidth{\ifdim\Gin@nat@width>\linewidth\linewidth\else\Gin@nat@width\fi}
\def\maxheight{\ifdim\Gin@nat@height>\textheight\textheight\else\Gin@nat@height\fi}
\makeatother
% Scale images if necessary, so that they will not overflow the page
% margins by default, and it is still possible to overwrite the defaults
% using explicit options in \includegraphics[width, height, ...]{}
\setkeys{Gin}{width=\maxwidth,height=\maxheight,keepaspectratio}
\ifxetex
  \usepackage[setpagesize=false, % page size defined by xetex
              unicode=false, % unicode breaks when used with xetex
              xetex]{hyperref}
\else
  \usepackage[unicode=true]{hyperref}
\fi
\hypersetup{breaklinks=true,
            bookmarks=true,
            pdfauthor={Robert Opiro, Augustine Dunn},
            pdftitle={Tsetse R01 Progress Report},
            colorlinks=true,
            citecolor=blue,
            urlcolor=blue,
            linkcolor=magenta,
            pdfborder={0 0 0}}
\urlstyle{same}  % don't use monospace font for urls
\setlength{\parindent}{0pt}
\setlength{\parskip}{6pt plus 2pt minus 1pt}
\setlength{\emergencystretch}{3em}  % prevent overfull lines
\setcounter{secnumdepth}{0}

\title{Tsetse R01 Progress Report\\\vspace{0.5em}{\large Sampling and Databasing}}
\author{Robert Opiro, Augustine Dunn}
\date{2014-10-30}

\begin{document}
\maketitle

{
\hypersetup{linkcolor=black}
\setcounter{tocdepth}{3}
\tableofcontents
}
\section{Specimen database}\label{specimen-database}

A multitude of data is associated with each fly collected. The type of
information includes the list of tissues collected, the collection date,
village name and location, sex, species, trap number, infection status,
notes on human activity surrounding the trap, and more. All of this
information needs to remain tied to all specimens, materials, and
\emph{data} derived from these as the project goes forward.

We are designing and implementing a custom database and web application
to manage, track, and facilitate analysis of the thousands of specimen
tubes associated with this project that will be generated and exist
already. The system consists of a web-based user interface, two
SQL-based relational databases, and a layer of custom python functions
that connect the two. The web-based interface uses the open-source
\href{http://getbootstrap.com/}{Bootstrap} web interface components. One
SQL database will act as the official storage system for the specimen
data, while the second will manage checkout requests by our researchers
and update the main storage database upon validation. The custom python
code is based on the open-source
\href{https://github.com/mitsuhiko/flask}{Flask} web-microframework.
This code manages the two databases according to requests made through
the web-interface. It also manages user registration and permissions
along with site-security. Finally, it will also allow us to easily
design and run complex analyses with the specimen data encoded in the
main database.

\section{Collections Overview}\label{collections-overview}

\subsection{Study sites}\label{study-sites}

The surveys were done in the Ugandan districts of Kole, Oyam, Nwoya,
Amuru, Adjumani, Moyo, Arua, Kitgum, Lamwo, and Pader. Additional
information on tsetse population distribution was obtained from the
District Entomology Offices of the relevant districts.

\subsection{Data collection}\label{data-collection}

Trapping for tsetse flies were carried out using biconicals traps
({Challier, A and Laveissiere} 1973). The coordinates for each trap site
were taken using a hand-held GPS. Vegetation types and human activities
at the trapping sites were also recorded. Each village is at least 5km
apart; a single village is taken to be a trapping site (with a number of
traps deployed in each).

\subsection{Dissection and
examination}\label{dissection-and-examination}

Trapped flies were identified, sexed, counted, recorded and transported
to a field dissection site. Live flies were dissected and examined
microscopically to determine the presence/absence of trypanosomes in the
midguts/salivary glands. The midguts, fly carcass, reproductive parts,
and heads were then preserved in parafilm-sealed and labeled cryo-tubes
in either 90\% ethanol or RNA-preservation solution for further
molecular studies.

\section{Collection Results Summary}\label{collection-results-summary}

\subsection{Kole District
(\texttt{2014-03-22 to 2014-03-30})}\label{kole-district-2014-03-22-to-2014-03-30}

Five villages were surveyed (Olepo {[}OLE{]}, Mwanya {[}MWA{]},
Akayo-debe {[}AKA{]}, Aputu-Lwaa {[}APU{]}, and Ocala {[}OCA{]}) with a
total of 40 traps. 1227 \emph{Gff} were captured (564 M and 663 F) and
yielded five infected individuals (1.2\% estimated infection rate).

\subsection{Oyam District
(\texttt{2014-05-17 to 2014-05-22})}\label{oyam-district-2014-05-17-to-2014-05-22}

Nine villages were surveyed (Ocala {[}OCA{]}, Odworo {[}OD{]}, Alege
{[}ALE{]}, Acankoma {[}ACA{]}, Oguk {[}OGU{]}, Agoba B {[}AG{]},
Abok{[}ABO{]}, Ocol {[}OCL{]} and Opuyu {[}OPU{]}) with 32 traps. 715
\emph{Gff} were captured (298 M and 417 F) and yielded 10 infected
individuals (3.0\% estimated infection rate).

\subsection{Oyam and Kole Districts
(\texttt{2014-07-14 to 2014-07-21})}\label{oyam-and-kole-districts-2014-07-14-to-2014-07-21}

This survey targeted sites that produced infected flies from the
previous surveys. The field team deployed 27 traps across four villages
that were divided between the two districts: \textbf{Oyam:} (Ocala
{[}OCA{]}, Odworo {[}OD{]}, Acankoma {[}ACA{]}) and \textbf{Kole:}
(Akayodebe {[}AKA{]}).

1198 \emph{Gff} were captured (432 M and 766 F) and yielded 27 infected
individuals (4.38\% estimated infection rate).

\subsection{Nwoya District
(\texttt{2014-07-22 to 2014-07-26})}\label{nwoya-district-2014-07-22-to-2014-07-26}

The field team deployed 20 traps across two villages (the Uganda
Wildlife Authority {[}UWA{]} and Te-Okot {[}TEO{]}). \emph{Gp} and
\emph{Gmm} were trapped in this region in addtion to \emph{Gff}; however
only the data for \emph{Gff} is reported here. 728 \emph{Gff} were
captured (291 M and 437 F) and three were positive by microscopic
examination.

\subsection{Amuru District
(\texttt{2014-07-27 to 2014-07-29})}\label{amuru-district-2014-07-27-to-2014-07-29}

Two villages were surveyed (Gorodona {[}GOR{]} and Okidi south
{[}OKS{]}) using 18 traps. 243 \emph{Gff} were captured (67 M and 176 F)
and yielded no infected individuals.

\subsection{Adjumani District
(\texttt{2014-07-30 to 2014-08-02})}\label{adjumani-district-2014-07-30-to-2014-08-02}

Three villages were surveyed (Olobo {[}OLO{]}, Olwi {[}OLW{]}, Osugo
East and West {[}OSG{]}) with 20 traps. 182 \emph{Gff} were captured (60
M and 122 F) and yielded no infected individuals.

\subsection{Moyo District
(\texttt{2014-06-16 to 2014-06-20})}\label{moyo-district-2014-06-16-to-2014-06-20}

Five villages were surveyed (Ori {[}ORI{]}, Orubakulem {[}ORB{]}, Lea
{[}LEA{]}, Cefo {[}CE{]},and Moyipi {[}MOP{]}) with 32 traps. 164
\emph{Gff} were captured (63 M and 101 F) and yielded no infected
individuals.

\subsection{Arua District
(\texttt{2014-06-21 to 2014-06-26})}\label{arua-district-2014-06-21-to-2014-06-26}

Seven villages were surveyed (Gangu {[}GAN{]}, Aliodri {[}ALI{]}, Jaiko
{[}JIA{]}, Duku {[}DUK{]}, Wende {[}WEN{]}, Aina {[}AIN{]}, and Orivu
{[}ORV{]}) with 34 traps. 681 \emph{Gff} were captured (287 M and 394 F)
and yielded three infected individuals (0.87\% estimated infection
rate).

\subsection{Kitgum, Lamwo, and Pader Districts
(\texttt{2014-10-06 to 2014-10-19})}\label{kitgum-lamwo-and-pader-districts-2014-10-06-to-2014-10-19}

In the three districts combined, 534 \emph{Gff} were captured (193 M and
341 F). 59 traps were deployed across 14 villiages. 330 flies were
dissected and 5 were found to be infected (1.52\% combined estimated
infection rate)

\subsubsection{Kitgum}\label{kitgum}

Four villages were surveyed (Kitgum town council {[}KTC{]}, Liba
{[}LIB{]}, Bola {[}BOL{]}, Tumangu {[}TUM{]}) with 18 traps. 281
\emph{Gff} were captured (120 M and 161 F), and 173 dissected. Four
infected individuals were detected (2.31\% estimated infection rate).

\subsubsection{Lamwo}\label{lamwo}

Four villages were surveyed (Lagwel {[}LAG{]}, Ngomoromo {[}NGO{]},
Pawor {[}PAW{]}, Lakwala {[}LAK{]}) with 15 traps. 101 \emph{Gff} were
captured (37 M and 64 F), and 48 dissected. Zero infected individuals
were detected.

\subsubsection{Pader}\label{pader}

Six villages were surveyed (Alim {[}ALI{]}, Chua {[}CHU{]}, Kilak
{[}KIL{]}, Aswa Bridge {[}ASW{]}, Omido {[}OMI{]}, Atanga Mission
{[}ATM{]}) with 26 traps. 152 \emph{Gff} were captured (39 M and 113 F),
and 109 dissected. One infected individual was detected (0.92\%
estimated infection rate).

\begin{longtable}[c]{@{}ccccccccc@{}}
\toprule
District & Collection~End Date & Villages & Traps & Flies~Collected &
Males & Females & Flies~Dissected & Flies~Infected\tabularnewline
\midrule
\endhead
Kole & 2014-03-30 & 5 & 40 & 1227 & 564 & 663 & 428 & 5\tabularnewline
Oyam & 2014-05-22 & 9 & 32 & 715 & 298 & 417 & 336 & 10\tabularnewline
Kole \& Oyam & 2014-07-21 & 4 & 27 & 1198 & 432 & 766 & 617 &
27\tabularnewline
Nwoya & 2014-07-26 & 2 & 20 & 728 & 291 & 437 & 252 & 3\tabularnewline
Amuru & 2014-07-29 & 2 & 18 & 243 & 67 & 176 & 140 & 0\tabularnewline
Adjumani & 2014-08-02 & 3 & 20 & 182 & 60 & 122 & 120 & 0\tabularnewline
Moyo & 2014-06-20 & 5 & 32 & 164 & 63 & 101 & 106 & 0\tabularnewline
Arua & 2014-06-26 & 7 & 34 & 681 & 287 & 394 & 346 & 3\tabularnewline
Kitgum & 2014-10-19 & 4 & 18 & 281 & 120 & 161 & 173 & 4\tabularnewline
Lamwo & 2014-10-19 & 4 & 15 & 101 & 37 & 64 & 48 & 0\tabularnewline
Pader & 2014-10-19 & 6 & 26 & 152 & 39 & 113 & 109 & 1\tabularnewline
\bottomrule
\end{longtable}

\section*{References}\label{references}
\addcontentsline{toc}{section}{References}

{Challier, A and Laveissiere}, C. 1973. ``Un nouveau pie'ge pour la
capture des glossines (Glossina: Diptera, Muscidae): description et
essais sur le terrain.'' \emph{Cah ORSTOM Ser Ent Med Parasitol} 11:
251--62.

\end{document}
