\documentclass[]{article}
\usepackage[T1]{fontenc}
\usepackage{lmodern}
\usepackage{amssymb,amsmath}
\usepackage{ifxetex,ifluatex}
\usepackage{fixltx2e} % provides \textsubscript
% use upquote if available, for straight quotes in verbatim environments
\IfFileExists{upquote.sty}{\usepackage{upquote}}{}
\ifnum 0\ifxetex 1\fi\ifluatex 1\fi=0 % if pdftex
  \usepackage[utf8]{inputenc}
\else % if luatex or xelatex
  \ifxetex
    \usepackage{mathspec}
    \usepackage{xltxtra,xunicode}
  \else
    \usepackage{fontspec}
  \fi
  \defaultfontfeatures{Mapping=tex-text,Scale=MatchLowercase}
  \newcommand{\euro}{€}
\fi
% use microtype if available
\IfFileExists{microtype.sty}{\usepackage{microtype}}{}
\usepackage[a4paper]{geometry}
\ifxetex
  \usepackage[setpagesize=false, % page size defined by xetex
              unicode=false, % unicode breaks when used with xetex
              xetex]{hyperref}
\else
  \usepackage[unicode=true]{hyperref}
\fi
\hypersetup{breaklinks=true,
            bookmarks=true,
            pdfauthor={},
            pdftitle={},
            colorlinks=true,
            citecolor=blue,
            urlcolor=blue,
            linkcolor=magenta,
            pdfborder={0 0 0}}
\urlstyle{same}  % don't use monospace font for urls
\setlength{\parindent}{0pt}
\setlength{\parskip}{6pt plus 2pt minus 1pt}
\setlength{\emergencystretch}{3em}  % prevent overfull lines
\setcounter{secnumdepth}{0}

\author{}
\date{}

\begin{document}

\section{Markx}\label{markx}

\subsection{Help}\label{help}

Last update: 26 Feb 2013

\href{https://github.com/yoavram/markx}{Markx} is a Markdown editor
specifically designed for academic and scientific authoring. It
highlights several advantages of Markdown (plain-text, multiple format
conversion, easy version control) while still supplying the basic
features which are a must for academic publications (citations, math
rendering, layouts).

\subsubsection{Writing Markdown}\label{writing-markdown}

\href{http://daringfireball.net/projects/markdown/}{Markdown} is a
markup language.

Markdown is very easy to learn and there are many online tutorials, just
use Google.

\paragraph{Markdown Flavor}\label{markdown-flavor}

You should use \textbf{Pandoc-flavored Markdown} as both converters
(\href{http://johnmacfarlane.net/pandoc}{Pandoc} and
\href{http://www.docverter.com/}{Docverter} use that). The HTML preview
on the right is processed using
\href{http://code.google.com/p/pagedown/}{PageDown}, so there could be
some thing it doesn't process like Pandoc does (found something? open an
\href{https://github.com/yoavram/markx/issues}{issue}).

\paragraph{Math}\label{math}

You can use $LaTeX$. Just write it between \texttt{\$}s or
\texttt{\textbackslash{}\textbackslash{}(} and
\texttt{\textbackslash{}\textbackslash{})} for inline equations and
\texttt{\$\$}s or \texttt{\textbackslash{}\textbackslash{}{[}} and
\texttt{\textbackslash{}\textbackslash{}{]}} for display mode:

\begin{itemize}
\itemsep1pt\parskip0pt\parsep0pt
\item
  Inline mode $e=mc^2$
\item
  Display mode:
  \[\frac{df(x)}{dt}=lim_{x \to 0}{\frac{f(x+h)-f(x)}{h}}\]
\end{itemize}

\paragraph{Citations}\label{citations}

To start working with citations in Markx you need to click the
\emph{Load a Bibliography file} from the \emph{Citations} menu. The file
should be in \href{http://www.bibtex.org/}{BibTeX} format - if you don't
have one than Markx doesn't support your citation format yet, but you
might be able to convert a different format to BibTeX using online
tools. If you use Mendeley it is easy to set it up to
\href{http://blog.mendeley.com/tipstricks/howto-use-mendeley-to-create-citations-using-latex-and-bibtex/}{sync
to a BibTeX file}. If you do not load a \texttt{.bib} file you will not
be able to use the citations features of Markx.

To insert a citation, find the citation key - usually the last name of
the first author, with a capital initial, and the year of publication,
without spaces. If the \texttt{.bib} file has more than one publication
with that key they are post-fixed with lowercase letters. Then add the
citation key to the editor, wrapped by \texttt{{[}@} and \texttt{{]}}.
For example: \texttt{{[}@Drake1991{]}}. Markx doesn't currently preview
the citation keys in the previewed text, but it does:

\begin{enumerate}
\def\labelenumi{\arabic{enumi}.}
\itemsep1pt\parskip0pt\parsep0pt
\item
  create a bibliography at the bottom of the preview text
\item
  allows you to download a \texttt{.bib} file corresponding to the
  citation keys in the Markdown text via the \emph{Download} menu
\item
  send the bibliography to
  \href{http://johnmacfarlane.net/pandoc}{Pandoc} for conversion
  (\href{http://www.docverter.com/}{Docverter} does not support
  citations)
\end{enumerate}

You must click the \emph{Update Citations} button in the
\emph{Citations} menu after adding, removing or changing citation keys,
as they will not be updated in real-time (this will be changed in the
near future).

\subsubsection{Toolbar}\label{toolbar}

\begin{enumerate}
\def\labelenumi{\arabic{enumi}.}
\itemsep1pt\parskip0pt\parsep0pt
\item
  Use the \emph{GitHub} button to \textbf{sign-in to
  \href{https://github.com/}{GitHub}} (see more details below).
\item
  Use the \emph{Screen} button to change between \textbf{editor},
  \textbf{preview} and \textbf{dual} modes.
\item
  Use the \emph{Books} menu to \textbf{parse citation keys} such as
  {[}@Drake1991{]}, to load a bibliography file, to see the loaded
  bibliography and the references you are citing.
\item
  Use the \emph{Download} menu to \textbf{download and convert} the
  Markdown text to various formats or to download a
  \href{http://daringfireball.net/projects/markdown/}{Markdown} of the
  text or a \href{http://www.bibtex.org/}{BibTeX} file of the citations
  referenced in the text.
\item
  Click on \textbf{P} or \textbf{D} to \textbf{change the Markdown
  converter} between \href{http://johnmacfarlane.net/pandoc}{Pandoc} and
  \href{http://www.docverter.com/}{Docverter}.
\end{enumerate}

\begin{itemize}
\itemsep1pt\parskip0pt\parsep0pt
\item
  Pandoc: must be installed on local machine when working locally, can't
  process image URLs, slow conversion to PDF on Windows, requires
  \emph{pdflatex} to convert to PDF on local machine.
\item
  Docverter: must be connected to the internet to be used, doesn't
  process citation keys and bibliography.
\end{itemize}

\begin{enumerate}
\def\labelenumi{\arabic{enumi}.}
\itemsep1pt\parskip0pt\parsep0pt
\item
  Click the \emph{Code} button to get change the \textbf{code
  highlighting styles}. Example code above.
\item
  The grey boxes with the numbers display the \textbf{word and character
  counts}.
\end{enumerate}

\subsubsection{GitHub Integration}\label{github-integration}

\textbf{Your \href{https://github.com/}{GitHub} username and password
are never sent to the Markx server}. They are sent by JavaScript to
directly to the GitHub API server using
\href{https://github.com/michael/github}{Github.js}. Your credentials
are not saved in cookies and are removed from the browser memory as soon
as the sign in is complete. You can also sign out of GitHub by clicking
the \emph{sign out} button. If you would like to check the security of
this feature please view the \texttt{signinToGithub} function in
\href{https://github.com/yoavram/markx/blob/master/static/js/markx.js}{markx.js}
and open an \href{https://github.com/yoavram/markx/issues}{issue} if you
find any problems.

After you sign in to GitHub you can use the GitHub toolbar to:

\begin{enumerate}
\def\labelenumi{\arabic{enumi}.}
\itemsep1pt\parskip0pt\parsep0pt
\item
  Click the \emph{GitHub} button - this doesn't do anything right now.
\item
  Click the \emph{Reload} button o reload the repositories.
\item
  \textbf{Choose a repository} that you can pull from and push to (you
  are owner or collaborator)
\item
  Click the first \emph{folder} button to load the \textbf{branch list}
\item
  \textbf{Choose a branch} if the selected repository
\item
  Click the second \emph{folder} button to \textbf{load the files list}
\item
  \textbf{Choose a file} in the selected branch
\item
  Click the \emph{download} button to \textbf{pull the file} to the
  editor. The current contents will be deleted without saving them.
\item
  Click the \emph{upload} button to \textbf{push the editor contents} to
  the selected file. This will create a new \emph{commit} on the
  repository. You must \textbf{fill a commit message} before pushing.
  Commit messages should be \textasciitilde{}50 characters and briefly
  explain the reason for this commit. After the push is finalized you
  will get a success or failure message. If you have citations, Markx
  will offer to push a bibliography file as well.
\item
  Click the \emph{new} button to \textbf{create and push} a new empty
  file. If you don't fill the commit message Markx fill create adefault
  message for you. After the push is finalized you will get a success or
  failure message and the files list will be updated.
\item
  Click the \emph{sign out} button in the general toolbar to
  \textbf{sign out of GitHub}.
\end{enumerate}

\subsubsection{Support}\label{support}

The best way to get support is to open an
\href{https://github.com/yoavram/markx/issues}{issue}. If you can't open
issue because you don't have a \href{https://github.com/}{GitHub} user,
just get one, they are free.

\subsubsection{Install locally}\label{install-locally}

If the hosted app at \url{http://markx.herokuapp.com} doesn't work for
you (which is possible as it is still rough around the edges) you can
install Markx locally:

\begin{enumerate}
\def\labelenumi{\arabic{enumi}.}
\itemsep1pt\parskip0pt\parsep0pt
\item
  Install Python 2.7.x (may work with other versions of python)
\item
  If you know about \texttt{virtualenv}, you can use the bundled
  \texttt{requirements.txt} file
\item
  Otherwise, install the requirements globally with
  \texttt{pip install flask requests}
\item
  Optionally, install \href{http://johnmacfarlane.net/pandoc}{Pandoc}
\item
  Run Markx with \texttt{python server.py}
\item
  Open your browser at \url{http://localhost:5000}
\end{enumerate}

\subsubsection{Contribution}\label{contribution}

We would love for you to contribute to Markx. The project code is hosted
in \href{https://github.com/yoavram/markx}{GitHub}. Fork the project or
open an \href{https://github.com/yoavram/markx/issues}{issue} so we can
talk on how we can collaborate.

The server side is written in Python with the
\href{http://flask.pocoo.org/}{Flask} web framework (Ruby and Node
equivalents are Sinatra and Express) and
\href{http://python-requests.org/}{requests} for connecting to
\href{http://www.docverter.com/}{Docverter}. The two Markdown converters
are \href{http://johnmacfarlane.net/pandoc}{Pandoc} and
\href{http://www.docverter.com/}{Docverter}, which is a cloud-based
Pandoc.

The client side is written with HTML+CSS+JS, using the JavaScript
libraries:

\begin{enumerate}
\def\labelenumi{\arabic{enumi}.}
\itemsep1pt\parskip0pt\parsep0pt
\item
  \href{http://blog.getbootstrap.com/}{Twitter Bootstrap} and
  {[}jQuery{]} for the UI
\item
  \href{http://code.google.com/p/pagedown/}{PageDown} as the Markdown
  real-time HTML converter
\item
  \href{http://codemirror.net/}{CodeMirror} as the Markdown editor
\item
  \href{https://github.com/michael/github}{Github.js} for the GitHub API
\item
  \href{http://sourceforge.net/projects/jsbibtex/}{Javascript BibTeX
  Parser} processing citation keys and bibliography files
\item
  \href{http://code.google.com/p/google-code-prettify/}{Google Code
  Prettifier} for code highlighting
\item
  \href{http://mathjax.org/}{MathJax} for math rendering
\end{enumerate}

\subsection{References}\label{references}

The references header is \textbf{your} job,
\href{http://johnmacfarlane.net/pandoc}{Pandoc} will only create a
citation list, without a header. However, it will always put it at the
end of the output file (at least via Markx) so you can just put a
\emph{References} header at the end of your file.

\end{document}
